\chapter{Kravspecifikation}
Formål: Et formelt dokument som fortæller hvilke funktioner som det endelige produkt skal have. 

\section{Krav til systemet}
\begin{enumerate}
	\item Systemet skal tilgås ved brug af en internetbrowser, altså skal det være en WebApp.
	\item Systemet skal benytte aktuelle tilbud fra diverse dagligvarebutikker.
	\item Systemet skal have mulighed for at overvåge tilbud på varer valgt af slutbrugeren.
	\item Systemet skal have mulighed for at indeholde en personlig indkøbsliste
	\begin{enumerate}
		\item Integreret med tilbud.
		\item Indkøbslisten deles mellem enheder
		\item Indkøbslisten deles mellem brugere fra samme husstand. 
	\end{enumerate}
	\item Systemet skal kunne præsentere opskrifter for brugere.
	\begin{enumerate}
		\item Systemet skal kunne modtage feedback på de opskrifter brugerne giver.
		\item Systemet skal kunne anbefale opskrifter på baggrund af:
		\begin{enumerate}
			\item Madvaner (varieret kost)
			\item Bedømmelse på opskrift
		\end{enumerate}
	\end{enumerate}
	\item Systemet skal have en række præferencer pr. bruger.
	\begin{enumerate}
		\item Systemet skal kunne fjerne forslag om eksempelvis kød til vegetarer.
	\end{enumerate}
\end{enumerate}
\section{Krav til brugergrænsefladen (“UI”)}
\begin{enumerate}
	\item Systemets UI skal være på dansk.
	\item Systemet skal kunne anvendes på forskellige enheder
	\begin{enumerate}
		\item Computere, > 11” og > 768 px i højden og > 1280 px i breden.
		\item Smartphones, < 6”
	\end{enumerate}
	\item Systemets UI skal være responsivt og tilpasse sig den anvendte platform.
	\item Systemets UI skal anvende design principperne “proximity” og “consistency”.
	\item Systemets UI skal anvende en top-menu til den øverste hierarkiske navigation. 
	\begin{enumerate}	
		\item Top-menuen skal være tilgængelig fra alle sider i applikationen. 
	\end{enumerate}
	\item Systemets UI skal anvende direct mapping, dvs. ikoner somrepræsenterer handlinger (metafor).
	\item Systemet skal tilbyde en fortryd-knap, i henhold til princippet “Effectiveness”.
\end{enumerate}