\chapter{Konklusion}
Projektgruppen valgte opgaven, der omhandlede at hjælpe personer med at handle ind efter tilbud, samt at inspirere i forhold til aftensmåltider.
Problemet opstår i og med, at personer har svært ved at overskue butikkernes tilbud, samt at beslutte sig for hvad aftensmaden skal byde på.
Nogle koordinerer ikke med deres samboende om, hvad de mangler derhjemme, imens andre blot glemmer, hvad de ikke har.
Dette kan resultere i, at forkerte varer bliver købt med hjem.

Gennem projektforløbet har projektgruppen udviklet indkøbsassistenten - ProjectFood. 
Indkøbsassistenten er udviklet som en hjemmeside, der er i stand til at hente faktiske tilbud fra eTilbudsavis’ API og fremvise opskrifter m.m. - Alt sammen integeret med indkøbslister.

Tidligt i udviklingsfasen blev der udført to interviewrunder.  
Den første interviewrunde blev foretaget ved at spørge flere personer, i døråbningen til føtex, omkring deres indkøbs- og spisevaner i henhold til aftensmåltider.
Ud fra den indsamlede information fra interviewrunden samt state of the art undersøgelsen, blev der udarbejdet en prototype for et muligt system, som den anden interviewrunde byggede på. 
Begge interviews bekræftede, at der kunne laves et system, der ville hjælpe brugere med at finde opskrifter til aftensmad og handle ind til disse med tilbud i fokus.  
Resultaterne fra disse interviews ledte ud i et problemområde.
Problemområdet og systemets anvendelsesområde blev analyseret ved metoderne fra OOA\&D.
Denne analyse resulterede i en kravspecifikation bestående af user stories.

Som følge af de funktioner som er tilgængelig i systemet, er de 18 user stories, nævnt i kravspecifikationen (\myref{sec:krav}), forsøgt opfyldt. 

For at sikre programmets kvalitet og holdbarhed er der udført flere forskellige typer af tests.
Der er anvendt black box test i form af unit tests, med en code coverage over 80\%, til at sikre at metoderne gør som forventet. 
Unit testene har også været hjælpsomme til at udbedre bugs. 
Der er ydermere udført to runder af brugertests. 
På tidspunktet for den første brugertestrunde havde hjemmesiden de fleste hovedfunktioner, men disse var ikke finpudset. 
Dette kunne ses i bruger testene af programmet, hvis respons overvejende var positiv, både hvad angik funktionalitet og brugergrænsefladen. 
Feedbacken fra første brugertestrunde blev brugt til at finpudse programmet samt udvikle yderligere funktioner. 
Forbedringen var tydelig fra den første brugertestrunde til den anden.
Størstedelen af problemerne, der blev fundet i første runde, var udbedret i den anden. 
Derudover var den generelle holdning, at hjemmesiden virkede brugervenlig, og at brugerne kunne se sig selv bruge \textsc{ProjectFood}.

Gruppen besluttede, at tjenesten skulle udvikles ved brug af den evolutionære udviklingsmodel. 
Igennem udviklingsfasen indså gruppen, at der manglede organisering af den valgte metode. 
For at løse dette problem blev det valgt at benytte dele af Scrum. 
Scrumboardet og daily scrum var særligt brugbart. 
Scrum hjalp til med at planlægge iterationer vha. product backloggen.
Gruppen havde ikke nogen product owner, men forstod kravene til hjemmesiden fra interviewrunderne.
Derfor kunne gruppen fungere som deres egen product owner for projektet. 
Scrum har hjulpet os til at bruge den evolutionære model, og derved har vi har opnået flere iterationer. 
Delene fra scrum har yderligere ledt til, at mængden af udført arbejde, samt kvaliteten deraf, steg betydeligt.

På baggrund af de ovenstående informationer, konkluderes det, at der er konstrueret en veltestet løsning, som desuden  opfylder alle vores user stories.

