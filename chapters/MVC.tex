\chapter{Model-View-Controller}\fxfatal{Der er ingen kilder på dette afsnit? og linjebrud i kildefilen er volsomt træls. - Troels}\fxfatal{Tror hele dette afsnite skal omskrives til at følge: ``A model represents the state of a particular aspect of the application. A controller handles interactions and updates the model to reflect a change in state of the application, and then passes information to the view. A view accepts necessary information from the controller and renders a user interface to display that information.'' Fra wikipedia: \url{http://en.wikipedia.org/wiki/ASP.NET_MVC_Framework}}
Model-View-Controller(MVC) er et designmønster som fordeler 
objekter\fxnote{Det fordeler logik over 3 lag ... ikke objekter. - Troels} i en applikation ud på tre forskellige roller, 
Model, View og Controller. Hver enkelt rolle er separeret fra 
hinanden, hvilket gør koden lettere at håndtere, teste og strukturere.

\textbf{Model} laget er skjult for brugeren. Dette lag bør 
indeholde det mest grundlæggende data for applikationen, 
dette værende diverse klasser der benyttes, data der bruges 
gentagende gange i applikationen. Denne data kan være gemt i en database eller i lokale filer, og bør indlæses som objekter. Model 
objekter skal repræsenterer viden og erfaring inden for et 
givent problemområde, hvilket også tillader et godt adskilt 
modellag, at blive genbrugt i andre lignende områder. På 
samme tid er det i dette lag hvor validerings logik, og andre 
data beregninger på objekter skal udføres. Om muligt skal 
et objekt i model laget ikke have nogen forbindelse til 
View laget, således der ikke er nogen direkte forbindelse til 
brugergrænsefladen. Kommunikation mellem dette lag og hvad 
brugeren ser, skal ske igennem Controller laget.\fxfatal{Hvis dette er sandt fatter jeg totalt meget hat af denne figur: \url{http://www.cs.cf.ac.uk/Dave/HCI/HCI_Handout_CALLER/MVC.gif} }

\textbf{View} laget er hvad brugeren kan se. Et objekt i view 
laget bruges primært til at fremvise data, samt tillade 
brugere at ændre data i model laget. På trods af den 
interaktion der er mellem lagene, er de frakoblet hinanden, 
kommunikation om opdatering af data, samt ændringer brugeren 
laver sker igennem Controller objekter. Essentielt er det en 
visuel repræsentation af den data der ligger i model laget.

\textbf{Controller}
Controller objekter fungerer som mellemled mellem objekter i 
model og view lagene. Derved bliver controller objekter til 
et kommunikationsværktøj mellem model og view lag. En brugers 
aktion på et objekt i view laget, fortæller et controller 
objekt om bruger aktionen, hvorefter controller objektet 
kontakter de objekter i model laget det vedrører, her 
behandles informationen og bliver på samme måde sendt tilbage 
til view laget og giver derved brugeren respons.
