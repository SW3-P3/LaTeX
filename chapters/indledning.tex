\chapter{Indledning}\label{chapter:indledning}

I dagligdagen kan det være svært at koordinere indkøb, opskrifter og tilbud således, at dette foregår uden problemer. Nogle kigger i tilbudsaviser og planlægger deres indkøb i en eller flere butikker, andre vælger blot at handle i en butik.
Hvis ikke man planlægger disse indkøbsture, kan det lede til madspild pga. overindkøb/impulskøb eller for mange unødige ture i supermarkederne, hvilket kan tage tid i ens hverdag. Derudover kan det være problematisk at koordinere, hvem der handler hvad ind med eventuelle samboende.

Det kan være svært at leve sundt og varieret som fødevareministeriet anbefaler, samt  at finde gode opskrifter man har lyst til at lave.
Hvis man planlægger sine indkøb efter opskrifter, skal man holde styr på, hvilke varer man har i sit hjem for at undgå at købe for meget ind. Det kan samtidig være svært at overskue, hvilken butik der er den bedste at handle ind i på et givent tidspunkt, eftersom at tilbud hele tiden ændrer sig og faste priser er forskellige. 

Dette projekt vil undersøge, om man kan gøre det nemmere at handle efter tilbudene i dagligvarebutikkerne, alt imens man reducerer problemerne vedr. det at beslutte sig for, hvad man skal lave til aftensmad.
Derfor vil der efter næste kapitel, om udviklingsmetoder, blive udforsket hvad der er på markedet til hjælp med denne problemstilling.
Desuden vil der også undersøges personers vaner, samt holdninger til elektronisk hjælp til problemstillingen i form af korte interviews.
