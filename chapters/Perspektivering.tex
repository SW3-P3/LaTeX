\chapter{Perspektivering}
I dette kapitel vil muligheden for fremtidig anvendelse samt videreudvikling af systemet diskuteres.
Herunder eventuelle funktionaliteter der kunne tilføjes til systemet eller eksisterende funktionaliteter der kunne forbedres, samt hvordan det ville stå på det nuværende marked af lignende systemer.

\section{Systemets brugbarhed}
Udvikling af et system er ikke det eneste der bestemmer dets succes i fremtiden.
Et system skal også være brugbart i den kontekst som det er udviklet til.
Gennem både vores to interviewrunder, samt de to brugertest har feedbacken med henblik på at benytte sådan et system, været overvejende positiv.
Gennem de samme tests og interviews har der været feedback, som vi enten har valgt af vælge fra, eller ikke har kunne nå at implementere.
En implementering af dette ville øge systemets brugbarhed.
Ligeledes vil mange funktioner i programmet kunne forbedres, ved at tilpasse dette efter den vores brugerfeedback. 

Dette betyder at vi har en system med nogle gode funktionaliteter, osm vores testpersoner synes om, men at systemets funktionalitet både kan forbedres og udvides.

%\section{Systemet på markedet}
%Som nævnt i state of the art analysen, \myref{s:SOTA}, er systemet ikke det eneste på market, derjar til formål at hjælpe den almene familie med den organisatoriske proces som hører til madlavning.
%De systemer som eksisterer, er dog typisk begrænset til færre funktionaliteter.
%Ses der på tilbudsugen eller eTilbudsavis, er det primære fokus her, tilbud der kan tilføjes til en indkøbsliste.
%Foruden de to kendte tilbudsapps, udgiver mange af butikskæder også egne apps til både android og iphones.
%De fleste af appsne kan det samme, hvor fakta skiller sig ud ved også at implementere deling, madplan og rabatkuponer.
%Ulempen ved de enkelte apps, er at de er designet specifikt til den enkelt butikskæde, det sker at disse apps får en dårlig vurdering af brugerne, hvilket også kan ses på \myref{tbl-smartphone}.
%Den udviklede hjemmeside i denne rapport, kan ses som en samling for disse individuelle apps, effektiv på tværs af alle butikskæderne, ligesom eTilbudsavis, og Tilbudsugen.
%Før systemet kan blive en rigtig konkurrent, ville det dog kræve ændringer som kan ses i \myref{udvidelse}.
%En af fordelene ved disse apps er at de er udviklet til de enkelte butikskæder, er bl.a. at det er butikskæderne der har ønsket dem udviklet, dette betyder at alt information for hele varekataloget muligvis er tilgængelig for dem.
%Sådanne fordele kan kun opnås gennem samarbejde med de individuelle butikskæder.

\section{Løsning og forbedring af eksisterende funktionalitet i systemet}
Efter endt udvikling og undervejs i systemudviklingen, er der identificeret forskellige funktionaliteter som kunne forbedres, disse er identciceret af både gruppen og vores testpersoner.
Disse løsninger og problemer omhander både brugergrænsefladen, samt det bagvedliggende system.

\subsection{Systemfunktionaliteter}
For det første ville bedre data om tilbud gøre oplevelsen i programmet en del bedre.
Som beskrevet i \myref{api:skoddata}, ville data med kategorier gøre det meget mere overskueligt at browse gennem tilbudene.
Desuden ville dette data gøre det nemmere at forbedre forskellige funktionaliteter i systemet, sådanne forbedringer ville være smartere søgefunktionaliteter.
Der mangler også billeder under opskrifter, man kan ikke se hvordan ens mad ender ud.
Hvis der var mere tid ville der udvikles en måde hvorpå man kunne uploade billeder til hjemmesiden. 
Dette ville gså kræve en større ændring i brugergrænsefladen for se billederne.

Brugere har i testene også udtrykt interesse for at kunnne se førpriser og vi kunne ligeledes også godt tænkt os at have normalpriserne på de forskellige butikkers varer.
Førpriserne ville vise hvor gode avisvarerne fra de forskellige butikker er.
Normalpriserne ville gøre det muligt at regne totalpriser ud på ting som indkøbslister samt opskrifter, for hver butik, vi ser dette som en smart funktion, som vi tror brugerne vil have stor gavn af.

\subsection{Brugergrænsefalde}
I den seneste brugertest, \myref{ss:bt2}, blev der opfanget en række af problemer.
Mange af disse problemer kan løses af brugeren ved blot at læse de to til tre linjer beskrivelse af hver systemdel, som er til rådighed i systemet, noget som testpersonerne også selv har nævnt.

Dette er dog ikke sandt for alle problemerne.
Herunder gives en beskrivelse af problemernes mulige oprindelse, såvel som mulige løsninger af problemerne.

Et af disse problemer eksisterer under indkøbsliste sektionen.
Her bliver indkøbslister delt ind i to kategorier, ``mine'' såvel som ``delte''.
Denne opdeling skabte forvirring for brugere, da handlingen at dele en liste, flyttede listen over i delte, det hjalp ikke på dette problem, at brugere delte en liste som hed ``min inkøbsliste''.
Hertil forslåes to løsningsmuligheder, en kunne være at en reference til listen eksisterede under begge kategorier, således ville en liste ikke forsvinde fra ``mine'' når en liste deles.
En anden mulighed ville være at listen, for vedkommende som oprettede den, ikke stadig blev under ```mine'', mens at den for de den er delt med, ligges under ``delte''.

Det andet problem der blev opfanget igennem denne test, var et problem vedrørende at vælge hvilken indkøbsliste, ens aktioner i systemet agere på.
Ud fra brugertest er det ikke muligt at konkludere hvor kritisk problemet var, størstedelen af de testede, slet ikke benyttede funktionaliteten idét de valgte at arbejde ud fra standard inkøbslisten.
Hertil nævnes det også fra de selvsamme brugere, at de ignorede funktionaliteten, da de alligevel kun havde en liste.
En af de personer som oprettede deres egen liste, nævnte at det skabte lidt forvirring at listen hedder ``min inkøbsliste'' og derfor ikke tænker videre over hvilken liste der tilføjes til.
Lignende kommentar blev nævnt for forrige problem, hertil menes det altså nødvendigt at ændre navnet, på denne standard oprettede liste.

\section{Udvidelse af systemet}\label{udvidelse}

Der er forskellige features som vi gerne så udviklet på programmet, givet mere tid. Dels fordi brugerne efterspurgt dem, dels fordi vi på gruppen synes det ville forbedre oplevelsen.

En mulig forbedring som vi afgrænsede os fra i /myref{sysdiffi} er \textit{tømkøleksabet}.
Der blev nævnt i interview og brugertests at et form for en udvidet søgning kunne være til fordel at tilføje.
En sådan funktion ville munde ud i muligheden for at kunne fortælle brugeren om hvilket mad der kunne lave med allerede ejet ingredienser.
En sådan opskrift kunne bliver anbefalet på hvilket der var billigst, i forhold til hvad mad der allerede var i køleskabet.
Funktionen blev fravalgt på baggrund af vi prioriterede anbefalinger højre, og vi lavede ikke begge ting grundet tiden ikke var til rådighed.

En anden forbedring eller feature ville være at kunne lægge en kommentar ved sin vurdering af en opskrift, således der kan læses om de forskellige opskrifter. 

En funktion der kunne implementeres ville være en mad plan.
Madplanen blev forslået af test personerne i vores prototype test fundet på \label{ch:protorespons}.
En madplan kunne implementeres for at give brugeren mulighed for at få foreslået mad igennem en periode, f.eks. en uge.
Denne madplan kunne eventuelt sørge for at brugeren ville få en varierede og sund kost.
Madplanen blev fra valgt da vi mente at den faldt uden for projektets omfang.

Ligeledes blev der også nævnt af i vores prototype test, at der var interesse for at kunne finde nærmest butikker ud fra hvilket varer man havde valgt.
Denne funktion ville give brugeren mulighed, for bedre at kunne finde den nærmeste butik, samt at kunne finde den korste rute imellem butikker, hvis der var tilbud fra flere på indkøbslisten.
Funktionen blev lige ledes hurtigt fravalgt da denne også faldt uden for projektet omfang.

