\chapter{Perspektivering}
I dette kapitel vil muligheden for fremtidig anvendelse samt videreudvikling af systemet diskuteres.
Herunder eventuelle funktionaliteter der kunne tilføjes til systemet, samt hvordan det ville stå på det nuværende marked af lignende systemer.

\section{Systemets brugbarhed}
Udvikling af et system er ikke det eneste der bestemmer dets succes i fremtiden.
Et system skal også være brugbart i den kontekst som det er udviklet til.
I de indledende interviews, \myref{section:interview1}, fandt vi ud af at enktle benyttede sig af appen eTilbudsavis for at finde tilbud, men foruden dette blev den nærmeste kun brugt til at lave en elektronisk indkøbsliste.
En af de adspurgte nævnte endda ``Mit liv er for kort til at bladre gennem diverse tilbudsaviser'', men havde en positiv holdning til en funktionalitet så som deling af indkøbslister.
Hertil stilles spørgsmålet, hvor brugbart ville et system med selv samme funktionalitet, såvel som yderligere funktionaliteter, egentlig være.
Ifølge de selv samme indledende interviews, var der en interesse i et sådant system, dette er blevet yderligere bekræftet igennem brugertests af det udviklet system, \myref{s:brugertests}, hvor feedback omkring hvorvidt testpersonerne ville benytte sig af et udgivet system, var positivt.
Dette betyder at der stadig er rig mulighed for et system at komme ind på markedet, givet der bliver gjort opmærksom på det.

\section{Systemet på markedet}
Som nævnt i state of the art analysen, \myref{s:SOTA}, er systemet ikke det eneste forsøg på at hjælpe den almene familie med den organisatoriske process som hører til madlavning.
De systemer som eksisterer, er dog typisk begrænset til enkelte funktionaliteter.
Ses der på tilbudsugen eller eTilbudsavis, er den primære fokus her på tilbud med indkøbsliste som en seddel, der ud fra indledende interviews, ikke bliver benyttet.
Foruden de to kendte tilbudsapps, udgiver de enkelte butikskæder apps til både android og iphones.
De fleste af appsne kan det samme, hvor fakta skiller sig ud ved også at implementere deling, madplan og rabatkuponer.
Ulempen her er at de enkelte apps, er designet specifikt til en enkelt butikskæde, og som følge kun er relevant for den kæde.
Den udviklede hjemmeside i denne rapport, kan ses som en samling for disse individuelle apps, effektiv på tværs af alle butikskæderne.
Dette giver systemet en unik plads på det nuværende marked, som en samlet version af de mindre apps.
Før systemet kan blive en rigtig konkurrent, ville det dog kræve ændringer som kan ses i \myref{udvidelse}.
En af de fordele de apps udviklet til de enkelte butikskæder har, er bl.a. at det er butikskæderne der har ønsket dem udviklet, dette betyder at alt information for hele varekataloget, er tilgængelig for dem.
Sådanne fordele kan kun opnås gennem samarbejde med de individuelle butikskæder.

\section{Løsning af eksisterende brugergrænsefladeproblemer}

\section{Udvidelse af systemet}\label{udvidelse}
