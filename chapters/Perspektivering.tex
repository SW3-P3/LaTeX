\chapter{Perspektivering}
I dette kapitel vil muligheden for fremtidig anvendelse samt videreudvikling af systemet diskuteres.
Herunder eventuelle funktionaliteter, der kunne tilføjes til systemet eller eksisterende funktionaliteter, der kunne forbedres.

\section{Løsning og forbedring af eksisterende funktionalitet i systemet}
Undervejs i systemudviklingen, og efter endt udvikling, er der identificeret forskellige funktionaliteter, som kunne forbedres; disse er identificeret af både gruppen og vores testpersoner.
Løsninger og problemer omhandler både brugergrænsefladen samt det bagvedliggende system.

\subsection{Systemfunktionaliteter}
Bedre tilbudsdata vil gøre brugeroplevelsen i programmet mærkbart bedre.
Som beskrevet i \myref{api:skoddata} om eTilbudsavisen' API, ville data med kategorier gøre det mere overskueligt at browse gennem tilbudene.
Desuden ville dette data gøre det nemmere at forbedre forskellige funktionaliteter i systemet.
Eksempelvis smartere søge-/filtreringsfunktioner ift. tilbud.
Der mangler også billeder under opskrifter, man kan ikke se hvordan ens mad ender ud.
En anden forbededring ville derfor være muligheden for at uploade billeder, der kunne vises med opskirfter. 
Dette ville også kræve en større ændring i brugergrænsefladen for at vise billederne.

Testpersonerne har også udtrykt interesse for at kunne se før-priser.
Før-priserne ville vise besparelsen for tilbuddene i de forskellige butikker.
Ligeledes kunne det være brugbart at have normalpriserne for alle varerne i de forskellige butikker.
Normalpriserne ville gøre det muligt at regne totalpriser ud på indkøbslister og opskrifter.
Desuden ville det være muligt at kunne lave en sammenligning af indkøbslistens totalpris ift. hvilken butik man handler i, og på denne måde finde den billigste indkøbstur.
Vi ser dette som en smart funktion, som vi tror brugerne ville have stor gavn af.

\subsection{Brugergrænseflade}
Mange af problemerne som testpersonerne stødte ind i ved brugergrænsefladen, kan løses af brugeren, ved blot at læse de to til tre linjer beskrivelse ved hver del af systemet.
Problemet som de fortæller det er dog, at de ikke læser teksten før de støder ind i fejlen, men så kan teksten heldigvis hjælpe dem videre, skulle de sidde fast.

Dette er dog ikke sandt for alle problemerne.
Herunder gives en beskrivelse af problemernes mulige oprindelse, såvel som mulige løsninger af problemerne.

Et af disse problemer eksisterer under indkøbsliste sektionen.
Her bliver indkøbslister delt ind i to kategorier, \textit{mine} og \textit{delte}.
Denne opdeling skabte forvirring for brugere, da handlingen \textit{at dele en liste} flyttede listen over i delte.
Det hjalp ikke på dette problem, at brugere under testen delte en liste som hed ``Min Inkøbsliste''.
Hertil forslås to løsningsmuligheder. 
En kunne være, at en reference til listen eksisterede under begge kategorier. 
Således ville en liste ikke forsvinde fra \textit{mine} når den deltes.
En anden mulighed ville være, at listen for vedkommende som oprettede listen forblev under \textit{mine}, mens at den for brugere, den er delt med, placeres under \textit{delte}.
En tredje mulighed, som også blev foreslået af brugerne, var at tilføje et ikon til delte indkøbslister, frem for at adskille i to lister.

Et andet problem, der blev opfanget gennem denne test, var et problem vedrørende det at vælge, hvilken indkøbsliste ens aktioner i systemet ageres på.
Ud fra brugertestene er det ikke muligt at konkludere, hvor kritisk problemet var.
Størstedelen af de testede benyttede slet ikke funktionaliteten, da de valgte at arbejde ud fra den automatisk oprettede indkøbsliste.
Hertil nævnes det også fra de selvsamme brugere, at de ignorerede funktionaliteten, da de alligevel kun havde én liste.
En af personerne, som oprettede deres egen liste, nævnte, at det skabte lidt forvirring, at den automatiske liste hedder ``Min Indkøbsliste'' og derfor ikke tænker videre over, hvilken liste der tilføjes til.
Lignende kommentarer blev nævnt for forrige problem, hertil menes det altså nødvendigt at ændre navnet på denne standard oprettede liste.

\section{Udvidelse af systemet}\label{udvidelse}
Der er forskellige features som vi gerne så udviklet på programmet. Dels fordi brugerne efterspurgte dem, og dels fordi vi på gruppen synes det ville forbedre oplevelsen på hjemmesiden.

En mulig forbedring som vi afgrænsede os fra i systemdefinitionen er \textit{tøm køleskabet}.
Der blev nævnt i interview og brugertests at en form for udvidet søgning kunne være til fordel at tilføje.
En sådan funktion ville munde ud i muligheden for at kunne fortælle brugeren om hvilke retter der kunne lave med allerede ejede ingredienser.
Disse kunne så evt. anbefales ud fra hvilke der var de billigste at lave.
Funktionen blev fravalgt på baggrund af at vi prioriterede anbefalinger højre.

En anden forbedring eller feature ville være at kunne lægge en kommentar ved sin vurdering af en opskrift.
På denne måde kunne man udover talvurderingen kan læses hvad andre brugere synes om opskrifterne. 

En anden funktionalitet der kunne implementeres ville være en mad plan.
Madplanen blev forslået af test personerne i vores prototype test fundet på \myref{ch:protorespons}.
En madplan kunne implementeres for at give brugeren mulighed for at få foreslået varierede opskrifter gennem en periode.
Vi fravalgte madplanen da vi mente at den faldt uden for projektets omfang.

Ligeledes blev der også nævnt i vores prototype test, at der var interesse for at se afstanden til de tilbud der vælges på indkøbslisten.
Denne funktion ville give brugeren mulighed, for bedre at kunne bedømme hvilke tilbud de vil have, hvis afstanden til butikkerne har relevans for brugeren.
Funktionen blev også fravalgt da den faldt uden for projektets omfang.
