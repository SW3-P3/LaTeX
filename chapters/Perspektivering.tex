\chapter{Perspektivering}
I dette kapitel vil muligheden for fremtidig anvendelse samt videreudvikling af systemet diskuteres.
Herunder eventuelle funktionaliteter der kunne tilføjes til systemet eller eksisterende funktionaliteter der kunne forbedres, samt hvordan det ville stå på det nuværende marked af lignende systemer.

\section{Systemets brugbarhed}
Udvikling af et system er ikke det eneste der bestemmer dets succes i fremtiden.
Et system skal også være brugbart i den kontekst som det er udviklet til.
I de indledende interviews, \myref{section:interview1}, fandt vi ud af at enktle benyttede sig af appen eTilbudsavis for at finde tilbud, men foruden dette blev den nærmeste kun brugt til at lave en elektronisk indkøbsliste.
En af de adspurgte nævnte endda ``Mit liv er for kort til at bladre gennem diverse tilbudsaviser'', men havde en positiv holdning til en funktionalitet så som deling af indkøbslister.
Hertil stilles spørgsmålet, hvor brugbart ville et system med selv samme funktionalitet, såvel som yderligere funktionaliteter, egentlig være.
Ifølge de selv samme indledende interviews, var der en interesse i et sådant system, dette er blevet yderligere bekræftet igennem brugertests af det udviklet system, \myref{s:brugertests}, hvor feedback omkring hvorvidt testpersonerne ville benytte sig af et udgivet system, var positivt.
Dette betyder at der stadig er rig mulighed for et system at komme ind på markedet, givet der bliver gjort opmærksom på det.

\section{Systemet på markedet}
Som nævnt i state of the art analysen, \myref{s:SOTA}, er systemet ikke det eneste forsøg på at hjælpe den almene familie med den organisatoriske process som hører til madlavning.
De systemer som eksisterer, er dog typisk begrænset til enkelte funktionaliteter.
Ses der på tilbudsugen eller eTilbudsavis, er den primære fokus her på tilbud med indkøbsliste som en seddel, der ud fra indledende interviews, ikke bliver benyttet.
Foruden de to kendte tilbudsapps, udgiver de enkelte butikskæder apps til både android og iphones.
De fleste af appsne kan det samme, hvor fakta skiller sig ud ved også at implementere deling, madplan og rabatkuponer.
Ulempen her er at de enkelte apps, er designet specifikt til en enkelt butikskæde, og som følge kun er relevant for den kæde, og mange af dem har dårlige ratings, som også kan ses i \myref{tbl-smartphone}.
Den udviklede hjemmeside i denne rapport, kan ses som en samling for disse individuelle apps, effektiv på tværs af alle butikskæderne, ligesom eTilbudsavis, og Tilbudsugen.
Før systemet kan blive en rigtig konkurrent, ville det dog kræve ændringer som kan ses i \myref{udvidelse}.
En af fordelene ved disse apps er at de er udviklet til de enkelte butikskæder, er bl.a. at det er butikskæderne der har ønsket dem udviklet, dette betyder at alt information for hele varekataloget muligvis er tilgængelig for dem.
Sådanne fordele kan kun opnås gennem samarbejde med de individuelle butikskæder.

\section{Løsning af eksisterende problemer ved brugergrænsefladen}
I den seneste brugertest, \myref{ss:bt2}, blev der opfanget en række af problemer.
Af disse problemer, er størstedelen af disse mindre problemer, som kun en enkelt bruger er stødt ind i, disse er således kategoriseret som kosmetiske problemer der ikke ville have en kritisk indflydelse på systemets brug.
På samme tid hænder det at mange af disse problemer kan løses ved blot at læse de to til tre linjer beskrivelse af hver systemdel, som er til rådighed i systemet, noget som testpersonerne også selv har nævnt.

Dette er dog ikke sandt for alle problemer, de mere alvorlige problemer vil i denne sektion opsummeres.
Yderligere gives en beskrivelse af problemets mulige oprindelse, såvel som en mulig løsning af problemet.

Et af disse problemer eksisterer under indkøbsliste sektionen.
Her bliver indkøbslister delt ind i to kategorier, ``mine'' såvel som ``delte''.
Denne opdeling skabte forvirring for brugere, da handlingen at dele en liste, flyttede listen over i delte, det hjalp ikke på dette problem, at brugere delte en liste som hed ``min inkøbsliste''.
Hertil forslåes to løsningsmuligheder, en kunne være at en reference til listen eksisterede under begge kategorier, således ville en liste ikke forsvinde fra ``mine'' når en liste deles.
En anden mulighed ville være at listen, for vedkommende som oprettede den, ikke stadig blev under ```mine'', mens at den for de den er delt med, ligges under ``delte''.

Det andet problem der blev opfanget igennem denne test, var et problem vedrørende at vælge hvilken indkøbsliste, ens aktioner i systemet agere på.
Ud fra brugertest er det ikke muligt at konkludere hvor kritisk problemet var, størstedelen af de testede, slet ikke benyttede funktionaliteten idét de valgte at arbejde ud fra standard inkøbslisten.
Hertil nævnes det også fra de selvsamme brugere, at de ignorede funktionaliteten, da de alligevel kun havde en liste.
En af de personer som oprettede deres egen liste, nævnte at det skabte lidt forvirring at listen hedder ``min inkøbsliste'' og derfor ikke tænker videre over hvilken liste der tilføjes til.
Lignende kommentar blev nævnt for forrige problem, hertil menes det altså nødvendigt at ændre navnet, på denne standard oprettede liste.

De yderligere problemer opfanget igennem testen, består af små kosmetiske problemer, opfanget af enkelte brugere, som ikke har nogen videre konsekvenser for funktionalitetsbrugen.
Disse problemer indeholder andre måder at sortere tilbud, tvivl om ting bliver gemt og lignende.
Svaret til nogle af de problemer som er nævnt, er desuden at finde på siden, hvis der blot bruges lidt tid på at læse.
Dette blev også bekræftet under testet, hvor en person der valgte at gøre brug af dette tekst, ungik sådan forvirring.


\section{Udvidelse af systemet}\label{udvidelse}

Der er forskellige features som vi gerne så udviklet på programmet, givet mere tid. Ikke fordi brugerne specifikt har efterspurgt dem alle, men fordi vi på gruppen synes det ville forbedre oplevelsen.

For det første ville bedre data for tilbud gøre oplevelsen i programmet en del bedre.
Som beskrevet i \myref{api:skoddata}, ville data med kategorier gøre det meget mere overskueligt at browse gennem tilbudene. 
Desuden hvis man kendte priserne på de generelle varer ville det gøre mange flere ting mulige i programmet, såsom at beregne en opskrifts pris. 
Desuden ville det være belejligt at kunne sortere varer fra som brugeren er allergisk overfor.

Der mangler billeder under opskrifter, man kan ikke se hvordan ens mad ender ud.
Hvis der var mere tid ville der udvikles en måde hvorpå man kunne uploade billeder til hjemmesiden. 
Derudover skulle der ændres i brugergrænsefladen så man kan se billederne.
En anden forbedring eller feature ville være at kunne lægge en kommentar ved sin vurdering af en opskrift, således der kan læses om de forskellige opskrifter. 

Der er lavet smartphone support i bl.a. Indkøbslister og Opskrifter.
Havde vi haft mere tid havde vi lavet bedre support til de resterende elementer på siden, samt at udvikle support til et lidt større display som tablets.
