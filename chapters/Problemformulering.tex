\chapter{Problemformulering}

State of the art undersøgelserne viste at der findes mange applikationer og løsninger, til at hjælpe med den daglige madlavning og indkøb til hjemmet. 
Det viste sig desuden at selvom der var mange løsninger, var disse ikke alle optimale, nogle grundet manglende funktionalitet, andre grundet dårlig udførelse af programmet. 
Vores interviews viste at der ikke blev benyttet mange hjælpemidler i forvejen, men at der dog var en interesse for emnet.
Dette kan bl.a. skyldes at de allerede eksisterende løsninger, ikke har interesse for personerne, dårlig kvalitet af programmer, eller måske bare dårlig markedsføring. Der vil derfor arbejdes videre med følgende problemstilling:

\textbf{Hvordan kan dagligvarebutikkernes tilbud gøres mere overskuelige med integrerede indkøbslister, således det er lettere at leve efter billige og varierede opskrifter til aftensmad?}

Som det første vil der til analysen benyttes en metode kaldet Objekt orienteret analyse og design(OOA\&D).
Dette gøres for at skabe klarhed over kravene til et IT-system, der skal hjælpe på området.
Desuden giver det et grundlag for systemets implementering. 