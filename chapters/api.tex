\chapter{eTilbudsavis API}\label{api}
\fxfatal{Dette afsnit er under udvikling ... - Troels}
For at kunne indhente infomationer om placering af butikker og deres tilbud, anvendes der et API.
Dette API udbydes af den danske virksomhed eTilbudsavis.dk, og er af typen REST.

\section{REST}
REST står for: \textit{Representational state transfer}.
Det er en metode hvorved man opbygger måden som et API. 
Der findes fire metoder som svarer til de fire som findes i CRUD (``CREATE, READ, UPDATE, DELETE'').
Disse metoder er: GET, POST, PUT og Delete.

\textbf{GET}
returnerer data typisk som JSON eller XML og med HTTP svarkoden 200 (OK). 
eTilbudsavis' API sender JSON. 
Svarer til READ i CRUD.

\textbf{POST}
bruges til at oprette nye ressourcer.
Svarer til CREATE i CRUD.

\textbf{PUT}
bruges til at opdatere infomationer, eksmpelvis til at forlænge en session.
Svarer til UPDATE i CRUD.

\textbf{DELETE}
bruges til at fjerne ressourcer. 
Svarer til DELETE i CRUD.

\section{Brug af API}
eTilbudsavis' API er kun et halv-offentligt API, dvs. at man skal have en API-nøgle (``APIKEY'') og dertilhørende hemmelighed (``Secret'').
API'en er placeret på \textit{https://api.etilbudsavis.dk/v2/}. \citep{eTilAPI}

For at kommunikere med API'en fra C\#-kode anvendet værktøjet RestSharp. \citep{RestSharp} 
For at kunne bruge infomationen laves der en klasse, som svarer til det JSON som API'en sender tilbage.

\subsection{Session}
For at kunne bruge API'en skal man først oprette en session. 
En session består af en API-nøgle, den tilsvaredne hemmelighed (``Secret''), en token og en signatur.
Har man en API-nøgle kan man kalde API'et sessions og få en token.
Den token kan man kombinere med sin hemmelighed og generere SHA-256 hashen af dette som er signaturen. 

Eksempelvis vil apikaldet som opretter en session returnere:
\begin{lstlisting}[language=json,firstnumber=1,caption="POST til sessions api'en med APIKEYen",label=apilst1]
{
    "token": "00hdcx7fnysn6541",
    "expires": "2013-03-03T13:37:00+0000",
    "user": null,
    "provider": null,
    "permissions": {
        "guest": [
            "api.public",
            "api.users.create"
        ]
    }
}
\end{lstlisting}

Dette svar kan oversættes til en klasse i C\#.
\begin{lstlisting}[caption="Klassen ``Session'' som svarer til retur JSONet sendt fra /v2/sessions' APIet"]
public class Session
{
    public string Token { get; set; }
    public string Expires { get; set; }
    public string User { get; set; }
    public string Provider { get; set; }
    public Permissions Permissions { get; set; } // List<string> Guest 
}
\end{lstlisting}

Hvis dette kald skulle gøres ved brug af C\#-kode så vil det kunne gøre på følgende måde, med RestSharp.
%[firstnumber=1,caption="Det foerste API kald"]
\begin{lstlisting}[caption="C\#-kode som opretter en RestClient og anvender den til at oprette et objekt med felter som svarer til JSON dataet givet fra API'en"]
/* [...] */
/* Create a RestClient*/
var client = new RestClient("https://api.etilbudsavis.dk");

/* Initiate the first request to get a session */
var SessionRequest = new RestRequest("v2/sessions", Method.POST);
SessionRequest.AddParameter("api_key", Global.Apikey);

/* Map the response to the class "Session" */
IRestResponse<Session> response2 = client.Execute<Session>(SessionRequest);

/* Save the response in an object */
Session sessionobj = response2.Data;

/* Generate a signature and set it to the global variable */
Global.Signature = EncryptionHelper.SHA256(Global.Secret + Global.SessionToken);
/* [...] */
\end{lstlisting}
Efter eksekveringen af dette kode, vil der være et objekt af datatypen ``Session'' kaldt sessionobj, som indeholder de samme felter som givet i JSON i \myref{apilst1}; derudover er signaturen (``Global.Signature'') også beregnet og gemt. 

\subsection{Fællestræk for den rasterende API}
Der findes en række fællestræk for brugen af API'en.
Centralt sendes SessionToken'en og signaturen som http--headers.
Derudover genanvendes de samme 3 parametre ofte: længdegrad, bredegrad og radius.
Disse er et udtryk for en placering af og afstand til butikker.

\subsection{Store, Catalog og Offer}
Det er muligt at spørge API'en om hvilke butikker som er inden for en given radius af en placering som beskrevet ovenfor.
Det vil med C\#-kode se ud som følger:
\begin{lstlisting}[caption="C\#-kode som anvender ``/v2/stores'' delen af API'en til at hente butikker.",label=apistorecs]
/* [...] */
/* Create request for getting stores near a coordinate */ 
var storeRequest = new RestRequest("v2/stores", Method.GET);
storeRequest.AddParameter("r_lat", latitude);
storeRequest.AddParameter("r_lng", longitude);
storeRequest.AddParameter("r_radius", radius);
storeRequest.AddHeader("X-Token", Global.SessionToken);
storeRequest.AddHeader("X-Signature", Global.Signature);

var storesResult = client.Execute<List<Store>>(storeRequest).Data;
/* [...] */
\end{lstlisting}

storesResult er af datatypen List<Store> hvor ``Store'' er klassen som svarer til det JSON API kaldet returner. 
Dette er udformet som følger:
\begin{lstlisting}[caption="Klassen Store",label=apistoreclass]
public class Store
{
    public string id { get; set; }
    public string ern { get; set; }
    public string street { get; set; }
    public string city { get; set; }
    public string zip_code { get; set; }
    public Country country { get; set; }
    public string latitude { get; set; }
    public string longitude { get; set; }
    public string dealer_url { get; set; }
    public string dealer_id { get; set; }
    public Branding branding { get; set; }
    public string contact { get; set; }
    public List<string> category_ids { get; set; }
}
\end{lstlisting}

Hvor Branding, Country og Pageflip klasserne er:

\begin{lstlisting}[caption="Klasserne Branding Pageflip og Country",label=apistoreclasshelpers]
public class Branding
{
    public string name { get; set; }
    public string url_name { get; set; }
    public string website { get; set; }
    public string logo { get; set; }
    public string logo_background { get; set; }
    public string color { get; set; }
    public Pageflip pageflip { get; set; }
}

public class Pageflip
{
    public string logo { get; set; }
    public string color { get; set; }
}

public class Country
{
    public string Id { get; set; }
    public string UnsubscribePrintUrl { get; set; }
}
\end{lstlisting}

Hele denne process er gentaget og tilpasset Catalog og Offer API'erne for at kunne indhente den infomation de kan give. 
Det er hermed muligt at hente infomationer om tilbud og butikker. \fxfatal{Dette er et helvede at beskrive på en måde som giver mening, så alt det ovenståene skal nok skrives om...}
