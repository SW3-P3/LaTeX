\chapter{Metode}\label{chapter:Metode}

Dette kapitel vil gennemgå og evaluere på de metoder som benyttes i forbindelse med projektet.
Slutteligt vil der være en kort beskrivelse af metodernes sammenhæng, samt hvordan de benyttes i gruppen.

\section{Udviklingsmetode}
Fra OOA\&D, ved vi at der eksistere to primære perspektiver på systemudvikling, det mekanistiske og det romantiske.
Det mekaniske perspektiv er en rationel holdning til udvikling der bygger på deduktion og matematisk problemløsning, hertil følges en på forhånd lavet plan, med meget få ændringer undervejs.
I dette perspektiv konstrueres hele systemudviklingen således på forhånd.
Det romantiske perspektiv er en diametral modsætning hertil, som ser hverden på en evolutionær måde.
I dette perspektiv er problemer sjældent klare og præcise, eksperimenter og empirisk viden er måden at løse et problem, perspektivet følger en evolutionær trial and error process.
Det to perspektiver munder ud i henholdsvis hver deres udviklingsmetode, konstruktion og evolution, som håndterer problemer efter hvert sit perspektiv.\citep{OOA&D2001}

\textbf{Konstruktion:}
Denne udviklingsmetode bygger på en rationel holdning til problemer, i den forstand kan alt planlægges på forhånd, denne type af arbejdsmetode kendes også som vandfaldsmodellen.

\textbf{Evolution:}
Denne udviklingsmetode bygger på at en problem ikke kan præciseres på forhånd, og benytter sig af ``trial and error'' konceptet for dets arbejdsmetode, denne type af arbejdsmetode kendes også som en iterativ arbejdsmetode.

På gruppen besluttes det at benytte den evolutionære arbejdsmetode for udvikling af et system.

\section{Scrum}

Scrum er en process som bruges i forbindelse med udarbejdelse af komplekse problemer som løbende kan udvikle sig. 
Scrum er en simpel process, men kan være svær at udføre godt.
Det bruges ofte i virksomheder hvor man har kunder som kan beskrive hvilke krav der er til projektet.
Disse krav forstås og holdes styr på af en Product Owner som så styre en \textbf{product backlog}.
Dette er en liste over opgaver der skal løses, og disse skrives ofte som en user story på formen:\\ \\ \textit{''Som en <rolle> vil jeg kunne <mål>''. }\\ \\
Disse user stories vil derefter vælges til et såkaldt sprint som varer 30 dage. 
I disse 30 dage arbejdes der i grupper af normalt 7 for at få alle user stories der er valgt i sprintet til at blive færdige og testet.
Der kan laves et scrumboard for at holde styr på hvilke opgaver der skal laves til hver enkelt user story. 
På denne måde skaber man overblik over opgaverne, og man kan hele tiden se hvad der mangler at blive lavet.
Hver dag når man mødes laver men at Daily Scrum, som er et kort møde, hvorpå man fortæller de andre i gruppen hvad man har lavet, hvad man vil lave den givne dag, samt problemer med arbejdet.
En Scrum master sørger for at gruppen følger Scrum, og at Daily Scrum forbliver på sporet. 
Personen står også for alt kontakt mellem gruppen og administrationen ved virksomheden.
Efter et sprint mødes man og evaluerer for kunderne hvad man har lavet ved et Sprint review meeting.
Desuden afholder teamet også et Sprint reflection meeting, til at evaluere på hvad der gik godt.
På denne måde kan man sørge for at næste sprint bliver bedre end det forrige. 
Men hvordan forholder Scrum sig i forhold til projektet?

\begin{sidewaystable}
    \colorlet{shadecolor}{gray!40}
    \rowcolors{1}{white}{shadecolor}
      \begin{tabular}{p{5cm}p{5cm}p{5cm}p{5cm}}
      %\hline
	       				 & Scrum  & Vores Projekt & Kommentarer  \\ \hline++

	   Iterationslængde  		
	   		& 30 dage 
	   		& 7 - 14 dage. 
	   		& Hvis der skulle bruges 30 dages iterationer ville der måske kun kunne laves 2 iterationer, på denne baggrund besluttes det at reducere iterationslængden, for at tilpasse modellen til projektmiljøet.\\
	   		
	   Team størrelse    		
	   		& 7 personer
	   		& 6 personer 
	   		& Dette passer fint og kan fungere fint. \\
	   		
	   Kunde involvering 		
	   		& Meget kunde involvering, fremvisning af produkt ofte.
	   		& Vi har ikke en kunde, men der er potentielle brugere som kan interviewes for at finde frem til deres krav til et produkt.
	   		& Det er op til gruppen at finde krav og danne userstories, hertil kan lave brugerundersøgelser. Product Owner er gruppen, da det er vores produkt, derfor styres product backlog af gruppen.\\
	   		
	   Kommunikationsmetoder	
	   		& Daily Scrum, Sprint review meeting, Sprint reflection meeting, Scrum master, kontakt med kunder og administration fra virksomheden.
	   		& Gruppen har ikke nogen kontakt med adiministration, men kan have opgaver fra kurser som kan distrahere. 
	   		& Da det er Scrum Masterens rolle at håndtere det administrative omkring metoden frem for selve udviklingen, går rollen på runde fra uge til uge, og udelukker sammentidig ikke involvering i udvikling. Review meeting kan holdes uden kunder, og fungerer som et godt tidspunkt at gennemgå koden sammen med resten af gruppen. \\
    \end{tabular}
  \caption{Sammenligningstabel over Scrum og vores projekt.}\label{tabel:sammenligningstabel}
\end{sidewaystable}

Der skal laves ændringer på den traditionelle Scrum process  for at tilpasse det til projektet, disse kan håndteres og ændres, som \myref{tabel:sammenligningstabel} også viser.

\section{Objekt Orienteret Analyse \& Design}

OOA\&D er en analysemtode som har til formål at finde krav til et system, fastlægge design, og at danne en forståelse for et system, dets omgivelser og hvilke krav der skal til for at implementere det.
Metoden stammer fra bogen af samme navn.\citep{OOA&D2001}
Den gør brug af UML til at beskrive struktur og klasser indenfor området.
Herudover finder metoden:

\begin{itemize}
\item Systemdefinition
\item Hændelser
\item Adfærd for klasserne
\item Funktioner
\item Tilstande i systemet
\item Aktører og deres brugsmønstre til systemet.
\end{itemize}

Med alle disse elementer kan laves et analyse- og designdokument.
Med disse dokumenter kan man nemmere lave et godt og velovervejet design.
Dog har vi valgt ikke at finde adfærd for klasserne, da dette var set som værende overflødigt.
Det beskriver klasserne, hvornår de laves i systemet, og gennemgangen af alle klassernes hændelser. 
Dokumentationen for disse fylder utroligt meget og selve adfærden er et mellemtrin i metoden, til at danne en god struktur, derfor dokumenteres strukturen i stedet.
De andre analyse dele er meget centrale og hjælper med at beskrive systemet og er derfor taget med. 

Scrum og OOA\&D hjælper projektet på hver sin måde.
Scrum som projektstyring og for at overskueliggøre projektets opgaver. 
En user story ved navn ``Rapport'' laves, og ud fra denne dannes der opgaver som skal laves i hvert sprint. 
Dette sørger for at rapporten ikke falder bagud i forhold til systemet udvikling.
OOA\&D hjælper derimod med at organisere og strukturere analysen som beskrevet i dette afsnit, og derfor komplimentere de to metoder hinanden udemærket.
Dog fordi Scrum benytter user stories, vil den endelige kravspecifikation ikke være beskrevet på traditionel OOA\&D vis, men i stedet vha. user stories.









