\chapter{Metode}\label{chapter:Metode}

Dette kapitel vil gennemgå og evaluere på de metoder som benyttes i forbindelse med projektet.
Slutteligt vil der være en kort beskrivelse af metodernes sammenhæng, samt hvordan de benyttes i gruppen.

\section{Scrum}

Scrum er en process som bruges i forbindelse med udarbejdelse af komplekse problemer som løbende kan udvikle sig. 
Scrum er en simpel process, men kan være svær at udføre godt.
Det bruges ofte i virksomheder hvor man har kunder som kan beskrive hvilke krav der er til projektet, og på denne måde styre \textbf{product backlog}.
Dette er en liste over opgaver der skal løses, og disse skrives ofte som en user story på formen:\\ \\ \textit{''Som en <rolle> vil jeg kunne <mål>''. }\\ \\
Disse user stories vil derefter vælges til et såkaldt sprint som varer 30 dage. 
I disse 30 dage arbejdes der i grupper af normalt 7 for at få alle user stories der er valgt i sprintet til at blive færdige og testet.
Der kan laves et scrumboard for at holde styr på hvilke opgaver der skal laves til hver enkelt user story. 
På denne måde skaber man overblik over opgaverne, og man kan hele tiden se hvad der mangler at blive lavet.
Hver dag når man mødes laver men at Daily Scrum, som er et kort møde, hvorpå man fortæller de andre i gruppen hvad man har lavet, hvad man vil lave i dag, samt problemer man med arbejdet.
En Scrum master sørger for at gruppen følger Scrum, og at Daily Scrum forbliver på sporet. 
Personen står også for alt kontakt mellem gruppen og administrationen ved virksomheden.
Efter et sprint mødes man og evaluerer og fremviser for kunderne hvad man har lavet ved et Sprint review meeting.
På denne måde kan man sørge for at næste sprint bliver bedre end det forrige. 
Men hvordan forholder Scrum sig i forhold til projektet?

\begin{sidewaystable}
    \colorlet{shadecolor}{gray!40}
    \rowcolors{1}{white}{shadecolor}
      \begin{tabular}{p{5cm}p{5cm}p{5cm}p{5cm}}
      %\hline
	       				 & Scrum  & Vores Projekt & Kommentarer  \\ \hline
	   Iterationslængde  		
	   		& 30 dage 
	   		& Der er ikke meget tid til udvikling cirka 2 måneder. 
	   		& Hvis der skulle bruges 30 dages iterationer ville der måske kun kunne laves 2 iterationer\\
	   		
	   Team størrelse    		
	   		& 7 personer
	   		& 6 personer 
	   		& Dette passer fint og kan fungere fint. \\
	   		
	   Kunde involvering 		
	   		& Meget kunde involvering, bl.a. til at styre product backlog.
	   		& Vi har ikke en kunde, men der er potentielle brugere som kan interviewes for at finde frem til deres krav til et produkt.
	   		& Vi kan selv finde kravene og danne user stories, men skal derfor også selv styre product backlog.\\
	   		
	   Kommunikationsmetoder	
	   		& Daily Scrum, Sprint review meeting, Scrum master, kontakt med kunder og administration fra virksomheden.
	   		& Gruppen har et grupperum på universitet, og skal ikke tale med administration. 
	   		& Scrum master rollen kan gå på runde fra sprint til sprint eller fra uge til uge. \\
    \end{tabular}
  \caption{Sammenligningstabel over Scrum og vores projekt.}\label{tabel:sammenligningstabel}
\end{sidewaystable}

Der skal laves ændringer på den traditionelle Scrum process  for at tilpasse det til projektet, men disse kan håndteres og ændres, som \myref{tabel:sammenligningstabel} også viser.

\section{Objekt Orienteret Analyse \& Design}

OOA\&D er en analysemtode som har til formål at finde krav til et system, fastlægge design, og at danne en forståelse for et system, dets omgivelser og hvilke krav der skal til for at implementere det.
Metoden stammer fra bogen af samme navn.\citep{OOA&D2001}
Den gør brug af UML til at beskrive struktur og klasser indenfor området.
Herudover finder metoden:

\begin{itemize}
\item Systemdefinition
\item Hændelser
\item Adfærd for klasserne
\item Funktioner
\item Tilstande i systemet
\item Aktører og deres brugsmønstre til systemet.
\end{itemize}

Med alle disse elementer kan laves et analyse- og designdokument.
Med disse dokumenter kan man nemmere lave et godt og velovervejet design.
Dog har vi valgt ikke at finde adfærd for klasserne, da dette var set som værende overflødigt.
Det beskriver klasserne, hvornår de laves i systemet, og gennemgangen af alle klassernes hændelser. 
Dokumentationen for disse fylder utroligt meget og selve adfærden er et mellemtrin i metoden, til at danne en god struktur, derfor dokumenteres strukturen i stedet.
De andre analyse dele er meget centrale og hjælper med at beskrive systemet og er derfor taget med. 

Scrum og OOA\&D hjælper projektet på hver sin måde.
Scrum som projektstyring og for at overskueliggøre projektets opgaver. 
En user story ved navn "Rapport" laves, og ud fra denne dannes der opgaver som skal laves i hvert sprint. 
Dette sørger for at rapporten ikke falder bagud i forhold til systemet udvikling.
OOA\&D hjælper derimod med at organisere og strukturere analysen som beskrevet i dette afsnit, og derfor komplimentere de to metoder hinanden udemærket.
Dog fordi Scrum benytter user stories, vil den endelige kravspecifikation ikke være beskrevet på traditionel OOA\&D vis, men i stedet vha. user stories.









