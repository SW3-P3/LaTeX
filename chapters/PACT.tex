\chapter{PACT}

\section{Personer}
Majoriteten af personerne, som anvender systemet, har interesse i at lave mad i husstanden, og hjælpe med indkøb til madlavningen.
Brugerne af systemet er hovedsageligt voksne mennesker, hvilket betyder, at alle har et grundlag for at læse simpel tekst i systemet.
Der tages hermed ikke hensyn til blinde, og folk med voldsomme læsevanskeligheder.
Personerne, der benytter systemet, forventes ligeledes at kunne forstå dansk.
Mere specifikt er personerne meget forskellige.
Det kan både være enlige, samboende, unge og gamle, og systemet kan også bruges af hele familier.
Personerne kan også være kræsne eller have forskellige krav(herunder bl.a. allergier) og præferencer, i forhold til mad og madvaner.

\section{Aktiviteter og kontekst}
Aktiviteterne er delt op i to dele, da der vil være to delløsninger. 
Den ene vil være delen med opskrifter og tilbudssøgning, imens den anden er indkøbslisten.
Samtidigt er konteksten og delt op i to: her handler det om, at systemet kan bruges i hjemmet, men også at det kan bruges i selve indkøbsøjeblikket, altså i supermarkedet.
Systemet skal kunne tilgås let som indkøbsliste, og således også være let brugbart i et indkøbscenter, mens der handles ind.
Indkøbslisten skal derfor være nem at komme til og synkronisere i realtime.
Man skal kunne sætte ting på indkøbslisten i det øjeblik, man kommer i tanke om, at man mangler dem.
I konteksten af hjemmet, behøver systemet ikke være lige så øjeblikkelig, da man som oftest har mere tid til at tænke over sine valg.
Dog er det altid vigtigt, at systemet er responsivt i alle kontekster og ved alle aktiviteter, for ikke at ødelægge den pågældende aktivitets flow. 

\section{Teknologier}
For at et software system skal være brugbart som indkøbsliste, kræves det, at platformen er portabel.
Der sættes således et krav om, at systemet vil være brugbart på en mobil platform som smartphones.
Yderligere indebærer problemet et fokusområde, som omhandler tilbud, opskrifter og præferencer.
Disse vil oftere være nemmere tilgængeligt via en enhed med en større skærm, såsom en computer eller tablet - men vil med fordel også kunne tilgås ved brug af en smartphone. 	