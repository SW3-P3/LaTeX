\chapter{State of the art}
I dette afsnit vil der blive kigget på hvilke løsninger der allerrede findes på markedet.
Der vil blive kigget på web services og applikationer der har til samme formål at løse nogle af de samme problemer vi vil undersøge nærmere.
Løsningerne der er præsenteret, er nogle af de løsninger der løser problemet i problemområdet bedst. 
\section{eTilbudsavisen}
eTilbudsavis er en online service der kan findes på \underline{www.eTilbudsavis.dk}. eTilbudsavisen er en online avis med en høj funktionalitet og en nem brugergrænseflade.
Der kan på siden oprettes et login, således der kan findes tilbage til ændringer på et senere tidspunkt eller andre enheder.
eTilbudsavisen har tre mærkbare funktioner som brugeren har adgang til, hvilket er tilbudsaviser, lister\fxnote{Lister af hvad? Indkøbsliter ? - Søren} og tilbud.\\
Tilbudsaviser kan tilgås på siden, hvor der er mulighed for at sætte en præference på, hvor store en radius man vil lede efter sagte tilbudsavis. 
Der kan der vælges imellem alle aviser, som er tilgængelige online og inden for den valgte radius. 
Aviserne bliver opdateret løbende, så når der er en ny tilgængelig bliver den gamle fjernet. 
Inde i aviserne kan man trykke på en vare og den bliver tilføjet til en liste.\\
Der er mulighed for at tilføje vare til to lister, indkøbsliste og ønskeliste. 
Når en vare er valgt bliver den tilføjet til den valgte liste.
Listen indeholder navn, butik, pris og den valgt mængde, for varen. 
Der kan foruden at vælge varer fra tilbudsaviserne, nemt skrives generiske varer på listen. Varen på listen kan da krydses af for at kunne holde styr på hvad der er blevet købt.\\ 
Hvis det ikke ønskes at skulle bladres tilbudsaviser igennem, er der den mulighed at få vist en hel side kun med tilbud. 
Alle aktuelle tilbud fra aviserne, er da vist som elementer med navn, beskrivelse, pris, butik, og afstand. Disse tilbud kan på samme måde nemt tilføjes til listerne.\\
eTilbudsavisen er en god online løsning som har mange gode features, derfor har vi også brugt dem som kilde til vores tilbud, dette vil blive beskrevet senere.\fxnote{Sæt her kilde til afsnit vedr. API fra etilbudsavisen. - Søren}
%\section{Fakta \& Føtex App}
\section{Smartphone Apps udgivet af butikker}
	Der findes som minimum Android apps de 7 butikker, som er med i tabellen \myref{tbl-smartphone}. 
	For at skabe et overblik er der udvalgt features og disse er blevet opsat i et skema for overskuelighed. 
	\begin{table}[H]
		\label{tbl-smartphone}
		\colorlet{shadecolor}{gray!40}
    	\rowcolors{1}{white}{shadecolor}
	    \begin{tabular}{l|lllllllllll}
	    %Table: http://bit.ly/1tD6EI6
	   	%Funktionalitet & Tilbudsavis & Indkøbsliste & Opskrifter & Varescan & Find butik & Budget & Madplan & Rabatkupon/fordelsordning & Deling af indkøbslister & Rating på Play & Senest opdateret \\ \hline
	    Funktionalitet & \rot{Tilbudsavis} & \rot{Indkøbsliste} & \rot{Opskrifter} & \rot{Varescan} & \rot{Find butik } & \rot{Budget} & \rot{Madplan} & \rot{Rabatkupon} & \rot{Deling} & \rot{Play rating} & \rot{Senest opdateret} \\ \hline
	   	Føtex                       & \cmark   & \cmark    & \cmark  & \cmark   & \cmark  & ~      & ~       & ~          & ~                       & 3.4 (354)      & 2014-07-24       \\
	    SPAR                        & \cmark   & \cmark    & \cmark  & ~        & \cmark  & \cmark & ~       & ~          & ~                       & 2.8 (64)       & 2014-05-15       \\
	    Fakta                       & \cmark   & \cmark    & \cmark  & ~        & \cmark  & ~      & \cmark        & \cmark     & \cmark               & 3.1 (454)      & 2014-08-02       \\
	    FaktaQ                      & \cmark   & ~         & \cmark  & ~        & \cmark  & ~      & ~       & ~          & ~                       & 4.4 (7)        & 2014-03-11       \\
	    REMA 1000                   & \cmark   & \cmark    & \cmark  & ~        & \cmark  & ~      & ~       & ~          & ~                       & 3.5 (674)      & 2014-04-16       \\
	    SuperBrugsen                & \cmark   & \cmark    & \cmark  & ~        & \cmark  & ~      & ~       & ~          & ~                       & 3.8 (987)      & 2014-06-30       \\
	    Kvickly                     & \cmark   & \cmark    & \cmark  & ~        & \cmark  & ~      & ~       & ~          & ~                       & 3.7 (632)      & 2014-07-20       \\
	    \end{tabular}
	    \caption{Denne tabel viser hvilke smartphone apps som har hvilke funktionaliteter.}
	\end{table}
	Vi har valgt at undersøge en af appsne nærmere, og valget faldte på Faktas app.
	Da den bliver opdateret og har flest features. 
	\subsection{Faktas Android App}\fxnote{Screenshot?}
		Faktas android app kalder de for ``Mit fakta'', den er som helhed overskuelig.
		Efter åbning af appen har man mulighed for 6 menupunkter og at ændre sine instillinger.
		Første menupunkt omhander ``Coop-kortet'' og giver bruger mulighed for at indtaste sine medlemsinfomationer, da disse gives særlige tilbud.
		Andet menupunkt er deres tilbudsavis, hvilket blot er en digital kopi af den fysiske tilbudsavis. 
		Dog kan man fra den også tilføje varer til sin indkøbsliste, eller se varene i et gitterformat.
		Tredje menupunkt er ``indkøbsliste'', her kan man have flere personlige og/eller delte indkøbslister.
		Her kommer deres Facebook integration også i spil, og tillader nem deling af indkøbsliste med brugerens Facebook venner.
		Fjerde menupunkt er en madplan, hvori man kan planlægge sin egen madplan, eller se Faktas anbefalinger til en under 20 kr pr. person pr. aften.
		Femte menupunkt er deres opskrifter.
		Her findes der et stort antal opskrifter, disse kan tilføjes direkte til sin madplan eller indkøbslister (både personlige og delte).
		Sjette menupunkt hedder ``Åbningstider'', her kan brugeren finde Faktas butikker og deres åbningstider. 

\section{Tilbudsugen}

Tilbudsugen minder på mange måder om eTilbudsavisen. 
Den har samtlige dagligvareaviser, samt flere inden for bl.a. Byggemarkeder, og Autoudstyr.
De giver et nemt overblik over diverse aviser, og man kan hurtigt og nemt læse dem på nettet. 
Der er desuden mulighed for at lave præferencer som ved etilbudsavisen, her kan man bl.a. vælge økologi eller nøglehulsmærket.
Når man tilføjer en vare til indkøbslisten, søger den automatisk efter tilbud på den valgte vare. 
Man bliver bedt om at vælge et specifikt tilbud, og netop dette tilbud bliver tilføjet til indkøbslisten, med pris, butik, udløbsdato, mængde, og et billede af varen.
Man kan som i etilbudsavisen også trykke på en vare direkte i avisen, for at tilføje den til sin liste.
Hvis man vil dele sin indkøbsliste er det også en mulighed vha. en "delekode" som man kan give til en anden bruger, og de kan på denne måde også se listen.
Funktionerne findes på hjemmesiden, men det er ikke altid de virker, f.eks. hvis man tilføjer noget uden at angive et antal, og du så prøver at dele listen med en, vil de ikke være at finde på listen.
Desuden kan man ikke ændre på antallet af varen du allerede har sat på din indkøbsliste. 
For at opnå dette skal man slette varen, og tilføje den igen med det nye antal.
De har desuden også en smartphone app, men denne crasher ofte når man benytter sig af deres indkøbsliste, men fungerer tilgengæld fint hvis man blot vil se på ugens tilbud i aviserne.


\section{``Tøm køleskabet''}
Der findes talrige tjenester, der tilbyder “at tømme dit køleskab”. 
Mere specifikt, tilbyder de en service, hvor du som bruger, angiver hvilke varer dit køleskab pt. indeholder, samt hvilke andre ingredienser du har til rådighed. 
Derefter får du så præsenteret en række forskellige opskrifter, der kan laves ud fra dine tilgængelige ingredienser. 
De fleste af tjenesterne (herunder dem vi her har undersøgt) viser også opskrifter, som indeholder yderligere ingredienser. 
Dette betyder naturligvis, at man som bruger ikke bliver fritaget fra at handle ind, hvis man mangler nogle ingredienser til lige netop den opskrift, man vælger at udføre. 
Tjenesten \textit{MyFridgeFood} (kan findes på www.myfridgefood.com) tilbyder, at oprette en indkøbsliste ud fra netop disse manglende ingredienser, hvilket kan lade sig gøre blandt andet fordi, alle opskrifter er interne på \textit{MyFridgeFood}. 
I modsætning til dette er der \textit{Supercook} (kan findes på www.supercook.com), som linker til eksterne opskrifter, og ikke tilbyder at generere en indkøbsliste. 
Umiddelbartt anbefales opskrifter ikke ud fra den enkelte brugers smag og madvaner, men udelukkende på baggrund af, hvad man “har i køleskabet”. 
Grundet dette virker tjenesterne mere som simple filtreringer af databaseopslag, end egentlige anbefalinger, der tager højde for brugerens smag og præferencer indenfor den gastronomiske verden.


\section{Osuma}

\section{Link uden navn}