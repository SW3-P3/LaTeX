\chapter{State of the art}
I dette afsnit vil der blive kigget på hvilket løsninger der allerrede findes på markede. Der vil blive kigget på web services og applicationer der har til samme formål at løse nogle af de problemer vi har fundet frem til i problemområdet. Løsningerne der er præsenteret, er nogle af de løsninger der løser problemt i problemområdet bedst. 
\section{eTilbudsavisen}
eTilbudsavis er en online servise der kan finde på \underline{www.eTilbudsavis.dk}. eTilbudsavisen er en online avise med en høj funktionalitet og en nem brugergrænseflade. Der kan på siden oprettet et login, således der kan findes tilbage til ændringer på et enere tidspunkt eller andre enheder. eTilbudsavisen har tre mærkbare functioner som brugeren har adgang til, hvilket er tilbudsaviser, lister og tilbud.\\
Tilbudsaviser kan tilgået på af siden, hvor der er mulighed for at sætte en præferance på, hvor store en radius man vil lede efter sagte tilbudsavis. Der kan da vælges imellem alle aviser, som er tilgængelige online og inden for den valgte radius. Aviserne bliver opdateret løbene, så lige så snart der er en ny tilgændelig bliver den gamle fjernet. Inde i aviserne kan man trykke på en vare og den bliver tilføjet til en liste.\\
Der er mulighed for at tilføje vare til to lister, indkøbsliste og ønskeliste. Nåren en vare er valgt bliver den tilføjet til en valgt liste, på listen har man da navn, butik, pris og valgt mængde. Der kan foruden at vælge vare fra tilbuds aviserne, nemt skrives generiske vare på listen. Varen på listen kan da krydes af for at kunne holde styr på hvad der er blevet købt.\\ 
Hvis det ikke ønskes at skulle bladre tilbudsaviser igennem, er der den mulighed at få vist en hel side kun med tilbud. Alle aktuelle tilbud fra aviserne, er da vist som elementer med navn, beskrivelse, pris, butik, og afstand. Disse tilbud kan på samme måde nemt tilføjes til listerne.\\
eTilbuds avisen er en rigtig god online løsning som har mange gode features, derfor har vi også senere i rapporten brugt dem som kilde til vores tilbud, dette vil blive beskrevet senere.
%\section{Fakta \& Føtex App}
\section{Smartphone Apps udgivet af butikker}
	Der findes som minimum smartphone apps til butikkerne: 
	\begin{table}
		\colorlet{shadecolor}{gray!40}
    	\rowcolors{1}{white}{shadecolor}
	    \begin{tabular}{l|lllllllllll}
	    %Table: http://bit.ly/1tD6EI6
	   	%Funktionalitet & Tilbudsavis & Indkøbsliste & Opskrifter & Varescan & Find butik & Budget & Madplan & Rabatkupon/fordelsordning & Deling af indkøbslister & Rating på Play & Senest opdateret \\ \hline
	    Funktionalitet & 1 & 2 & 3 & 4 & 5 & 6 & 7 & 8 & 9 & 10 & 11 \\ \hline
	   	Føtex                       & \cmark   & \cmark    & \cmark  & \cmark   & \cmark  & ~      & ~       & ~          & ~                       & 3.4 (354)      & 2014-07-24       \\
	    SPAR                        & \cmark   & \cmark    & \cmark  & ~        & \cmark  & \cmark & ~       & ~          & ~                       & 2.8 (64)       & 2014-05-15       \\
	    Fakta                       & \cmark   & \cmark    & \cmark  & ~        & \cmark  & ~      & \cmark        & \cmark     & \cmark               & 3.1 (454)      & 2014-08-02       \\
	    FaktaQ                      & \cmark   & ~         & \cmark  & ~        & \cmark  & ~      & ~       & ~          & ~                       & 4.4 (7)        & 2014-03-11       \\
	    REMA 1000                   & \cmark   & \cmark    & \cmark  & ~        & \cmark  & ~      & ~       & ~          & ~                       & 3.5 (674)      & 2014-04-16       \\
	    SuperBrugsen                & \cmark   & \cmark    & \cmark  & ~        & \cmark  & ~      & ~       & ~          & ~                       & 3.8 (987)      & 2014-06-30       \\
	    Kvickly                     & \cmark   & \cmark    & \cmark  & ~        & \cmark  & ~      & ~       & ~          & ~                       & 3.7 (632)      & 2014-07-20       \\
	    \end{tabular}
	\end{table}

\section{Tilbudsugen}

\section{``Tøm køleskabet''}

\section{Osuma}

\section{Link uden navn}