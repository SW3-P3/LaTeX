\chapter{Konceptuelle scenarier}

\begin{description}

  \item[1. Tilgå systemet] \hfill \\
For at tilgå vores system kræver det en enhed med internetadgang; dette kunne typisk være en eller flere af følgende: computer, tablet, smartphone.
For at benytte alle funktioner, heriblandt en indkøbsliste man kan tage med sig, kræver det et mobilt apparat med internetforbindelse, her ville en smartphone være mest oplagt.
  
  \item[2. Finde opskrifter til brug] \hfill \\
Brugeren skal kunne benytte en database over opskrifter, der findes i systemet. Hvis brugeren har valgt nogle bestemte præferencer i forhold til opskrifter, dette kunne eksempelvis være fiskefri opskrifter, så vil databasen kun vise de relevante opskrifter. 
Når brugeren har benyttet en opskrift, eller hvis brugeren kender en opskrift i forvejen, kan brugeren vælge at bedømme denne. Ud fra en række bedømmelser vil systemet finde anbefalinger på opskrifter, som forventes at falde i brugerens smag.
  
  \item[3. Finde tilbud til indkøb] \hfill \\
Personer med adgang til systemet vil kunne søge aktuelle tilbud i vores service, disse tilbud vil kunne personliggøres ud fra en persons præferencer.
Personen vil også have mulighed for at sætte alarmer eller overvågning op på forskellige varer eller kategorier. 
Når personen har fundet et tilbud, vil det kunne tilføjes til en elektronisk indkøbsliste.

  \item[4. Benytte indkøbsliste] \hfill \\
Personer, der benytter systemet, skal kunne tilføje varer til deres indkøbslister, enten via tilbuds-funktionen, tekst-input eller gennem ingredienslisten fra en opskrift.
Indkøbslister skal kunne deles i husstanden, så man kan oprette fælles indkøbslister.    

  \item[5. Sætte præferencer] \hfill \\
En bruger af systemet skal kunne sætte præferencer for opskrifter og tilbud, hvis en bruger eksempelvis kun vil se tilbud på økologisk, har allergier eller er vegetar, kan dette sættes under funktionen 'Præferencer'.

\end{description}