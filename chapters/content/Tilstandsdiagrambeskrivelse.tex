Tilstandsdiagrammet, \myref{tilstandsdiagram}, viser de forskellige stadier, systemet kan være i samt hvilke funktionaliteter, der er tilgængelige fra disse stadier.
Diagrammet hjælper med at danne overblik over navigeringen i systemet, og hvilken proces der opstår under udførelse af diverse brugsmønstre.
Markeret med grønt, ses de fem hovedmenuer.
Efter en bruger er logget ind, kan disse hovedmenuer til enhver tid tilgås.
Menuen er repræsenteret gennem den længere sorte centerlinje.
Alle pile, der går ind her, giver adgang til alle de pile, der går ud.
Pilene repræsenterer en hændelse igangsat af brugeren.
En klynge kan til hver en tid, uanset tilstand, tilgå hændelser, der går ud af klyngens boks.
Herfra kan man således se, at menuen altid er tilgængelig, såvel som logud funktionen.

Indikeret med pile kan man se, hvilke funktionaliteter der er tilgængelig fra en given tilstand, samt om disse er sekventielle eller iterative hændelser.
En sekventiel funktion skifter tilstanden på systemet, mens en iterativ blot opdaterer systemets information, mens den forbliver i tilstanden.
Eksempelvis har den grønne menu-boks \textit{Præferencer} en iterativ funktion på sig, da denne blot opdaterer information i systemet frem for at skifte tilstand i modsætning til funktionen \textit{Vælg opskrift}, som er sekventiel.
Hvis brugeren fra menuen \textit{Se opskrifter} giver brugerinputtet \textit{Vælg opskrift}, følges sekvensen til en ny tilstand \textit{Opskrift valgt}. 
Herfra åbnes således op for nye funktionaliteter for brugeren.
Hele diagrammet følger disse to typer af hændelser, som er gældende for alle tilstande og funktioner.
