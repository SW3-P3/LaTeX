Tilstandsdiagrammet, \myref{}, viser de forskellige stadier systemet kan være i, samt hvilke funktionaliteter er tilgægngelige fra disse stadier.
Markeret med grønt, ses de fem hovedmenuer. 
Efter en bruger er logget ind, kan disse til hver en tid tilgås gennem en menu.
Menuen er reprsenteret gennem den længere sorte centerlinje,  alle pile der går ind her, giver adgang til alle de pile der går ud, pilene repræsentere brugerinput, således kan et menuskifte gennem denne linje altid foretages.
Centerlinjen fungerer som en samling af punkter, den betyder blot at alt der går ind, kan gå ud alle stedet, samme betydning har den tykkere linje, som ses på venstre del i diagrammet, de funktioner der går ind her, skifter til samme tilstand.
Indikeret med pile, kan man se hvilke funktionaliteter der i tilgængelig fra en given tilstand, samt om disse er sekventielle eller iterative.
En sekventiel funktion skifter tilstanden på systemet, mens en iterativ blot opdatere systemets information, mens den forbliver i tilstanden.
Eksempelvis den grønne menu-boks ''Præferencer''.
Denne har en iterativ funktion på sig, da denne blot opdatere information i systemet, frem for at skifte tilstand i modsætning til funktionen ''Vælg opskrift'', som er sekventiel.
Hvis brugeren fra menu'en ''Se opskrifter'' giver brugerinputtet ''Vælg opskrift'' følges sekvensen til en ny tilstand ''Opskrift valgt'' Herfra åbnes således op for nye funktionaliteter for brugeren.
Hele diagrammet følger disse to processer, som er gældende for alle tilstande of funktioner.