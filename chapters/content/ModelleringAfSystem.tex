\section{Modellering af System}
Der to komplementære synsvinkler for den model, som beskriver systemet. 
Dette er hhv. at systemet modellerer noget, problemområdet, samt det anvendes af brugerne, anvendelsesområdet. 
Disse er defineret som følger:

\textit{\textbf{Problemområde:} ''Den del af omgivelserne, der administreres, overvåges eller styres ved hjælp af et system''}

\textit{\textbf{Anvendelsesområde:} ''En organisation, der administrerer, overvåger eller styrer et problemområde''}\citep{OOA&D2001}
\subsection{Problemområdet}
Problemområdet bruges som en modellering af et problem fra den virkelige verden, hvor et system skal benyttes for at administrerer, overvåge eller styre et område. 
Dette gøres ved at beskrive diverse klasser, som vil indgå i systemet, ud fra disse ses på hvilke hændelser, som er involveret i klasserne.
Ydermere ses der på hvilken adfærd der er mellem diverse klasser, hændelser og objekter i systemet, dette giver en beskrivelse af hvilken opførsel og struktur problemområdet skal modellere.
I dette projekt omfatter problemområdet planlægning af indkøb, inspiration til mad såvel som at spare penge.
For at beskrive dette nærmere, ses der på klassediagrammer, tilstandsdiagrammer og hændelsestabeller, for at danne overblik over systemet og ende ud med en sammenhængende model for problemområdet.
\subsection{Anvendelsesområdet}
Hvor problemområdet beskriver systemet, beskriver anvendelsesområdet hvordan systemet skal anvendes.
Ud fra dette spørgsmål opstilles en række af krav for systemets funktioner og grænseflade.
Til dette formål ses der på brugen af systemet, hvilke typer af brugere er der, hvilke brugsmønstre er der for de individuelle funktionaliteter i systemet, samt hvordan skal funktionaliteterne tilgås fra grænsefladen.
I dette projekt indebære anvendelsesområdet at kunne administrere sine indkøb, få en let oversigt over tilbud, inspiration til mad i form af opskrifter, at kunne overvåge vare, som man har interesse for at købe på tilbud.

