\section{Analyse af problemområde}

Ud fra systemdefinitionen i \myref{Sysdef} ved vi at systemet skal holde styr på følgende:

\begin{itemize}
	\item Tilbud
	\item Varer
	\item Indkøbslister
	\item Opskrifter
\end{itemize}

Med disse informationer kan systemet hjælpe brugeren til at finde billige varer i bestemte butikker, og eventuelt anbefale opskrifter der bruger disse tilbudsvarer.
I de følgende afsnit vil disse emner blive beskrevet vha. klassebeskrivelser, en hændelsestabel, og et klassediagram.

\subsection{Klasser}
Her vil klassernes sammenhæng analyseres, derudover vil yderligere klasser tilføjes hvis nødvendigt.

\textbf{Vare:}
En vare indgår i opskrifter, og indkøbslister.
Når man laver sin indkøbsliste kan man vælge varer man vil købe, og tilføje dem til indkøbslisten.
Desuden kan en vare have et antal tilbud, hvilket betyder at der også skal laves en relation til tilbudsklassen.

\textbf{Tilbud:}
Når der kommer nye varer på tilbud modelleres disse og kobles vha. en association til varer.

\textbf{Opskrift:}
En opskrift har en liste over ingredienser, hvilket altså er varer, samt mængden af varen.
I interviewene i \myref{section:interview2}, blev det nævnt at brugerne gerne ville kunne vurdere en opskrift, og dermed få anbefalet yderligere opskrifter som minder om denne.
For at kunne lave vurderinger skal der laves en vurderingsklasse.

\textbf{Vurdering:}
En vurdering ved tal gives for at rangere opskrifter. 
Vurderingen danner også grundlag for at systemet kan anbefale opskrifter.

\textbf{Anbefaling:}
En anbefaling, af en opskrift, kan gives til personer når de har givet positive vurderinger af andre opskrifter, som minder om den vurderede opskrift.

\textbf{Person:}
Personklassen gør det muligt at holde styr på forskellige personer, da disse tilsluttes opskrifter, vurderinger, og indkøbsliter.
Derudover vil en person også have præferencer for butikker de handler i, såvel som madvarer.

\textbf{Indkøbsliste:}
Indkøbslisten laves af en person, og fyldes op med objekter fra vareklassen.
Indkøbslisterne kan deles mellem flere brugere.

\subsection{Hændelser}\label{handelser}
På baggrund af de nævnte funktionaliteter i prototype interviewene, \myref{section:interview2}, er der fundet forskellige hændelser, relevante for funktionaliteterne.
Ud fra disse laves en hændelsestabel, der beskriver hvilke klasser forskellige hændelser påvirker.
Formålet med at identificere hændelserne samt at analysere disse i en hændelsestabel, er at forstå problemområdet bedre og dermed hjælpe med forståelsen for hvordan en løsning ville kunne designes for at afhjælpe de problemer der findes i problemområdet. Desuden kan tabellen hjælpe med strukturen på klasserne.
Hvis to klasser har samme hændelser, kan disse klasser ofte tilpasses under en klasse og dermed opnå en bedre struktur.

\begin{table}[H]
  \centering
    \colorlet{shadecolor}{gray!40}
    \rowcolors{1}{white}{shadecolor}
      \begin{tabular}{l|lccccccc}
      %\hline
                     				& \rot{Tilbud} 					 & \rot{Indkøbsliste} & \rot{Tilbudsavis} & \rot{Opskrift} & \rot{Vare} & \rot{Bruger} \\ \hline
      Vare tilføjet                 &                                                         & \cmark                           &                           &                               & \cmark                   & \cmark                     \\ 
      Vare fjernet                  &                                                        & \cmark                           &                           &                               & \cmark                   & \cmark                     \\ 
      Vare aftjekket                &                                                         & \cmark                           &                           &                               & \cmark                   & \cmark                     \\ 
      Opskrift valgt                & \cmark                                                & \cmark                           &                           & \cmark                       & \cmark                   & \cmark                     \\ 
      Opskrift udført               &                                                     &                                   &                           & \cmark                       &                           & \cmark                     \\ 
      Tilbud ankommet               & \cmark                                 & \cmark                           & \cmark                   &                               & \cmark                   &                             \\ 
      Tilbud udgået                 & \cmark                                    & \cmark                           & \cmark                   &                               & \cmark                   &                             \\ 
      Vare tilføjet til overvågning & \cmark                                      &                                   &                           &                               & \cmark                   & \cmark                     \\ 
      Vare fjernet fra overvågning  & \cmark                                             &                                   &                           &                               & \cmark                   & \cmark                     \\ 
      Del indkøbsliste              &                                                  & \cmark                           &                           &                               &                           & \cmark                     \\ 
      Indkøbsliste oprettet         &                                                 & \cmark                           &                           &                               &                           & \cmark                     \\ 
      Indkøbsliste slettet         &                                                & \cmark                           &                           &                               &                           & \cmark                     \\ 
    \end{tabular}
  \caption{Hændelsestabel. Viser hvilket klasser, problemområdets hændelser påvirker.}\label{tabel:haendelsestabel}
\end{table}


Hændelsestabellen i \myref{tabel:haendelsestabel} viser både hvilke hændelser der findes i problemområdet, samt hvilke klasser de påvirker.
Hvis tabellen læses vandret kan det ses at klasser der bliver påvirket af mange hændelser er klasser som \textbf{Indkøbsliste}, \textbf{Vare} og \textbf{Person}.

Ud fra analysen indtil nu kan der dannes et overblik, over klassernes interne interaktion, såvel som hvilke hændelser der involverer hvilke klasser.
Denne information kan vi nu bruge til at lave en struktur over klasserne i problemområdet.
\fxnote{Vi bruger denne osm værktøj til at analysere os frem til en struktur. Dette beskriver vi endda. Dvs. det eneste det gør er at vise os evt. fejl i vores klasseopgivelse. Skal vi evt. slette det, og blot nævne vi har brugt den, og fortælle hvad den viste? - Søren}

\subsection{Struktur}\label{sec:struktur}
\begin{figure}
	\centering
		\includegraphics[scale=0.6]{images/Diagrams/klassediagram_model_simple.pdf}
	\caption{Klassediagram over problemområdet.}
	\label{figur:PDklasse}
\end{figure}

Klassediagrammet som ses på \myref{figur:PDklasse}, beskriver forholdet mellem de forskellige klasser som findes i problemområdet.
Diagrammets sammenhænge er dannet ud fra hændelsestabellen, og beskrivelserne af klasserne.\fxnote{Sørg for at denne model stemmer over ens med modellen vi ender ud med i programmet. I programmet kan vi så henvise til at denne model over klasserne blev brugt til at danne associationerne, imellem klasserne. Der er nemlig ingen henvendelse lige pt, og det kan helt klart bruges. Beskrivelsen nedenfor er næsten identisk med den man finder i Arkitektur - Søren}

Personer kan lave indkøbslister, disse indkøbslister kan være ejet og administreret af en enten en eller flere personer.
En indkøbsliste kan bestå af nul til mange varer.
En vare kan være på tilbud i mere end en butik, og derfor have nul til mange tilbud.
En vare kan desuden indgå i en opskrift, og er derfor forbundet med nul til mange.
Desuden har opskrifterne en til mange varer på listen over ingredienser.
Varen kan også være tilføjet til overvågningslisten hos en person, så derfor har personer og varer en nul til mange relation på hinanden.
En person i problemområdet kan give hver opskrift en vurdering, derfor kan personen give nul til mange vurderinger.
En vurdering gives til en opskrift alene, imens en opskrift kan have mange vurderinger, eller ingen vurderinger.
Anbefalinger består af en opskrift, imens personen kan modtage nul til mange anbefalinger.


Ovenstående analyse vil hjælpe til at designe systemets implementering, først foretages dog en analyse af anvendelsesområdet, for at undersøge hvad der er muligt at foretage sig i systemet.
