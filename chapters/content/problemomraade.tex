\section{Problemområde}

Ud fra systemdefinitionen ved vi at systemet skal holde styr på følgende:

\begin{itemize}
	\item Tilbud
	\item Varer
	\item Opskrifter
\end{itemize}

Med disse informationer kan systemet hjælpe brugeren til at finde billige varer i bestemte butikker, og eventuelt anbefale opskrifter der bruger disse tilbudsvarer.
I de følgende afsnit vil disse emner blive beskrevet vha. klassebeskrivelser, en hændelsestabel, og et klassediagram.

\subsection{Klasser}
For at kunne håndtere de 3 allerede nævnte klasser, tilbud, varer og opskrifter, skal der være flere for at danne en sammenhæng.
Denne sammenhæng vil analyseres her.

\textbf{Vare:}
En vare indgår i opskrifter, og indkøbslister.
Når man laver sin indkøbsliste vælger man hvilke varer man vil købe og skriver dem derpå. 
Desuden kan en vare have et antal tilbud hver uge, hvilket betyder at der også skal laves en tilbuds klasse.

\textbf{Tilbud:}
Hver uge kommer der nye tilbud fra de fleste danske dagligvarebutikker. 
Disse modeleres og kobles på varen og dermed dannes der en kobling fra varen til tilbudene.

\textbf{Opskrift:}
En opskrift har en liste over ingredienser, hvilket altså er varer.
I interviewene i \myref{section:interview2}, blev det nævnt at brugerne gerne ville kunne vurdere en opskrift, og dermed få anbefalet yderligere opskrifter som minder om denne.
Dette leder til at der laves en såkaldt vurderingsklasse.

\textbf{Vurdering:}
Vurderinger bliver foretaget når en bruger af systemet har lavet en opskrift, og vil give den en vurdering, både til at hjælpe andre, men også for at få lignende opskrifter anbefalet, hvis man nu synes den var god.

\textbf[Anbefaling:]
En anbefaling bliver givet til en person når der er givet vurderinger, således der kan laves en anbefaling af en opskrift til personen.

\textbf{Person:}
Vurderingerne skal kobles på personen som vurdere opskriften. 
Derfor laves der en person klasse, som samtidig indholder attributter som gør det muligt at sætte personens præferencer.
Det er desuden også personerne som har deres egne indkøbsliter.

\textbf{Indkøbsliste:}
Indkøbslisten laves af personen, og fyldes op med objekter fra vare klassen.
Indkøbslisterne kan være delte, og derfor skal koblingen på personen klassen ikke være singulær, men altså 1..*.


\subsection{Hændelser}
Vi har fundet frem til de forskellige hændelser der sker i problemområdet.
Ud fra disse laves en hændelsestabel, der beskriver hvilke klasser forskellige hændelser påvirker.
Formålet med at identificere hændelserne samt at analysere disse i en hændelsestalbel, er at forstå problemområdet bedre og dermed hjælpe med forståelsen for hvordan en løsning ville kunne designes for at afhjælpe de problemer der findes i problemområdet. Desuden kan tabellen hjælpe med strukturen på klasserne.
Hvis 2 klasser har alle de samme hændelser, kan der ofte foretages ændringer og dermed opnå en bedre struktur.

\begin{table}[H]
  \centering
    \colorlet{shadecolor}{gray!40}
    \rowcolors{1}{white}{shadecolor}
      \begin{tabular}{l|lccccccc}
      %\hline
                     				& \rot{Tilbud} 					 & \rot{Indkøbsliste} & \rot{Tilbudsavis} & \rot{Opskrift} & \rot{Vare} & \rot{Bruger} \\ \hline
      Vare tilføjet                 &                                                         & \cmark                           &                           &                               & \cmark                   & \cmark                     \\ 
      Vare fjernet                  &                                                        & \cmark                           &                           &                               & \cmark                   & \cmark                     \\ 
      Vare aftjekket                &                                                         & \cmark                           &                           &                               & \cmark                   & \cmark                     \\ 
      Opskrift valgt                & \cmark                                                & \cmark                           &                           & \cmark                       & \cmark                   & \cmark                     \\ 
      Opskrift udført               &                                                     &                                   &                           & \cmark                       &                           & \cmark                     \\ 
      Tilbud ankommet               & \cmark                                 & \cmark                           & \cmark                   &                               & \cmark                   &                             \\ 
      Tilbud udgået                 & \cmark                                    & \cmark                           & \cmark                   &                               & \cmark                   &                             \\ 
      Vare tilføjet til overvågning & \cmark                                      &                                   &                           &                               & \cmark                   & \cmark                     \\ 
      Vare fjernet fra overvågning  & \cmark                                             &                                   &                           &                               & \cmark                   & \cmark                     \\ 
      Del indkøbsliste              &                                                  & \cmark                           &                           &                               &                           & \cmark                     \\ 
      Indkøbsliste oprettet         &                                                 & \cmark                           &                           &                               &                           & \cmark                     \\ 
      Indkøbsliste slettet         &                                                & \cmark                           &                           &                               &                           & \cmark                     \\ 
    \end{tabular}
  \caption{Hændelsestabel. Viser hvilket klasser, problemområdets hændelser påvirker.}\label{tabel:haendelsestabel}
\end{table}


Hændelsestabellen i Tabel \ref{tabel:haendelsestabel} viser både hvilke hændelser der findes i problemområdet, samt hvilke klasser de påvirker.
Hvis tabellen læses vandret kan det ses at klasser der bliver påvirket af mange hændelser er klasser som "Indkøbsliste", "Vare" og "Person".

Ud fra analysen indtil nu kan der dannes et overblik, over klassernes interne interaktion, såvel som hvilke hændelser, involverer hvilke klasser.
Denne information kan vi nu bruge til at lave en struktur over klasserne i problemområdet.


\subsection{Struktur}\label{sec:struktur}

\begin{figure}
	\centering
		\includegraphics[scale=0.6]{images/Diagrams/klassediagram_model_simple.pdf}
	\caption{Klassediagram over problemområdet.}
	\label{figur:PDklasse}
\end{figure}

Klassediagram som ses på Figur \ref{figur:PDklasse}, dette beskriver forholdet mellem de forskellige klasser som findes i problemområdet. Diagrammets sammenhænge er dannet ud fra hændelsestabellen, og beskrivelserne af klasserne. Følgende afsnit gennemgår diagrammets sammenhænge.

Personer i indkøbssituationer kan lave indkøbslister, disse indkøbslister kan været ejet og administreret af en enten en eller flere personer.
En indkøbsliste kan bestå af nul til mange varer.
En vare kan være på tilbud i mere end en butik, og derfor have nul til mange tilbud.
En vare kan desuden indgå i en opskrift, og er derfor forbundet med mul til mange.
Desuden har opskrifterne en til mange varer på listen over ingredienser.
Varen kan også være tilføjet til overvågningslisten hos en person, så derfor har personer og varer en nul til mange relation på hinanden.
En person i problemområdet kan, vurderer en opskrift, men kan give mange vurderinger, og derfor kan en person give nul til mange vurderinger. 
En vurdering gives til en opskrift alene, imens en opskrift kan have mange vurderinger, eller ingen vurderinger.
Anbefalinger består af en opskrift, imens personen kan have nul til mange anbefalinger.
