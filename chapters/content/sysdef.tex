\section{Systemdefinition}\label{sysdeffi}
På baggrund af Problemanalysen i \myref{chapter:problemanalyse} udarbejdes en BATOFF-analyse, som beskrevet i OOA\&D\citep{OOA&D2001}, og efterfølgende formuleres en systemdefinition baseret på disse kriterier.

Denne analyse har til formål at definere retningen for det videre arbejde i projektet.
Systemdefinitionen er en kort tekst, der har til formål at beskrive systemets overordnede krav og funktionaliteter.

Nedenfor ses BATOFF udarbejdelsen og efterfølgende systemdefinitionen.
Under arbejdet med med BATOFF og systemdefinitionen har vi, som  projektgruppe, taget nogle valg om af afgrænse problemområdet, i forhold til nogle af de løsninger der er blevet undersøgt og forslået, i hendholsvis State of the art og Prototype interviewsne.
Dette har vi gjort af flere grunde, vi mener at vi kan opnå en klarere systemdefinition, når man har færre funktionaliteter at forholde sig til.
Yderligere kan der med færre funktionaliteter sættes mere fokus på at udvikle en tilbudsorienteret indkøbsassistent, med opskrifter der danner inspiration til din madlavning.
BATOFF kriterierne hjælper med at danne et overblik over diverse emner, som en systemdefinition bør omfatte.

\subsection{BATOFF}
\begin{description}
\item [Betingelser]\hfill
\begin{itemize}[nolistsep,noitemsep]
\item Adgang til tilbud
\item Interesse for indkøb, opskrifter og tilbud
\end{itemize}

\item [Anvendelsesområde]\hfill
\begin{itemize}[nolistsep,noitemsep]
\item Tilbud
\item Brugere
\item Servere
\item Klienter
\item Overvågning af tilbud
\item Midler til lagring af data
\item Styring af anbefalinger
\end{itemize}

\item [Teknologi]\hfill
\begin{itemize}[nolistsep,noitemsep]
\item Smartphone (Mobil web-device)
\item Tablets
\item Til udvikling: Computer m/udviklerværktøjer
\item Browser m/internetadgang
\end{itemize}

\item [Objekter]\hfill
\begin{itemize}[nolistsep,noitemsep]
\item Opskrifter
\item Indkøbsvarer (tilbud)
\item Brugere
\item Vurdering
\item Præferencer
\item Indkøbsliste
\item Tilbudsaviser
\end{itemize}

\item [Funktioner]\hfill
\begin{itemize}[nolistsep,noitemsep]
\item Overvågning af tilbud
\item Håndtering af indkøbslister
\item Bedømmelse af opskrifter
\end{itemize}

\item [Filosofi]\hfill
\begin{itemize}[nolistsep,noitemsep]
\item Indkøbsassistent og inspirationsgenerering
\end{itemize}
\end{description}



\subsection{Konkret systemdefinition}\label{Sysdef}

Systemet hjælper på problemer, der kan opstå i forbindelse med indkøb og madlavning i hjemmet.
Systemet organiserer indkøbslister, opskrifter og aktuelle tilbudsvarer, samt anbefaler opskrifterne, baseret på bedømmelser af opskrifter og præferencer angående madvarer.
Aktuelle tilbud hentes fra internettet, og kan tilføjes, sammen med generiske varer, til indkøbslister.
Desuden kan det overvåge hvornår, en valgt generisk vare kommer på tilbud.
Systemet tilgås via en webbrowser, således det kan bruges på både computer, tablet og smartphone.
Systemet udvikles som en serverside-applikation med adgang til databaser til håndtering af system- og brugerdata.
Udviklingen af systemet kræver computere med de relevante udviklingsværktøjer.

Denne systemdefinition vil nu være udgangspunkt for vores videre arbejde i rapporten.
