\section{Analyse af anvendelsesområdet}\label{sec:anvendelses}
Dette afsnit tager udgangspunkt i metoder fra ''Objektorienteret Analyse og Design'' og benytter metoder herfra til at analyserer anvendelsområdet, dette omfatter brug af systemet, funktioner i systemet samt grænsefladen der er tilknyttet.\citep{OOA&D2001}
Afsnittet skal give et overblik over funktionaliteten af systemet, samt formidle hvordan brugeren interagerer med systemet.

\subsection{Brug}
Denne del af analysen har til formål at fastlægge interaktion mellem systemet og aktører.
Dette gøres ved at identificere brugsmønstre for aktørerernes aktioner i systemet.
\subsubsection*{Aktører}
I dette IT-system er der blevet identificeret to aktører.
Den første værende datakilden, eTilbudsaviss API, hvor tilbudende hentes fra og den anden værende de brugere, som benytter systemet.
For disse to aktører er der således udarbejdet en aktørtabel \myref{aktortabel}, der giver overblik over hvilke brugsmønstre der er, samt hvilke aktører der er relevante for disse.

\begin{table}[h]
\centering
    \colorlet{shadecolor}{gray!40}
    \rowcolors{1}{white}{shadecolor}
\begin{tabular}{rcc}
%\hline
				    & Bruger               		& eTilbudsavis  \\ \hline
Login               & \cmark                    & 		 		\\
Listehåndtering     & \cmark                    & 		 		\\
Søgning             & \cmark                    & 				\\
Indstil præferencer & \cmark                    & 				\\
Vurder opskrift     & \cmark                    &  	   			\\
Se anbefalinger     & \cmark                    & 				\\
Hent tilbud         &  						    & \cmark 		\\ \hline
\end{tabular}
\caption{Aktørtabel. Viser hvilke aktører er involveret i hvilke brugsmønstre}\label{aktortabel}
\end{table}

\subsubsection*{Bruger}

\textbf{Formål:} En person, som ønsker at bruge en eller flere af IT-systemets funktionaliteter, til at hjælpe med planlægning af mad og indkøb.

\textbf{Karakteristik:} Systemets brugere har meget varierende erfaring med IT-systemer, samt er bredt ud over mange forskellige aldersgrupper, majoriteten er dog mellem 18 - 30 år og har middel erfaring med IT.

\textbf{Eksempler:}Bruger A er en 21-årig universitetsstuderende, som for nyligt er flyttet hjemmefra. A har meget erfaring med IT, og bruger det dagligt til at navigere rundt på internettet.
A har let ved at navigere rundt i systemet og bruge dets funktionaliteter til at lave besparelser på det allerede lave budget, samt at undgå at få pasta med ketchup til aftensmad hver dag.

Bruger B er en 47-årig familiefar, der kun har smartphone, da dette er arbejdstelefonen givet af hans arbejdsgiver.
B benytter systemet til at handle ind på vej hjem fra arbejde, hvor indkøbslisten lavet af konen eller datteren bruges som guide i supermarkedet.
B vil således gerne kunne tilgå listen fra telefonen, så der ikke er behov for at køre hjem og hente den på papirsformat.

\subsubsection*{eTilbudsavis}

\textbf{Formål:} eTilbudsavis har til formål at gøre tilbudsdata tilgængelig for systemet, dette sker igennem deres application programming interface ("API").
Dette giver information om navnet på tilbudet, pris, periode, butik og meget andet. 
Dette vil blive beskrevet nærmere i \myref{api}.
Da dataene der hentes igennem API'et kan være af meget svingende kvalitet, filtreres det, så kun forståeligt tilbudsdata kommer igennem.

\textbf{Karakteristik:} eTilbudsavis er konsistent i metoden hvorpå data leveres, information der gives omkring dataene kan dog være meget varierende, hvilket resultere i en del ubrugeligt data, der som følge skal sorteres fra.
Da eTilbudsavis altid sender dataene på samme måde, kan ubrugeligt data sorteres fra uden større problemer. 

\textbf{Eksempel:} eTilbudsavis gør 2.000 tilbud tilgængelig, heri er nogle på et format der ikke klargøre hvad der er på tilbud, men blot giver en række mærkenavne som er på tilbud.
En sådan uforståelig række bliver filtreret ud såvel som andre uforståelige tilbud, og systemet ender tilbage med 1.337 brugbare tilbud, der kan vises til brugeren.

\subsubsection*{Brugsmønstre}
For en yderligere beskrivelse af de funktionaliteter i systemet, som vedrører en given aktør, modelleres en række brugsmønstre lavet på baggrund af hændelserne repræsenteret på \myref{handelser}, dette er de samme brugsmønstre som ses i aktørtabellen \myref{aktortabel}.
Hver enkelt af disse mønstre vil blive beskrevet igennem en brugsmønstrespecifikation.
Det er ikke alle mønstre som ses på denne liste, nogle af disse er sammentrækninger af flere andre mønstre, som alene virker simple og repetitive at beskrive.
Andre er udeladt da de ikke passer helt ind i en sammentrækning, men variationen fra det mønster og andre brugsmønstre ellers beskrevet, er så minimal, at mønstret er anset som værende ubetydeligt at beskrive.

\subsubsection*{Brugeridentifikation}
\textbf{Brugsmønster:} Brugeridentifikation sker ved at en \textbf{bruger} logger ind i systemet. 
Brugeren vil blive præsenteret for en side hvor e-mail og password kan indtastes.
Herefter godkender eller afviser systemet den opgivne data og håndterer resultatet, enten ved at logge brugeren ind, eller bede om informationen igen.
Alternativt kan brugeren oprette en ny konto i systemet, brugeren vil blive bedt om samme information som login, men håndterer i stedet informationen ved at oprette en bruger med dataene, der er blevet opgivet.
Når en bruger er logget vil den kunne tilgå bruger-specifik information såsom præferencer og indkøbslister såvel som ændre på disse.

\textbf{Objekter:} Person.

\textbf{Funktioner:} Register bruger, Log ind, Log ud.

\subsubsection*{Listehåndtering}
\textbf{Brugsmønster:} Dette brugsmønster dækker over indkøbslisten såvel som overvågningslisten. 
Brugeren kan inden for listens brugsmønstre tilgå alle funktionaliteter i vilkårlig rækkefølge, givet der er oprettet en liste på forhånd. 
En bruger kan altid oprette en indkøbsliste, ved at oprette en tom liste.
Ligeledes kan en bruger slette lister igen.
Overvågningslisten på den anden hånd, eksisterer altid og kan hverken oprettes eller slettes.
Brugeren kan på begge lister tilføje eller fjerne vare.
Varerne på overvågningslisten er varer som brugeren er interesserede i at få tilbud om. 
Når en varer på denne liste kommer på tilbud modtager brugeren en notifikation derom.
Ved indkøbslisten er der tre funktionaliteter til at tilføje ting til listen.
Man kan tilføje varer, tilbud, eller ingredienser fra en opskrift.
Ydermere kan en bruger aftjekke eller fjerne vare fra listen, såvel som dele deres liste med andre brugere.

\textbf{Objekter:} Indkøbsliste, Overvågningsliste, Varer, Tilbud, Personer, Opskrifter.

\textbf{Funktioner:} Opret liste, Fjern liste, Tilføj til liste, Fjern fra liste, Aftjek på liste, Del liste.

\subsubsection*{Søgning}
\textbf{Brugsmønster:} Dette brugsmønster igangsættes af brugeren.
Brugeren kan søge efter varer, hvorefter systemet filtrerer efter søgestrengen for at finde relevante resultater.
Dette brugsmønster benyttes flere steder, både til at finde vare og tilbud til at tilføje til lister, såvel som at finde opskrifter.


\textbf{Objekter:} Vare, Tilbud, Opskrifter.

\textbf{Funktioner:} Søg efter tilbud, Søg efter opskrifter.

\subsubsection*{Tilpas præferencer}
\textbf{Brugsmønster:} Brugsmønstret benyttes af en bruger.
I denne del af systemet har brugeren mulighed for at indstille sine præferencer vedrørende madtyper.
Brugeren kan blackliste forskellige typer af mad såsom fisk, resulterende i at opskrifter indeholdende fisk vil være sorteret fra under opskrifter, samt fjernelse af tilbud med fisk. 
Ligeledes er det muligt for brugeren at fjerne bestemte butikker, hvor der ikke ønskes at ses tilbud fra.

\textbf{Objekter:} Blacklist.

\textbf{Funktioner:} Sæt Præferencer, Filtrer efter præferencer.

\subsubsection*{Vurder opskrift}
\textbf{Brugsmønster:} 
Efter en opskrift er vurderet, kan andre brugere se den gennemsnitlige vurdering af en opskrift, og de brugere som har vurderet opskrifter, vil få anbefalet opskrifter som ligner.

\textbf{Objekter:} Opskrift, Vurderinger, Liste af vurderinger.

\textbf{Funktioner:} Vurder opskrift.

\subsubsection*{Se anbefalinger}
\textbf{Brugsmønster:} Brugeren kan se hvilke opskrifter, som er foreslået ud fra tidligere vurderede, såvel som afprøvede, opskrifter.
Ydermere kan brugeren få foreslået dagligdagsvarer, som ofte købes, baseret på tidligere indkøbsvaner.

\textbf{Objekter:} Opskrift, Vare.

\textbf{Funktioner:} Send anbefaling.

\subsubsection*{Hent tilbud}
\textbf{Brugsmønster:} Dette brugsmønster igangsættes af systemet, der efter et specifikt tidsinterval kalder \textbf{eTilbudsavis}.
eTilbudsavis henter som følge deraf, tilbud fra en API, og opdaterer systemets database med nye tilbud.

\textbf{Objekter:} Tilbud.

\textbf{Funktioner:} Hent tilbud.

\subsubsection*{Tilstanden af system}
\begin{figure}
	\centering
	\includegraphics[scale=0.6, angle=90]{images/Diagrams/Tilstandsdiagram.PNG}
	\caption{Tilstandsdiagram for systemet}
	\caption{Grønne bokse indikerer menupunkter, gule bokser indikere tilstande, den sorte centerlinje indikere menutilgang, pile indikere hændelsessekvenser, nærmere beskrivelese ses på \myref{tdiabeskriv}}\label{tilstandsdiagram}
\end{figure}

Tilstandsdiagrammet, \myref{tilstandsdiagram}, viser de forskellige stadier, systemet kan være i samt hvilke funktionaliteter, der er tilgængelige fra disse stadier.
Diagrammet hjælper med at danne overblik over navigeringen i systemet, og hvilken proces der opstår under udførelse af diverse brugsmønstre.
Markeret med grønt, ses de fem hovedmenuer.
Efter en bruger er logget ind, kan disse hovedmenuer til enhver tid tilgås.
Menuen er repræsenteret gennem den længere sorte centerlinje.
Alle pile, der går ind her, giver adgang til alle de pile, der går ud.
Pilene repræsenterer en hændelse igangsat af brugeren.
En klynge kan til hver en tid, uanset tilstand, tilgå hændelser, der går ud af klyngens boks.
Herfra kan man således se, at menuen altid er tilgængelig, såvel som logud funktionen.

Indikeret med pile kan man se, hvilke funktionaliteter der er tilgængelig fra en given tilstand, samt om disse er sekventielle eller iterative hændelser.
En sekventiel funktion skifter tilstanden på systemet, mens en iterativ blot opdaterer systemets information, mens den forbliver i tilstanden.
Eksempelvis har den grønne menu-boks \textit{Præferencer} en iterativ funktion på sig, da denne blot opdaterer information i systemet frem for at skifte tilstand i modsætning til funktionen \textit{Vælg opskrift}, som er sekventiel.
Hvis brugeren fra menuen \textit{Se opskrifter} giver brugerinputtet \textit{Vælg opskrift}, følges sekvensen til en ny tilstand \textit{Opskrift valgt}. 
Herfra åbnes således op for nye funktionaliteter for brugeren.
Hele diagrammet følger disse to typer af hændelser, som er gældende for alle tilstande og funktioner.


\subsection{Funktioner}\label{subsec:funktioner}

I dette afsnit beskrives funktionerne der skal bruges for at kunne håndtere hændelserne fra problemområdet, og brugsmønstrene beskrevet ovenfor.
Der findes fire typer funktioner: Aflæsnings-, opdaterings-, beregnings-, og signaleringsfunktioner.\citep{OOA&D2001}
Først identificeres funktionerne, hvorefter der gives en kategori til funktionerne, og en bedømmelse af deres kompleksitet. Herefter gives en kortfattet beskrivelse af funktionen hvor dette er tilstrækkeligt, ellers gives en dybere beskrivelse.

På figur \ref{tabel:functionstable} ses en tabel over de forskellige funktioner samt deres kompleksitet og kategori.
Kompleksitet beskriver hvor kompleks en given funktion er, dette fastsættes på baggrund af om en given funktion har mere end en kategori, som også kan ses i tabellen, samt en vurdering af hvor kompleks funktionens arbejde er.

\begin{table}[H]
  \centering
    \colorlet{shadecolor}{gray!40}
    \rowcolors{1}{white}{shadecolor}
      \begin{tabular}{l|lllll}
      %\hline
      \textbf{Funktioner}			& {Kompleksitet}	& {Kategori}  	\\ \hline
      Log ind						& Medium			& Beregning, Opdatering		\\
      Log ud						& Simpel			& Opdatering	\\
      Registrer bruger				& Simpel			& Opdatering	\\
      Tilføje varer til lister		& Simpel       		& Opdatering	\\
      Fjerne varer fra lister		& Simpel       		& Opdatering	\\
      Oprette og slette lister		& Simpel       		& Opdatering	\\
      Dele lister					& Medium       		& Opdatering	\\
      Søgning på tilbud for varer   & Medium     		& Beregning		\\
      Sætte præferencer				& Simpel       		& Opdatering	\\
      Filtrere for præferencer		& Kompleks     		& Beregning		\\
      Give vurdering				& Simpel       		& Opdatering	\\
      Sende anbefaling				& Meget kompleks	& Aflæsning, signalering, beregning		\\
      Meddele tilbud på varer		& Medium      		& Signalering	\\
	    Se opskrifter					& Simpel       		& Aflæsning		\\
      Oprette opskrift      & Simpel          & Opdatering  \\
      Ændre opskrift        & Simpel          & Opdatering \\
      Kopier opskrift       & Simpel          & Opdatering \\
	    Se tilbud						& Simpel       		& Aflæsning		\\
      Hente tilbud					& Simpel	       	& Opdatering	\\
    \end{tabular}
  \caption{Funktionstabel. Viser de forskellige funktioner der skal bruges, samt deres kompleksitet og kategori.}\label{tabel:functionstable}
\end{table}

\textbf{Log ind:} Her bliver sendt brugerinformationer, som verificeres med brugerne registreret i systemet.
Hvis dette fuldføres, opdateres modellaget således brugeren er registreret som værende logget ind.

\textbf{Log ud:} Brugeren der før var logget ind, opdateres til at være logget ud.

\textbf{Registrer bruger:} En ny bruger sender sine brugerinformationer, og et nyt login oprettes i modellaget.

\textbf{Tilføje varer til lister:} Denne funktion skal tilføje et objekt af vare klassen til en indkøbsliste, eller en overvågningsliste.

\textbf{Fjerne varer fra lister:} Funktionen her skal så fjerne disse objekter igen.

\textbf{Oprette og slette lister:} Denne funktion bruges når en indkøbsliste skal oprettes således man kan tilføje varer til denne.

\textbf{Dele lister:} Skal opdatere modellaget således en anden bruger kan tilgå samme indkøbsliste som brugeren der deler sin liste.

\textbf{Søgning på tilbud for varer:} Søger efter tilbud som passer til varen der søges for.

\textbf{Sætte præferencer:} Funktionen skal sætte forskellige præferencer som vælges af brugeren, således brugerens oplevelse rettes efter brugerens præferencer.

\textbf{Filtrere for præferencer:} Funktionen skal findes i to udgaver. Der skal være filtrering i forhold til præferencer for både tilbud, og opskrifter.

\textbf{Give vurdering:} Funktionen skal modtage en vurdering fra brugeren og gemme denne i modellaget.

\textbf{Sende anbefaling:} Funktionen her er meget kompleks da der indgår mange forskellige typer handlinger.
Brugerne har vurderet forskellige opskrifter, og der beregnes en anbefaling ud fra disse.
Når anbefalingen er lavet skal den gives eller signaleres til brugeren.

\textbf{Meddele tilbud på varer:} Når en varer på overvågningslisten kommer på tilbud sender denne funktion et signal til brugeren derom.

\textbf{Se opskrifter:} Funktionen skal hente opskrifterne fra modellaget.

\textbf{Se tilbud:} Funktionen henter tilbudene fra modellaget.

\textbf{Hente tilbud:} Denne funktion henter tilbud, vha. API'et fra eTilbudsavis, disse skal derefter gemmes i modellaget.

Denne analyse af funktionerne hjælper med at danne et overblik over hvilke funktionaliteter der skal designes, samt hjælper det til at stille krav til det systemet.
I det følgende afsnit stilles der krav til systemet.
