\section{Problemformulering}\label{section:problemformulering}

I state of the art undersøgelsen \myref{s:SOTA} blev der undersøgt eksisterende løsninger på problemet fremstillet i indledningen. 
Der blev ikke fundet nogle løsninger som løste alle de problemstillinger fundet i den første interviewrunde, det blev desuden også klart at var ingen af de adspurgte fra første interviewrunde i \myref{section:interview1} brugte disse programmer.
Grunden til at disse ikke er blevet brugt er ikke nærmere undersøgt.
Den viste samtidigt at de interviewede havde en interesse for en software løsning på dette problem, hvis den havde  features de andre programmer ikke havde. 

I prototype interviewene blev vores powerpoint prototype fremvist og testet, her blev flere funktionaliteter som kunne inkluderes i systemet udforsket .
Disse informationer har ledt til følgende problemstilling:

\textbf{Hvordan designes og implementeres et system, i C\#, der kan give brugere lettere adgang til dagligvarebutikkernes tilbud integreret med indkøbslister og opskrifter imens det gøres mere overskueligt, at handle ind til aftensmad? }
