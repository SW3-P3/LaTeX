\section{Problemformulering}\label{section:problemformulering}

%State of the art undersøgelserne viste at der findes mange applikationer og løsninger, til at hjælpe med den daglige madlavning og indkøb til hjemmet.
%Det viste sig desuden at selvom der var mange løsninger, var disse ikke alle optimale, nogle grundet manglende funktionalitet, andre grundet dårlig udførelse af programmet.
%Vores første interviewrunde viste at der ikke blev benyttet mange hjælpemidler i forvejen, men at der dog var en interesse for emnet.
%Dette kan bl.a. skyldes at de allerede eksisterende løsninger, ikke har interesse for personerne, dårlig kvalitet af programmer, eller måske bare dårlig og/eller manglede markedsføring. \fxnote{Behøver vi sige hvad det skyldes, vi gætter jo bare? - søren}
%
%Anden interviewrunde viste hvilke funktionaliteter en løsning skulle have.
%Med disse informationer arbejdes der videre med følgende problemstilling:
%
%
%\textbf{Hvordan designes og implementeres et system, i C\# .NET, der kan give brugere lettere adgang til dagligvarebutikkernes tilbud integreret med indkøbslister og opskrifter imens det gøres mere overskueligt, at handle ind til aftensmad? }


I state of the art undersøgelsen \myref{s:SOTA} blev der undersøgt eksisterende løsninger på problemet fremstillet i indledningen. 
Der blev ikke fundet nogle løsninger som løste alle de problemstillinger fundet i den første interviewrunde, her blev det også klart at der var ingen som brugte disse programmer.
Grunden til at disse ikke er blevet brugt er at de ikke er markedsført tilstrækkeligt, dårlig kvalitet eller manglende interesse for de eksisterende. 
Den viste samtidigt at de interviewede havde en interesse for en software løsning på dette problem, hvis den havde yderligere features og var af højere kvalitet. 

I prototype interviewene blev vores powerpoint prototype fremvist og testet, her blev flere funktionaliteter udforsket. 
Alt dette har ledt til følgende problemstilling:

\textbf{Hvordan designes og implementeres et system, i C\#, der kan give brugere lettere adgang til dagligvarebutikkernes tilbud integreret med indkøbslister og opskrifter imens det gøres mere overskueligt, at handle ind til aftensmad? }
