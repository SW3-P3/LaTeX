\section{eTilbudsavis API}\label{api}
I dette afsnit beskrives først teori om hvad et API er, og hvordan sådan et er struktureret.
Herefter gennemgås design og implementering for brug af eTilbudsavis' API, samt motivationen for at bruge API'en. Til sidst vil der blive konkluderet på API'ens betydning for projektet

For at kunne indhente infomation om butikker og deres tilbud, anvendes der et API.
Dette API udbydes af den danske virksomhed eTilbudsavis.dk, og er af typen REST.

\subsection{Teori}
REST er en abstraktion for arkitekturen af internettet, den simplificerer den måde som applikationer kan kommunikere med hinanden.
Dette bruges ofte til API'er for at skabe en simpel og struktureret måde at kommunikere med en service\citep{REST}.

\subsubsection{REST}
REST står for: \textit{Representational state transfer}.
REST er er måden man kommunikerer med API'et.
Der findes fire metoder som svarer til de fire som findes i CRUD (``CREATE, READ, UPDATE, DELETE'').
Disse metoder er: GET, POST, PUT og DELETE.

\textbf{GET}
returnerer data, typisk som JSON eller XML og med HTTP svarkoden 200 (OK).
eTilbudsavis' API sender JSON.
Svarer til READ i CRUD.

\textbf{POST}
bruges til at oprette nye ressourcer.
Svarer til CREATE i CRUD.

\textbf{PUT}
bruges til at opdatere infomation, eksempelvis til at forlænge en session.
Svarer til UPDATE i CRUD.

\textbf{DELETE}
bruges til at fjerne ressourcer.
Svarer til DELETE i CRUD.

\subsection{Design}
eTilbudsavis' API er kun et halv-offentligt API, dvs. at man skal have en API-nøgle (``APIKEY'') og dertilhørende hemmelighed (``Secret'').
API'en er placeret på \textit{https://api.etilbudsavis.dk/v2/}. \citep{eTilAPI}

For at kommunikere med API'en fra C\#-kode anvendes værktøjet RestSharp. RestSharp er en simpel REST og HTTP API klient for et .NET miljø. \citep{RestSharp}
For at kunne bruge informationen laves der en klasse, som svarer til det JSON som API'en sender tilbage.

\subsection{Motivation}\label{api:skoddata}
Vi har valgt at benytte et API, især eTilbudsavis' API, da det giver os muligheden for at arbejde med og vise faktiske tilbud.
Der er dog visse ulemper ved, at vi benytter os af eTilbudsavis' API. 
Det data vi modtager er ikke, hvad vi ville betegne som ``optimalt''. 
Med dette menes der, at dataet for eksempel ikke indeholder; informationer om grupperede tilbud, kategori-inddeling eller forekomster af allergener i varerne. 
Dette er funktioner vores testpersoner har adspurgt. 
Hvis disse tre attributter havde været tilstede, ville systemet for eksempel være i stand til at frasortere tilbud, der indeholder spor af jordnødder eller lignende, hvis den pågældende bruger eksempelvis er allergisk overfor jordnødder. 
Ligeledes ville det med kategori-inddeling være muligt at sortere mere intuitivt i tilbud, og information om grupperede tilbud ville give systemet mulighed for at præsentere visse tilbud bedre for brugeren, så der ikke opstår forvirring.

\subsection{Implementering}
Denne implementering er en del af controlleren: ``OfferController.cs'' placeret i ``ProjectFood/Controllers'', i metoden ``ImportOffers''.
\subsubsection{Session}
For at kunne bruge API'et skal man først oprette en session.
En session består af en API-nøgle, den tilsvarende hemmelighed (``Secret''), en token og en signatur.
Har man en API-nøgle kan man kalde API'et sessions og få en token.
Den token kan man kombinere med sin hemmelighed og generere SHA-256 hashen af dette som er signaturen, der brugers til at underskrive requests.

Eksempelvis vil apikaldet som opretter en session returnere:
\begin{lstlisting}[language=json,firstnumber=1,caption="POST til sessions api'en med APIKEYen",label=apilst1]
{
    "token": "00hdcx7fnysn6541",
    "expires": "2013-03-03T13:37:00+0000",
    "user": null,
    "provider": null,
    "permissions": {
        "guest": [
            "api.public",
            "api.users.create"
        ]
    }
}
\end{lstlisting}

Dette svar kan oversættes til en klasse i C\#, som har tilsvarende felter, så den kan indlæses med RestSharp som vist i \myref{lst:session}.

%[firstnumber=1,caption="Det foerste API kald"]
\begin{lstlisting}[caption="C\#-kode som opretter en RestClient og anvender den til at oprette et objekt med felter som svarer til JSON dataet givet fra API'en", label=lst:session]
/* [...] */
/* Create a RestClient*/
var client = new RestClient("https://api.etilbudsavis.dk");

/* Initiate the first request to get a session */
var SessionRequest = new RestRequest("v2/sessions", Method.POST);
SessionRequest.AddParameter("api_key", Global.Apikey);

/* Map the response to the class "Session" */
IRestResponse<Session> response2 = client.Execute<Session>(SessionRequest);

/* Save the response in an object */
Session sessionobj = response2.Data;

/* Generate a signature and set it to the global variable */
Global.Signature = EncryptionHelper.SHA256(Global.Secret + Global.SessionToken);
/* [...] */
\end{lstlisting}
Efter eksekveringen af denne kode, vil der være et objekt af datatypen ``Session'' kaldet sessionobj, som indeholder de samme felter som givet i JSON i \myref{apilst1}; derudover er signaturen, en SHA256 hash af hemmeligheden og sessiontokenen, (``Global.Signature'') også beregnet og gemt.

\subsubsection{Indlæsning af tilbud}
API'ets anvendelse i programmet er at indhente aktuelle tilbud.
For at kunne modtage tilbud skal der konstrueres et ``request'', som minder om det, vist i \myref{lst:session}, men anmoder tilbud.
Der er en øvre grænse sat af eTilbudsavis.dk på at maksimalt 100 tilbud kan hentes ad gangen, da der vil skulle hentes mere end 100 tilbud ad gangen, er det nødvendigt at fortage flere anmodninger.
Derfor bruges en do..while løkke som gentages så længe der returneres 100 tilbud fra API'et, denne er vist i \myref{apiofferscs}.

\begin{lstlisting}[caption="C\#-kode som anvender ``/v2/offers'' delen af API'et til at hente tilbud.", label=apiofferscs]
/* [...] */
var listofApiOffers = new List<ApiOffer>();
do {
    var nextOffersRequest = new RestRequest("v2/offers", Method.GET);
    nextOffersRequest.AddParameter("r_lat", Global.Latitude);
    nextOffersRequest.AddParameter("r_lng", Global.Longitude);
    nextOffersRequest.AddParameter("r_radius", Global.Radius);
    nextOffersRequest.AddHeader("X-Token", Global.Session.Token);
    nextOffersRequest.AddHeader("X-Signature", Global.Session.Signature);
    nextOffersRequest.AddParameter("limit", "100");
    nextOffersRequest.AddParameter("dealer_ids", "9ba51,8c4da,bdf5A,11deC,c1edq,71c90,101cD,0b1e8,ecddz,8c4da,1e1eB");
    nextOffersRequest.AddParameter("offset", listofApiOffers.Count);
    offersResult = client.Execute<List<ApiOffer>>(nextOffersRequest).Data;
    listofApiOffers.AddRange(offersResult);
} while (offersResult.Count == 100)
/* [...] */
\end{lstlisting}

Denne kode skal oversættes til vores model af et tilbud, dette gøres i en foreach løkke, som vist i \myref{apiofferstooffercs}.

\begin{lstlisting}[caption="C\#-kode som bruger API-data til at oprette instancer af Offer-klassen\, og tilføjer dem til databasen", label=apiofferstooffercs]
/* [...] */
foreach (var o in listofApiOffers)
{
    var newOffer = new Offer
    {
        eTilbudsavisID = o.id,
        Heading = o.heading,
        Begin = o.run_from,
        End = o.run_till,
        Store = o.branding.name,
        Price = o.pricing.price,
        Unit = o.quantity.unit != null ? o.quantity.size.@from + " " + o.quantity.unit.symbol : " "
    };

    // If the offer doesn't already exist in the database add it.
    if (!_db.Offers.Any(x => x.eTilbudsavisID == newOffer.eTilbudsavisID))
    {
        _db.Offers.Add(newOffer);
    }
}
/* [...] */
\end{lstlisting}

Foruden at indhente data fra eTilbudsavis' API, er det også muligt at importere tilbud direkte fra JSON-filer (på samme format som eTilbudsavis'), dette sker via metoden ``ImportOffersFromFiles'' fra controlleren ``AdminPanelController.cs''. \fxnote{Indsæt infomation om adminpanellet i generel info.}
Denne metode henter filer fra stien ``ProjectFood\/App\_Data\/jsonData\/'', og oversætter dem til instanser af klassen Offer og tilføjer dem til databasen. 
Grunden til at begge metoder findes er, at denne gør det muligt at indhente data hurtigere og over en længere periode, da den ikke begrænses af: API'ets udløbstid (det er kun mulig at hente nuværende og fremtidige tilbud), API'ens hastighed og evt. offline brug. 
For at kalde denne metode skal man tilgå administrator kontrolpanelet, som findes på stien ``\/AdminPanel''.

\subsection{Konklusion}
Ved brug af frameworket RestSharp er det muligt at indhente data omkring tilbud fra eTilbudsavis' API. 
Dette gør programmet i stand til at opføre sig mere realistisk, da alle tilbud vist i det vil være faktiske tilbud i butikker.
Der er også mulighed for at importere tilbud fra lokale JSON-filer, så der lettere kan testes på overvågning og forekomst af nye tilbud. 
Dog er eTilbudsavis' API data ikke detaljeret nok til at implementere visse funktionaliteter, som ville kunne udvide systemet i en positiv retning.
