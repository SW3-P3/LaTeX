\subsection{MoSCoW}
MoSCoW-analysen benyttes for at vurdere forskellige funktioner og krav imod hinanden.

De opdeles i fire katagorier: Must have, Should have, Could have og Won't have.
Deraf kommer navnet MoSCoW.

Ved hjælp af MoSCoW'en kan vi arragerer disse krav i kategorier alt efter hvor vigtige de er.
Baggrunden for denne ragering er henholdvis baseret på den feedback interviewsne i \myref{section:interview2} gav og en vudering fra gruppens side.
Vores vudering er foretaget på baggrund af estimater om hvor tidskrævende hver funktion vil være at implementerer holdt op imod hvor vigtig denne funktion virker for vores respondenter, samt om funktionens relavans i forhold til projektets mål.


\noindent\parbox[t]{2.7in}{\raggedright
	\textbf{Must have}
	\begin{itemize}
			\item Indkøbslister
			\item Tilbudsintegration
			\item Varer
			\item Overvågning af tilbud
			\item Opskrifter
			\item Desktop interface
	\end{itemize}
}
\parbox[t]{2.7in}{\raggedright%
	\textbf{Should have}
	\begin{itemize}
		\item Deling af inkøbslister
		\item Notifikationer
		\item Vurdering af opskrifter
		\item Anbefalinger
		\item Mobil interface
		\item Tilbud -> Varer
	\end{itemize}
}

\noindent\parbox[t]{2.7in}{\raggedright
	\textbf{Could have}
	\begin{itemize}
			\item E-mail notifikationer
			\item Brugerlavede opskrifter
			\item Navigation af butik
	\end{itemize}
}
\parbox[t]{2.7in}{\raggedright%
	\textbf{Won't have}
	\begin{itemize}
		\item Kalorietæller
		\item Stregkode integration
		\item Overvågning af køleskab
	\end{itemize}
}
