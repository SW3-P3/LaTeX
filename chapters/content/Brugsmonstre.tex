\subsubsection*{Brugeridentifikation}
\textbf{Brugsmønster:} Brugeridentifikation sker ved at en \textbf{bruger} logger ind i systemet.
Brugeren vil blive præsenteret for en side hvor, e-mail og password kan indtastes.
Herefter godkender eller afviser systemet den opgivne data og håndterer resultatet, enten ved at logge brugeren ind, eller bede om informationen igen.
Alternativt kan brugeren oprette en ny konto i systemet, brugeren vil blive bedt om samme information som login, dog vil der også spørges efter et navn, men håndterer i stedet informationen ved at oprette en bruger med dataene, der er opgivet.
Når en bruger er logget vil den kunne tilgå bruger-specifik information såsom præferencer og indkøbslister såvel som ændre på disse.

\textbf{Objekter:} Person.

\textbf{Funktioner:} Registrer bruger, Log ind, Log ud.

\subsubsection*{Listehåndtering}
\textbf{Brugsmønster:} Dette brugsmønster dækker over indkøbslisten såvel som overvågningslisten.
Brugeren kan inden for listens brugsmønstre tilgå alle funktionaliteter i vilkårlig rækkefølge, givet der er oprettet en liste på forhånd.
En bruger kan altid oprette en indkøbsliste, ved at oprette en tom liste.
Ligeledes kan en bruger slette lister igen.
Overvågningslisten på den anden hånd, eksisterer altid og kan hverken oprettes eller slettes.
Brugeren kan på begge lister tilføje eller fjerne en vare.
Varerne på overvågningslisten er varer som brugeren er interesserede i at få tilbud om.
Når en vare på denne liste kommer på tilbud modtager brugeren en notifikation derom.
Ved indkøbslisten er der tre funktionaliteter til at tilføje ting til listen.
Man kan tilføje varer, tilbud, eller ingredienser fra en opskrift.
Ydermere kan en bruger aftjekke eller fjerne vare fra listen, såvel som dele deres liste med andre brugere.

\textbf{Objekter:} Indkøbsliste, Overvågningsliste, Varer, Tilbud, Personer, Opskrifter.

\textbf{Funktioner:} Opret liste, Fjern liste, Tilføj til liste, Fjern fra liste, Aftjek på liste, Del liste.

\subsubsection*{Søgning}
\textbf{Brugsmønster:} Dette brugsmønster igangsættes af brugeren.
Brugeren kan søge efter varer, hvorefter systemet filtrerer efter søgestrengen for at finde relevante resultater.
Dette brugsmønster benyttes flere steder, både til at finde vare og tilbud til at tilføje til lister, såvel som at finde opskrifter.


\textbf{Objekter:} Vare, Tilbud, Opskrifter.

\textbf{Funktioner:} Søg efter tilbud, Søg efter opskrifter.

\subsubsection*{Tilpas præferencer}
\textbf{Brugsmønster:} Brugsmønstret benyttes af en bruger.
I denne del af systemet har brugeren mulighed for at indstille sine præferencer vedrørende madtyper.
Brugeren kan blackliste forskellige typer af mad såsom fisk, resulterende i at opskrifter indeholdende fisk vil være sorteret fra under opskrifter, samt fjernelse af tilbud med fisk.
Ligeledes er det muligt for brugeren at fjerne bestemte butikker, hvor der ikke ønskes at ses tilbud fra.

\textbf{Objekter:} Blacklist.

\textbf{Funktioner:} Sæt Præferencer, Filtrer efter præferencer.

\subsubsection*{Opskrift håndtering}
\textbf{Brugsmønster:}
Brugeren kan interagere med opskrifter på forskellig vis. 
Som bruger har man mulighed for at oprette opskrifter, kopiere opskrifter, ændre opskrifter og vurdere opskrifter.
Når en opskrift oprettes vil brugeren, udover at kunne tilføje almindelige ting, også tilføje instruktioner, tid og ingredienser.
En bruger vil have mulighed for at ændre i sine egne opskrifter, og ligeledes kunne kopiere andres opskrifter og derefter tilføje ændringer i disse.
Fra indgredienslisten på en opskrift kan brugerne tilføje en eller flere varer til deres indkøbslister.
Brugerne har også mulighed for at vurdere opskrifter, ud fra brugerens vurderinger, vil systemet anbefale nye opskrifter til brugeren.

\textbf{Objekter:} Opskrift, Vurdering, liste af vurderinger, varer, indkøbsliste.

\textbf{Funktioner:} Se opskrift, Oprette opskrift, ændre opskrift, kopier opskrift, vurder opskrift, tilføje vare til liste.

%\subsubsection*{Vurder}
%\textbf{Brugsmønster:}
%Efter en opskrift er vurderet, kan andre brugere se den gennemsnitlige vurdering af en opskrift, og de brugere som har vurderet opskrifter, vil få anbefalet opskrifter som ligner.
%
%\textbf{Objekter:} Opskrift, Vurderinger, Liste af vurderinger.
%
%\textbf{Funktioner:} Vurder opskrift.

\subsubsection*{Se anbefalinger}
\textbf{Brugsmønster:} Brugeren kan se hvilke opskrifter, som er foreslået ud fra tidligere vurderede opskrifter.

\textbf{Objekter:} Opskrift, Vare.

\textbf{Funktioner:} Send anbefaling.

\subsubsection*{Hent tilbud}
\textbf{Brugsmønster:} Dette brugsmønster igangsættes af systemet, der efter et specifikt tidsinterval kalder \textbf{eTilbudsavis}.\fxnote{Vi har jo ikke implementeret det således, hvordan skal vi formulere det?  evt. Hvis hjemmesiden blev deployed til en server ville det køre på et specifikt tidsinterval, men i dette projekt gøres det manuelt ved at trykke på en knap. Mere om det i myref whatever? - Søren}
eTilbudsavis henter som følge deraf, tilbud fra en API,\fxnote{Det er vel programmet som henter vha. etilbudsavis, og ikke dem som henter vha, API?, jeg ved godt det er fordi de er aktøren, det er bare lidt forvirrende - Søren} og opdaterer systemets database med nye tilbud.

\textbf{Objekter:} Tilbud.

\textbf{Funktioner:} Hent tilbud.
