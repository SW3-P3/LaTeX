\subsubsection*{Brugeridentifikation}
\textbf{Brugsmønster:} Brugeridentifikation sker ved at en \textit{bruger} logger ind i systemet. 
\textit{Brugeren} vil blive præsenteret for en side hvor email og password kan indtastes.
Herefter godkender eller afviser systemet den opgivne data og håndterer resultaten, enten ved at logge \textit{brugeren} ind, eller bede om informationen igen.
Alternativt kan \textit{brugeren} oprette en ny konto i systemet, \textit{brugeren} vil blive bedt om samme information som login, men håndterer istedet informationen ved at oprette en bruger med dataene, der er blevet opgivet.
En indlogget \textit{bruger} vil kunne tilgå bruger-specifik information såsom præferencer og indkøbslister såvel som ændre på disse.

\textbf{Objekter:} User.

\textbf{Funktioner:} RegisterUser, Login, Logout.

\subsubsection*{Listehåndtering}
\textbf{Brugsmønster:} Dette brugsmønster dækker over indkøbslisten såvel som overvågningslisten. 
\textit{Brugeren} kan inden for listens brugsmønstre tilgå alle funktionaliteter i vilkårlig rækkefølge, givet der er oprettet en liste på forhånd. 
En \textit{bruger} kan altid oprette en indkøbsliste, dette kan gøres på to måder, ved at tilføje en vare til en liste mens ingen eksisterer og herved oprette en ny, eller ved at oprette en tom liste ved brug af den dertil designet knap.
Ligeledes kan en \textit{bruger} slette lister igen med en dertil designet knap.
Overvågningslisten på den anden hånd, eksisterer altid og kan hverken oprettes eller slettes.
\textit{Brugeren} kan fra overvågningslisten tilføje eller fjerne vare, hvorom der er en interesse for overvågning af tilbud, når et tilbud på sådan en vare så bliver oprettet, får \textit{brugeren} en notifikation derom.
Ved inkøbslisten er der tre funktionaliteter til at tilføje ting til listen.
En \textit{bruger} kan enten tilføje en generisk vare og se hvilke tilbud der er på disse vare, hvorefter der kan yderligere specifiseres en bestemt vare, tilføje et tilbud fra listen over tilbud, eller tilføje de vare som indgår i en opskrift.
Ydermer kan en \textit{bruger} aftjekke eller fjerne vare fra listen, såvel som dele deres liste med andre \textit{brugere}.

\textbf{Objekter:} List, GenericWare, Offer, ListGroup.

\textbf{Funktioner:} CreateList, DeleteList, AddToList, RemoveFromList, CheckFromList, ShareList, BoughtWares.

\subsubsection*{Søgning}
\textbf{Brugsmønster:} Dette brugsmønster igangsættes af \textit{brugeren}.
\textit{Brugeren} bliver præsenteret for et tomt søgefelt, heri filtreres systemet efter søgestrengen for at finde relevante resultater.
Dette brugsmønster benyttes flere steder, både til at finde vare og tilbud til at tilføje til lister, såvel som at finde opskrifter.


\textbf{Objekter:} Ware, Offer, .

\textbf{Funktioner:} Search, TypeAhead.

\subsubsection*{Tilpas præferencer}
\textbf{Brugsmønster:} Brugsmønstret benyttes af en \textit{bruger}.
I denne del af systemet har \textit{brugeren} mulighed for at indstille sine præferencer vedrørende madtyper.
\textit{Brugeren} kan blackliste forskellige typer af mad såsom fisk, resulterende i at opskrifter indeholdende fisk vil være sorteret fra under opskrifter, ligeledes er det muligt for \textit{brugeren} at fjerne bestemte butikker, hvor der ikke ønskes at ses tilbud fra.

\textbf{Objekter:} Blacklist.

\textbf{Funktioner:} PreferenceFilterStore, PreferenceFilterRecipe.

\subsubsection*{Vurder opskrift}
\textbf{Brugsmønster:} Dette brugsmønster kan først igangsættes af \textit{brugeren}, efter at den givne opskrift er blevet markeret som prøvet.
Dette er for såvidt som muligt at undgå fejlagtig vurderinger fra diverse \textit{brugere}.
Efter en opskrift er vurderet, kan andre \textit{brugere} se den gennemsnitlige vurdering af en opskrift, og de \textit{brugere} som har vurderet opskrifter, vil få anbefalet opskrifter som ligner.

\textbf{Objekter:} RecipeTried, Recipe.

\textbf{Funktioner:} RecipeSuggest, RecipeRating.

\subsubsection*{Se anbefalinger}
\textbf{Brugsmønster:} \textit{Brugeren} kan se hvilke opskrfiter, som er foreslåget ud fra tidligere vurderede, såvel som afprøvede, opksrifter.
Ydermer kan \textit{brugeren} få foreslåget dagligdagsvare, som ofte købes, baseret på tidligere indkøbsvaner.

\textbf{Objekter:} Recipe, Ware.

\textbf{Funktioner:} RecipeSuggest, RecipeRating, BoughtWares.

\subsubsection*{Hent tilbud}
\textbf{Brugsmønster:} Dette brugsmønster igangsættes af systemet, der efter et specifikt tidsinterval kalder \textit{eTilbudsavisen}.
\textit{eTilbudsavisen} henter som følge deraf, tilbud fra en API, og opdaterer systemets database med nye tilbud.

\textbf{Objekter:} Offer.

\textbf{Funktioner:} GetOffers.