\section{Indledende interview}\label{section:interview1}
I dette ansnit præsenteres resultaterne af nogle interviews, som blev foretaget som start på projektet.
Formålet med disse interviews er at bekræfte, at en datologisk løsning kan afhjælpe eller minimere problemet, som vi her analyserer.
Interviewene har også til formål at lede projektet i den retning, hvor mulige slutbrugere føler der er størst behov for en løsning.


Onsdag d. 17 september gennemførte vi en række semi-strukturerede interviews i Aalborg midtby blandt folk omkring Føtex.
Vi adspurgte syv personer i forskellige målgrupper, herunder unge der bor alene, unge der bor sammen med nogen, og  folk over 30 med begge førnævnte sociale statusser.
Aldersspændet var fra 20 - 51, og kønnene var ligeligt fordelt.
Formålet med disse interviews var at danne et basalt overblik over vores problemområde, samt undersøge, hvordan folk håndterede vores initierende problem på nuværende tidspunkt.
Vores interviews bestod af en håndfuld forberedte spørgsmål, samt løsere samtale, for at følge op på disse spørgsmål.
Ud af interviewene fandt vi følgende hovedpunkter:
\begin{itemize}
	\item Generelt er folk uorganiserede når det kommer til indkøb, og især unge mennesker bruger sjældent indkøbslister.
	\item Inspiration til aftensmåltider findes ofte i indkøbssituationen eller ud fra princippet “hvad man lige har lyst til”.
	\item Det er ikke mange, der gennemgår tilbudsaviser, og endnu færre går bevidst efter tilbud.
	Dog kan det påvirke beslutninger om aftensmåltid, hvis en given varer er på tilbud i butikken, hvor der handles ind.
	\item Især den yngre målgruppe (18-28) er interesseret i en løsning med tilbud, indkøbslister og opskrifter til inspiration kædet sammen. Mens den ældre målgruppe (28+) virker mere interesseret i at kunne dele indkøbslister med familie, for bedre at kunne organisere indkøb.
	\item Opskrifter er som sagt noget, der mest interesserer det yngre segment.
	Dog påpeger nogle, at stor tendens til at være kræsen kræver at man kan sætte præferencer.
\end{itemize}

Disse resultater ligger til grundlag for systemdefinitionen og analysen af systemets problemområde \myref(chapter:systemanalyse).

Fremadrettet vil der arbejdes videre med følgende områder:

\begin{itemize}
	\item Indkøbslister
	\item Tilbud
	\item Opskrifter.
\end{itemize}
