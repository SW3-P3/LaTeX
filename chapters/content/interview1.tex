\section{Indledende interview}\label{section:interview1}
I dette afsnit præsenteres resultaterne af den første interviewrunde, som blev foretaget som start på projektet.
Formålet med denne interviewrunde er at bekræfte, at en datologisk løsning kan afhjælpe eller minimere problemet, som vi her analyserer.
Interviewene har også til formål at undersøge hvilke problemer personer har med madlavning og indkøb dertil.

Onsdag d. 17 september gennemførte vi en række semi-strukturerede interviews i Aalborg midtby udenfor Føtex.
Vi adspurgte syv personer i forskellige målgrupper, herunder unge der bor alene, unge der bor sammen med nogen, og folk over 30 med begge førnævnte sociale statusser.
Aldersspændet var fra 20 - 51, og kønnene var ligeligt fordelt.\fxnote{Så 3.5 mænd og 3.5 kvinder?}
Der vælges et så bredt aldersspænd, for at kunne danne en baggrund for hvor den primære målgruppe ligger, ud fra interesse.
Formålet med denne interviewrunde var at danne et basalt overblik over vores problemområde, samt undersøge, hvordan folk håndterede vores initierende problem på nuværende tidspunkt.
Vores interviews bestod af en håndfuld forberedte spørgsmål, samt løsere samtale, for at følge op på disse spørgsmål.
Spørgsmålene findes på \myref{Interview1}.
Ud fra denne interviewrunde er der gjort følgende observationer:
\begin{itemize}
	\item Generelt er folk uorganiserede når det kommer til indkøb, og især unge mennesker bruger sjældent indkøbslister.
	\item Inspiration til aftensmåltider findes ofte i indkøbssituationen eller ud fra princippet “hvad man lige har lyst til”.
	\item Det er ikke mange, der gennemgår tilbudsaviser, og endnu færre går bevidst efter tilbud.
	Dog kan det påvirke beslutninger om aftensmåltid, hvis en given varer er på tilbud i butikken, hvor der handles ind.
	\item Især den yngre målgruppe (18-28) er interesseret i en løsning med tilbud, indkøbslister og opskrifter. Den øgede interesse skyldes, ifølge de interviewede, at de allerede bruger teknologi\fxnote{Hvilken teknologi, da ikke til tilbud, men her er det smartphones og lign. ikke? overvej omformulgering hvis andre ser problemer her. - Troels} til disse formål, mens den ældre målgruppe (28+) udviser større interesse for at kunne dele indkøbslister med familie, for bedre at kunne organisere indkøb.
	\item Opskrifter er som sagt noget, der mest interesserer det yngre segment.
	Dog påpeger nogle, at stor tendens til at være kræsen kræver at man kan sætte præferencer.
\end{itemize}

Det viste sig altså at der er interesse for en løsning der kan hjælpe med inspiration til madlavningen. 
Yderligere viste det sig at den primære målgruppe består af unge som ikke bor hjemme (18-28), her var der stor interesse for flere af de foreslåede funktionaliteter, dette betyder dog ikke at en ældre målgruppe er irrelevant, da der her også blev udvist en interesse for enkelte funktionaliteter.
På baggrund af dette arbejdes der videre med unge som den primære målgruppe, og den lidt ældre som sekundær, det vil derfor give mening stadig at få noget feedback fra en ældre gruppe, og ikke udelukkende en yngre.\fxnote{Nævner vi at vi interviewde sassypantses far? Altså at han er lidt ældre?}
Tilbud havde en effekt på hvad man valgte at spise, og derfor kan tilbud være en central del af en sådan løsning.
Der var ikke mange af de interviewede der brugte løsninger som allerede var på markedet, om dette skyldes mangel på viden om dem, vides ikke.
Vi vil derfor undersøge hvilke løsninger der er på markedet i det næste afsnit som hjælper med følgende emner:

%Disse resultater ligger til grundlag for systemdefinitionen og analysen af systemets problemområde \myref{chapter:systemanalyse}.

\begin{itemize}
	\item Indkøbslister
	\item Tilbud
	\item Opskrifter.
\end{itemize}
