\section{Opskrifter}
I dette afsnit beskrives hvordan implenmentationen af opskrifter er foregået, og hvordan denne har opfyldt userstory 13, 15, 16, 17, og 18. Til sidst samles alt dette op i en konklusion. Opskrifter i dette projekt har som udgangspunkt rødder i user storie 13, som beskriver at brugeren ville have mulighed for at kunne finde opskrifter.
\subsection{Design}
Idéen ved opskrifter, er at give brugeren mulighed for nemt at finde opskrifter som har en pæn vurdering af andre brugere, hvorefter ingredienserne fra disse, nemt kan tilføjes til deres indkøbsliste(r).
Når man først går ind på opskrifterne, bliver man mødt af alle offentliggjorte opskritfer, som er blevet tilføjet af brugerne af systemet.
Opskrifterne er opstillet således der på forsiden for opskrifter kan ses den mest nødvendige information om hver enktle opskrift.
Opkrifterne er opstillet som cards for at give en klar opdeling og en gøre det overskueligt.
Opskrifterne indeholder alle sammen en titel, tid, antal personer, bedømmelse, ingredienser og en fremgangsmåde.


Hver opskrift er gemt under den bruger som har oprettet den, denne bruger har senere mulighed for at redigere eller slette opskriften efter valg.
Dette er gjort da user storie 18 ligger til grundlag for dette.
Er man ikke den bruger der har oprettet opskriften, har man mulighed for at kunne klone opskriften.
Når en opskrift bliver klonet, bliver alt data fra den tidligere opskrift kopieret og derefter kan man foretage ændringer til opskriften, gemme den under et nyt navn, således der kan personliggøres.


Brugerne har efterspurgt mulighed for at kunne tilføje ingridienser fra opskriften til deres indkøbs liste, i følge user storie 16.
Ingredienser på opskriften er opstillet således de nemt kan tilføjes til brugerens indkøbslister.
Det er muligt kun at vælge de ting man mangler, således man ikke er tvunget til at tilføje alle ingredienser fra hele opskriften.
Der kan ved hjælp af UI'et vælges hvilket indkøbsliste varene skal tilføjes til, og det kan ses hvilke ting der allerrede er blevet tilføjet fra opskriften, i form af små grønne flueben.


Der er på baggrund af user storie 17 gjort det muligt at der inde på den enkle opskrift kan vælges hvor mange personer man ønsker at opskriften skal skaleres til.
Der er taget udgangspunkt i fire personer, men der er mulighed for at både sænke og hæve denne værdi for at tilpasse mængden på ingredienserne, så ledes den passer til det valgte antal mennesker.


Hver opskrift har den mulighed, at en bruger kan vurdere dem på en skala fra 1 til 5, i form af stjerner. Dette er gjort da der i følge user stories 15 er en efterspørgesel på at kunne vuderes opskrifter.
Brugerne kan se den gennemsnitlige vurdering på den enkelte opskrift, og ud fra dette taget en beslutning om hvorvidt man skal prøve den ene eller den anden opskrift.
Brugen af disse vurderinger, bliver beskrevet i \myref{anbefaling}.


\subsection{Implementation}
\textbf{Opskrift}
Selve opskriften er en klasse set på \myref{diagram:klassediagram} Objektet bliver instanitseret når en bruger prøver at oprettet en ny opskrift eller prøver at oprette en kopi af en allerrede eksiterene opskrift.


\begin{lstlisting}[caption="Klassen Recipe som svarer til objektet\, opskrift"]
public class Recipe
{
    public int ID { get; set; }
    public string OriginalAuthorName { get; set; }
    public string AuthorName { get; set; }
    public string Title { get; set; }
    public ICollection<Item> Ingredients { get; set; }
    public int Minutes { get; set; }
    public string Instructions { get; set; }
    public ICollection<Rating> Ratings { get; set; }

    public Recipe()
    {
        Ingredients = new List<Item>();
        Ratings = new List<Rating>();
    }
}
\end{lstlisting}

\textbf{AddIngredient}
AddIngredient som ses på  er en metode til at tilføje generiske vare til opskrifterne, således de kan skaleres med antallet af personer valgt. AddIngredient er en metode der ligner AddItem fra ShoppingListsController. AddItem udskiller sig så på er at der ikke bliver fundet tilbud på varen, og den varen bliver lagt til en opskrift i stedet for en indkøbsliste.


\begin{lstlisting}[caption="AddIngredient som opretter og tilføjer ingredienser til en specifik opskrift"]
 public ActionResult AddIngredient(int id, string name, double? amountPerPerson, string unit, int numPersons)
 {
 	var recipe = _db.Recipes.Include(r => r.Ingredients).First(x => x.ID == id);
 	
 	if (name.Trim() == string.Empty)
 	{
 		return RedirectToAction("Ingredients/" + id);
 	}
 	//allows for ingredienst with no amount and unit
 	if (amountPerPerson == null)
 	{
 		amountPerPerson = 0;
 	}
 	
 	// If numpersons is set to zero, this value shouldn't be infinity (will give db exception)
 	amountPerPerson = numPersons > 0 ? (double)amountPerPerson / numPersons : 0;
 	
 	//Search in GenericLItems for item
 	Item knownItem = null;
 	if (_db.Items.Any())
 	{
 		knownItem = _db.Items.FirstOrDefault(i => i.Name.CompareTo(name) == 0);
 	}
 	
 	var tmpIngredient = knownItem ?? new Item { Name = name };
 	
 	// If recipe contains tempIngredient add ingredient, else initiate ingredient and then add it.
 	if (recipe.Ingredients.Contains(tmpIngredient))
 	{
 		_db.Recipe_Ingredient.First(i => i.Recipe == recipe && i.Ingredient == tmpIngredient).AmountPerPerson = (double)amountPerPerson;
 	}
 	else
 	{
 		var recipeIngredient = new Recipe_Ingredient { RecipeID = id, Ingredient = tmpIngredient, AmountPerPerson = (double)amountPerPerson, Unit = unit };
 		recipe.Ingredients.Add(tmpIngredient);
 		_db.Recipe_Ingredient.Add(recipeIngredient);
 	}
 	
 	_db.SaveChanges();
 	return RedirectToAction("Ingredients/" + id);
 }
\end{lstlisting}


\subsection{Konklusion}
Implementationen af opskrifter virker efter hensigten. 
Alle user stories vedrørende opskrifter bliver opfyldt. 
Opskrifter er skrevet sammen med indkøbslisten og der er funktionalitet til at tilføje vare dertil. Brugeren kan ændre på enge opskrifter eller oprette egen version af andres opskrifter. 
Opskrifterne kan skalere efter antal personer og har en vurdering. 
Alle opskrifter med dets information bliver gemt i databasen på en hensigtsmæssig måde.
