\section{Opskrifter}
\subsection{Design}
Opskrifter i dette projekt har som udgangspunkt rødder i user storie 13, som beskriver at brugeren ville have mulighed for at kunne finde opskrifter.
Idéen ved opskrifter, er at give brugeren mulighed for nemt at finde opskrifter som har en pæn vurdering af andre brugere, hvorefter ingredienserne fra disse, nemt kan tilføjes til deres indkøbsliste(r).
Når man først går ind på opskrifterne, bliver man mødt af alle offentliggjorte opskritfer, som er blevet tilføjet af brugerne af systemet.
Opskrifterne er opstillet således der på forsiden for opskrifter kan ses den mest nødvendige information om hver enktle opskrift.
Opkrifterne er opstillet som cards for at give en klar opdeling og en gøre det overskueligt.
Opskrifterne indeholder alle sammen en titel, tid, antal personer, bedømmelse, ingredienser og en fremgangsmåde.


Hver opskrift er gemt under den bruger som har oprettet den, denne bruger har senere mulighed for at redigere eller slette opskriften efter valg.
Dette er gjort da user storie 18 ligger til grundlag for dette.
Er man ikke den bruger der har oprettet opskriften, har man mulighed for at kunne klone opskriften.
Når en opskrift bliver klonet, bliver alt data fra den tidligere opskrift kopieret og derefter kan man foretage ændringer til opskriften, gemme den under et nyt navn, således der kan personliggøres.
Brugerne har efterspurgt mulighed for at kunne tilføje ingridienser fra opskriften til deres indkøbs liste, i følge user storie 16.
Ingredienser på opskriften er opstillet således de nemt kan tilføjes til brugerens indkøbslister.
Det er muligt kun at vælge de ting man mangler, således man ikke er tvunget til at tilføje alle ingredienser fra hele opskriften.
Der kan ved hjælp af UI'et vælges hvilket indkøbsliste varene skal tilføjes til, og det kan ses hvilke ting der allerrede er blevet tilføjet fra opskriften, i form af små grønne flueben.
Der er på baggrund af user storie 17 gjort det muligt at der inde på den enkle opskrift kan vælges hvor mange personer man ønsker at opskriften skal skaleres til.
Der er taget udgangspunkt i fire personer, men der er mulighed for at både sænke og hæve denne værdi for at tilpasse mængden på ingredienserne, så ledes den passer til det valgte antal mennesker.
Hver opskrift har den mulighed, at en bruger kan vurdere dem på en skala fra 1 til 5, i form af stjerner. Dette er gjort da der i følge user stories 15 er en efterspørgesel på at kunne vuderes opskrifter.
Brugerne kan se den gennemsnitlige vurdering på den enkelte opskrift, og ud fra dette taget en beslutning om hvorvidt man skal prøve den ene eller den anden opskrift.
Brugen af disse vurderinger, bliver beskrevet i \myref{anbefaling}.


\subsection{Implementation}
\subsection{Konklusion}

\section{Anbefalingssystem}\label{anbefaling}

\subsection{Teori}
Inden for anbefalingssystemer findes der mange forskellige retninger.
Hvilken retning et givent projekt benytter sig af ender ud i nogle valg om hvilke typer anbefalinger man ønsker, samt hvilke data man er i besiddelse af, hvorpå man kan basere sine anbefalinger.
I følgende afsnit gennemgåes kort nogle forskellige grene og designs indenfor anbefalingssystemer.

Nogle af de simpleste anbefalingssystemer, er upersonlige anbefalinger.
Eksempler på dette kan være en simpel sotering af objekter.
Sådanne sorteringer vil ofte være baseret på ting som popularitet, salg, sidevisninger eller ligende.
Den anden overordnede retning indenfor anbefaling er personlig anbefaling, dette vil sige at systemet benytter noget data som det har om en bruger, til at anbefale ting, der kunne være særlig interessant for brugeren.
Ved sådanne personlige anbefalinger skal systemet bruge en smagsprofil af dets brugere, for at personliggøre afbefalingerne ud fra.
Der findes flere forskellige metoder til at genere personlige anbefalinger, vi vil herunder kort fortælle om to af de mest af almindelige hovedretninger\citep{RecommenderSystems}.

\subsubsection{Indholdsbaserede anbefalinger}
Indholdsbaserede anbefalinger anbefaler en bruger ting, ud fra en brugers tidligere handlinger i et system.
For hver bruger i et sådan system kan man tilskrive denne en smagsprofil, der beskriver hvor godt en bruger synes om forskellige attributter, baseret på tidligere handlinger af brugeren.
Sådanne smagsprofiler kan genereres ud fra handlinger, som vurderinger, sidevisninger, eller tidligere køb.
Hver objekt i sådan et system vil så have nogle attributter, og ud fra disse, vil man kunne udregne en brugers smagsprofil.
Når systemet så har en smagsprofil for en person, vil man kunne sammenligne et objekts attributter med smagsprofilen, for at se om objektets attributter består af noget brugeren synes om.

Denne anbefalingsimplementation kræver at man har nogle attributter tilhørende de objekter man gerne vil anbefale brugerne.
Den største fordel ved denne metode  er at man kan lave personlige anbefalinger lige så snart en bruger har generet data, til en smagsprofil.
Ulemperne er man skal have gode attributter til sit data, for denne metode virker. Ligeledes kan denne implementation heller ikke tage højde for betingede regler, eksempelvis hvis kun kan lide attribut1 i forbindelse med atrribut2 men ikke kan fordrage attribut1 i forbindelse med attribut3.


\subsubsection{Kollaborativ anbefalinger}
Kollabrativ anbefalinger er også bygget op om smagsprofiler.
Denne implementation kræver, modsat den indholdsbaserede, ingen data om de objekter som systemet skal anbefale. I denne metode tager man i stedet brugernes smagsprofiler og holder op mod hinanden for at finde anbefalinger til en bruger. I myref{tabel:kollabrativ} ses et eksempel på hvordan denne implementation kan fungerer i praksis.

Fordelene ved denne er som nævt for at den kan bruges på mangelfuld data, hvortil dårlig eller ingen attributter er tilgængelige.
Ligeledes slipper man også for problemet om betingede regler.
Dog er en stor ulempe ved denne metode det man betegner som ‘cold start’-problemet, hvor man mangler brugeres vurderinger, for at kunne abefale dem ting\citep{RecommenderSystems}.

\subsection{Design}
Vi har valgt
\begin{table}[H]
  \centering
    \colorlet{shadecolor}{gray!40}
    \rowcolors{1}{white}{shadecolor}
      \begin{tabular}{l|lccccccc}
      %\hline
      \textbf{Opskrifter}   & A        & B       & C       & D       & E   \\ \hline
      Indgrediens1          & +        & +       & +       &         &     \\
      Indgrediens2          &          & +       &         & +       &     \\
      Indgrediens3          &          &         & +       &         & +   \\
      Indgrediens4  		    &          &         &         & +       & +   \\ \hline
      Brugers vudering      & 1        & 5       &         &         &     \\
      Systems forudsigelse  &          &         & 3       & 4       &  3  \\


    \end{tabular}
  \caption{Indholdsbaseret anbefalingsmodel.}\label{tabel:opskriftanbefaling}
\end{table}


\subsection{Implementation}

\subsection{Konklusion}

