\chapter{Indledende Interviewrunde}

\begin{itemize}[nolistsep,noitemsep]
	\setlength{\itemsep}{0em}
   \item Interview-type: Semi-struktureret
   \item Antal forventede interviews: 5 - 10
   \item Interviewsted: Føtex
   \item Tidspunkt: 17/09/14 - 13:00
   \item Interviewer: Mathias Sass Michno
   \item Referent: Marc Tom Thorgersen
   \item Målgrupper:
   	\begin{itemize}[nolistsep,noitemsep]
   		\item Unge under uddannelse(18 - 28), Alene boende
   		\item Unge under uddannelse(18 - 28), Sammenboende
   		\item Familie med hjemmeboende børn(28 - 50)
	    \item Alene boende(28 - 50)
   		\item Par uden børn(28 - 60)	
   	\end{itemize} 
\end{itemize}

Hej! i forbindelse med vores projekt, på 3. semester software, om indkøb, madlavning og vaner derom,  vil vi gerne spørge dig om nogle spørgsmål. Det tager omkring 6 minutter.

\textbf{Demografiske spørgsmål}
\begin{enumerate}[topsep=0ex]
	\setlength{\itemsep}{0em}
	\item  Hvor gammel er du ?
	\item  Bor du alene eller sammen med nogen?
	\begin{itemize}
	\item Hvem bor du med?(Børn, Venner, Kæreste)
	\end{itemize} 
\end{enumerate}

\textbf{Undersøgelsesspørgsmål}
		   
\begin{enumerate}[topsep=0ex]
	\setlength{\itemsep}{0em}
	\item  Hvordan finder du/i ud af hvad du/i skal spise derhjemme?
	\item  Er der noget der er svært når i skal finde ud af hvad i skal spise?
	\item  Er det dig der laver mad der hjemme?
	\item (hvis sammenboende) Er du den eneste der handler ind?
	\item Er der noget der er svært når du er ude og handle ind ? - F.eks. beslutninger om hvilke varer, huske hvad man skal have, osv.
	\item Hvordan aftales der på hjemmefronten hvem der handler ind?
	
	\item Kigger du i tilbudsaviser?
	\begin{itemize}
		\item Hvis ja, handler du så efter de gode tilbud, eller handler du det samme sted hver gang?
		\item Hvis ikke, hvordan beslutter du så hvilke varer og hvor du køber ind ? (lyst, opskrifter)
	\end{itemize}
	
	\item Benytter du/i indkøbslister?
	\begin{itemize}
	\item Hvis ja, hvilken type? (Elektronisk, eller papir)
	\item Følger du altid denne? og hvorfor bruger du indkøbslisten?
	\item Hvis nej, hvad gør du/i så for at huske hvad i mangler?
	\item Får du flere impulskøb med hjem når du ikke bruger indkøbsliste?
	\end{itemize}
	
	\item Bruges opskrifter i forbindelse med din/jeres madlavning til f.eks. inspiration eller til at følge dem fuldt ud.
	\begin{itemize}
	\item Hvis ja, hvordan og hvor ofte?
	\end{itemize}
	
	\item Hvis du på en elektronisk indkøbsliste fik opskriftsidéer med den varer du køber, ville du benytte dette?
\end{enumerate}		   


















 


