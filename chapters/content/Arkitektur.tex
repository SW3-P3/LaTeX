\section{Arkitektur}
Dette afsnit beskriver arkitekturen der benyttes i implementationen for systemet.
Først gennemgås design mønsteret der benyttes, derefter systemets klasser, samt dets komponenter.

Nogle komponenter har vi valgt at benytte, ASP.NETs indbyggede programdele.
For at opfylde user story 10, skal der altså laves et login system.
Vi benytter os af en færdig login løsning, der er tilgængeligt i ASP.NET teknologien (se eventuelt \myref{aspnet}).
Dette er en gennemtestet løsning udviklet af microsoft kaldet OWIN, hvilket gør det sikkert for os at vælge dette som vores login platform.
Dette har vi valgt, da vi har vurderet at login ikke er en central problemstilling i forhold til studieordningen og projektets mål.
Dette tillader os at bruge mere tid på andre dele af programmet, som vi finder mere relevante i forhold til de opstillede mål.

\subsection{MVC-mønsteret}\label{MVC}

Et af de standardiserede design mønstre, som bruges af mange udviklere er MVC-mønstret - som står for \textit{Model-View-Controller}, en figur over modellen kan ses på \myref{MVC-Figure}.
MVC-mønsteret har til formål, at dele systemet op i tre komponenter, \textit{Model}, \textit{View} og \textit{Controller}.
Denne segregering adskiller således forretnings-logik, input-logik og UI-logik, og gør herved systemet mere fleksibelt, samt fremmer muligheden for at udvikle parallelt på de forskellige komponenter.
Dette kan være nyttigt i udviklingen af systemet, men også efter udgivelsen, idet blandt andet UI-logik kan blive ændret oftere end for eksempel forretnings-logik.
Opdelingen hjælper også til at skabe overblik over koden, og gør det nemmere at unit teste systemet via dets controllere. \citep{MVC_Overview}

Nedenfor beskrives de tre komponenter.

\textbf{Model}\\
Objekterne der udgør model-laget indeholder forretnings-logik, samt alle data der skal modelleres i systemet.

\begin{wrapfigure}{r}{0.4\textwidth}
	\vspace{0pt}
	\begin{center}
		\includegraphics[width=0.38\textwidth]{images/Images/mvc.png}
	\end{center}
	\vspace{-20pt}
	\label{MVC-Figure}
	\caption{MVC-mønsteret}
	\vspace{-30pt}
\end{wrapfigure}

\textbf{View}\\
Det er igennem forskellige views, at brugeren får præsenteret brugergrænsefladen. 
Derfor giver det også mening, at placere UI-logik i denne del af MVC-mønsteret.
Indholdet i et view genereres ud fra data fra modellen.
Et eksempel på dette ville være visning af en liste af objekter ud fra en model, der indeholder netop en liste.

\textbf{Controller}\\
Controlleren håndterer interaktionen mellem brugeren og systemet.
Derfor er det også i de forskellige controllerer, at vi finder input-logikken. 
Her bestemmes ud fra input fra brugeren, hvilket data der skal arbejdes med i hvilken model og hvilket view, der skal præsenteres for brugeren.



\subsection{Program komponenter}\label{subsec:komp}

Der kan dannes forskellige komponenter i programmet ud fra de user stories der blev opgivet på \myref{sec:krav}.
Disse kan ses på \myref{figure:komp}.
Figuren viser hvordan de forskellige user stories kan deles op i komponenter, og hvordan disse komponenter afhænger og bruger hinanden som beskrevet ud fra vores user stories.

\begin{figure}
	\vspace{-20pt}
	\begin{center}
		\includegraphics[scale=0.6]{images/Diagrams/Komponenter.png}
	\end{center}
	\vspace{-20pt}
	\caption{UML 2 Komponent diagram }
	\label{figure:komp}
	\vspace{-20pt}
\end{figure}

\textbf{Indkøbslistekomponentet} står for at lave indkøbslister, og tilføje varer og tilbud til disse.
Tilbudene skal importeres fra \textbf{tilbudskomponentet}, for at lave koblingen imellem varerne og tilbud.
Det er afhængigt af \textbf{brugerkomponentet} da indkøbslisterne hører til en bruger for at danne personlige lister. Brugerkomponentet er ansvarlig for alt der har med en bruger at gøre. Det indebærer at registrere hvem der er logget på, håndtere brugerens personlige indstillinger. Det er altså dette komponent der sørger for den personlige oplevelse på hjemmesiden.

\textbf{Opskriftskomponentet} er afhængig af brugerkomponentet, da det er brugerene i systemet der tilføjer opskrifter, samt giver opskrifterne en rating.
Dog er det opskriftskomponentet der står for dette, men det skal dog vide hvem der er logget ind på hjemmesiden.
Der er en adgang her fra til indkøbslisterne for at kunne tilføje ingredienser fra opskrifterne til indkøbslister.

\textbf{Overvågningskomponentet} står for at overvågningen af tilbud for en specifik varer kan foregå.
Der dannes en afhængighed herfra og til indkøbslistekomponentet da overvågningslisten gør brug af de metoder som der stilles til rådighed i dette komponenet, såsom tilføje varer, og finde tilbud.


\subsection{Klassediagram}
For at illustrere modellaget i vores MVC-mønster, har vi produceret et klassediagram (se \myref{diagram:klassediagram} nedenfor) i UML, der simplificerer strukturen.

\begin{figure}[H]
\centering
\includegraphics[width=0.8\linewidth]{/Diagrams/klassediagram_model_expanded_implemented.pdf}
\caption{UML klassediagram for modellaget i MVC-mønsteret}\label{diagram:klassediagram}
\end{figure}

Der er mange relationer mellem klasserne, da de forskellige klasse indeholder lister af hinanden.
En ShoppingList har en liste over Items, og disse Items har en liste over tilbud, for at binde tilbudene til det pågældende Item.
Recipe har en liste over Items, som bruges som ingredienser. 
På denne måde kan de tilføjes til ShoppingList gennem RecipeController, uden at skulle lave nye objekter, som indkøbslisten kan læse.
User har en liste over ShoppingLists, samt en liste over Prefs, eller præferencer, for at gøre hjemmesiden mere personlig og fokuseret på brugeren som er logget ind.
Der er relationstabeller mellem many-to-many forholdene, som gør det muligt at gemme ekstra værdier ved et af forholdene. 
F.eks. når man tilføjer et Item til en ShoppingList eller en Recipe, kan man angive mængder af disse Items, dette gøres igennem relationstabellen.

I de følgende afsnit, vil der gives en gennemgang af udvalgte dele af komponenterne fra \myref{subsec:komp}.


