\section{Arkitektur}
Dette afsnit beskriver systemets arkitektur.
Først gennemgås design mønsteret, MVC, der benyttes, derefter systemets klasser samt dets komponenter.

\subsection{MVC-mønsteret}\label{MVC}

Et af de standardiserede design mønstre, som bruges af mange udviklere er MVC-mønstret - som står for \textit{Model-View-Controller}, en figur over modellen kan ses på \myref{MVC-Figure}.
MVC-mønsteret har til formål, at dele systemet op i tre komponenter, \textit{Model}, \textit{View} og \textit{Controller}.
Denne segregering adskiller således forretnings-logik, input-logik og UI-logik, og gør herved systemet mere fleksibelt, samt fremmer muligheden for at udvikle parallelt på de forskellige komponenter.
Dette kan være nyttigt i udviklingen af systemet, men også efter udgivelsen, idet blandt andet UI-logik kan blive ændret oftere end for eksempel forretnings-logik.
Opdelingen hjælper også til at skabe overblik over koden, og gør det nemmere at unit teste systemet via dets controllere. \citep{MVC_Overview}

Nedenfor beskrives de tre komponenter.

\textbf{Model}\\
Objekterne der udgør model-laget indeholder forretnings-logik, samt alle data der skal modelleres i systemet.

\begin{wrapfigure}{r}{0.4\textwidth}
	\vspace{0pt}
	\begin{center}
		\includegraphics[width=0.38\textwidth]{images/Images/mvc.png}
	\end{center}
	\vspace{-20pt}
	\label{MVC-Figure}
	\caption{MVC-mønsteret}
	\vspace{-30pt}
\end{wrapfigure}

\textbf{View}\\
Det er igennem forskellige views, at brugeren får præsenteret brugergrænsefladen. 
Derfor giver det også mening, at placere UI-logik i denne del af MVC-mønsteret.
Indholdet i et view genereres ud fra data fra modellen.
Et eksempel på dette ville være visning af en liste af objekter ud fra en model, der indeholder netop en liste.

\textbf{Controller}\\
Controlleren håndterer interaktionen mellem brugeren og systemet.
Derfor er det også i de forskellige controllerer, at vi finder input-logikken. 
Her bestemmes ud fra input fra brugeren, hvilket data der skal arbejdes med i hvilken model og hvilket view, der skal præsenteres for brugeren.



\subsection{Program komponenter}\label{subsec:komp}
Der kan dannes forskellige komponenter i programmet ud fra de user stories, der blev opstillet i kravspecfikationen.
Disse user stories kan ses på \myref{figure:komp}.
Figuren viser, hvordan de forskellige user stories kan deles op i komponenter, og hvordan disse komponenter afhænger af og bruger hinanden, som beskrevet ud fra vores user stories.

\begin{figure}
	\vspace{-20pt}
	\begin{center}
		\includegraphics[scale=0.6]{images/Diagrams/Komponenter.png}
	\end{center}
	\vspace{-20pt}
	\caption{UML 2 Komponent diagram }\label{figure:komp}
	\vspace{-20pt}
\end{figure}

\textit{Indkøbslistekomponenten} står for at lave indkøbslister og tilføje varer og tilbud til disse.
Tilbudene skal importeres fra \textit{tilbudskomponenten}, for at lave koblingen imellem varerne og tilbud.
Det er afhængigt af \textit{brugerkomponenten}, da indkøbslisterne hører til en bruger for at danne personlige lister. 

\textit{Brugerkomponenten} er ansvarlig for alt, der har med en bruger at gøre. 
Det indebærer at registrere, hvem der er logget på, håndtere brugerens personlige indstillinger. 
Det er altså denne komponent, der sørger for den personlige oplevelse på hjemmesiden.

\textit{Opskriftskomponenten} er afhængig af brugerkomponenten, da det er brugerene i systemet, der tilføjer opskrifter og giver opskrifterne en vurdering.
Det er opskriftskomponenten, der står for vurderingen og skal vide, hvem den aktive bruger er.
Der er en adgang herfra til indkøbslisterne for at kunne tilføje ingredienser fra opskrifterne til indkøbslister.

\textit{Overvågningskomponenten} står for, at overvågningen af tilbud for en specifik vare, kan foregå.
Der dannes en afhængighed herfra og til indkøbslistekomponenten, da overvågningslisten gør brug af de metoder, som stilles til rådighed i dette komponent såsom at tilføje varer og finde tilbud.

Komponenterne på \myref{figure:komp}, har lav kobling med hinanden.
Dette skyldes at de forskelliges komponenters implementeringer let kan skiftes ud, uden at skulle ændre på de andre komponenters kode.


Der er desuden, for at teste programmet, udviklet et administrationspanel, i \class{AdminPanelController.cs}. 
Der er forskellige funktioner heri, som gør det muligt at ændre systemtiden i programmet, hente tilbud ind via JSON data gemt i en lokal fil, og meget mere.

\subsection{Klassediagram}
For at illustrere modellaget i vores MVC-mønster, har vi produceret et klassediagram (se \myref{diagram:klassediagram} nedenfor) i UML, der simplificerer strukturen.

\begin{figure}[H]
\centering
\includegraphics[width=0.8\linewidth]{/Diagrams/klassediagram_model_expanded_implemented.pdf}
\caption{UML klassediagram for modellaget i MVC-mønsteret}\label{diagram:klassediagram}
\end{figure}

Der er mange relationer mellem klasserne, da de forskellige klasse indeholder lister af hinanden.
En ShoppingList har en liste over Items, og disse Items har en liste over tilbud, for at binde tilbudene til det pågældende Item.
Recipe har en liste over Items, som bruges som ingredienser. 
På denne måde kan de tilføjes til ShoppingList gennem RecipeController, uden at skulle lave nye objekter, som indkøbslisten kan læse.
User har en liste over ShoppingLists, samt en liste over Prefs, eller præferencer, for at gøre hjemmesiden mere personlig og fokuseret på brugeren som er logget ind.
Der er relationstabeller mellem many-to-many forholdene, som gør det muligt at gemme ekstra værdier ved et af forholdene. 
F.eks. når man tilføjer et Item til en ShoppingList eller en Recipe, kan man angive mængder af disse Items, dette gøres igennem relationstabellen.

I de følgende afsnit, vil der gives en gennemgang af udvalgte dele af komponenterne fra \myref{subsec:komp}.


