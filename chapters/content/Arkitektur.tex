\section{Arkitektur}
Dette kapitel beskriver arkitekturen der benyttes i implementationen for systemet.
Først gennemgås design mønsteret der benyttes, derefter systemets klasser, samt dets komponenter.


\subsection{MVC-mønsteret}\label{MVC}

Et af de standardiserede design mønstre, som bruges af mange udviklere er MVC-mønstret - som står for \textit{Model-View-Controller}, en figur over modellen kan ses på \myref{MVC-Figure}.
MVC-mønsteret har til formål, at dele systemet op i tre komponenter, \textit{Model}, \textit{View} og \textit{Controller}.
Denne segregering adskiller således forretnings-logik, input-logik og UI-logik, og gør herved systemet mere fleksibelt, samt fremmer muligheden for at udvikle parallelt på de forskellige komponenter.
Dette kan være nyttigt i udviklingen af systemet, men også efter udgivelsen, idet blandt andet UI-logik kan blive ændret oftere end for eksempel forretnings-logik.
Opdelingen hjælper også til at skabe overblik over koden, og gør det nemmere at unit teste systemet via dets controllere. \citep{MVC_Overview}

Nedenfor beskrives de tre komponenter.

\textbf{Model}\\
Objekterne der udgør model-laget indeholder forretnings-logik, samt alle data der skal modelleres i systemet.

\begin{wrapfigure}{r}{0.4\textwidth}
	\vspace{0pt}
	\begin{center}
		\includegraphics[width=0.38\textwidth]{images/Images/mvc.png}
	\end{center}
	\vspace{-20pt}
	\label{MVC-Figure}
	\caption{MVC-mønsteret}
	\vspace{-30pt}
\end{wrapfigure}

\textbf{View}\\
Det er igennem forskellige views, at brugeren får præsenteret brugergrænsefladen. 
Derfor giver det også mening, at placere UI-logik i denne del af MVC-mønsteret.
Indholdet i et view genereres ud fra data fra modellen.
Et eksempel på dette ville være visning af en liste af objekter ud fra en model, der indeholder netop en liste.

\textbf{Controller}\\
Controlleren håndterer interaktionen mellem brugeren og systemet.
Derfor er det også i de forskellige controllerer, at vi finder input-logikken. 
Her bestemmes ud fra input fra brugeren, hvilket data der skal arbejdes med i hvilken model og hvilket view, der skal præsenteres for brugeren.



\subsection{Klassediagram}
For at illustrere modellaget i vores MVC-mønster, har vi produceret et klassediagram (se \myref{diagram:klassediagram} nedenfor) i UML, der simplificerer strukturen.

\begin{figure}[H]
\centering
\includegraphics[width=0.8\linewidth]{/Diagrams/klassediagram_model_expanded_implemented.pdf}
\caption{UML klassediagram for modellaget i MVC-mønsteret}\label{diagram:klassediagram}
\end{figure}

Der er mange relationer mellem klasserne, da de forskellige klasse indeholder lister af hinanden.
En ShoppingList har en liste over Items, og disse Items har en liste over tilbud, for at binde tilbudene til det pågældende Item.
Recipe har en liste over Items, som bruges som ingredienser. 
På denne måde kan de tilføjes til ShoppingList gennem RecipeController, uden at skulle lave nye objekter, som indkøbslisten kan læse.
User har en liste over ShoppingLists, samt en liste over Prefs, eller præferencer, for at gøre hjemmesiden mere personlig og fokuseret på brugeren som er logget ind.
Der er relationstabeller mellem many-to-many forholdene, som gør det muligt at gemme ekstra værdier ved et af forholdene. 
F.eks. når man tilføjer et Item til en ShoppingList eller en Recipe, kan man angive mængder af disse Items, dette gøres igennem relationstabellen.

I de følgende afsnit, vil der gives en gennemgang af udvalgte dele af komponenterne fra \myref{subsec:komp}.



\fxnote{De resterende komponeneter}

Der kan dannes forskellige komponenter i systemet ud fra de user stories der blev opgivet på \myref{sec:krav}. 
Disse kan ses på figur \myref{figure:komp}

\begin{wrapfigure}{r}{0.6\textwidth}
	\vspace{-20pt}
	\begin{center}
		\includegraphics[width=0.6\textwidth]{images/Diagrams/Komponenter.png}
	\end{center}
	\vspace{-20pt}
	\caption{Komponeneter i systemet.}
	\label{figure:komp}
	\vspace{-20pt}
\end{wrapfigure}

Her tænker jeg at dele use cases op i forskellige dele.
F.eks. et UserManagement Komponent, der sørger for at man kan logge ind og have præferencer osv.
Og et shoppingliste komponent der styrer indkøbslisten, tilføjelse og deling af varer her. Der vil så være en forbindelse imellem user management og shoppingliste, til at forbinde disse til de rigtige personer osv. og forklarer det ud på den måde og endelig med et UML diagram som viser forbindelserne imellem komponenterne.

User forbinder sig jo til -> Indkøbslister, Opskrifter(her indgår vurdering også i opskrifts komponentet), hvor tilbud forbinder sig til varer som så forbinder sig til shoppinglisterne hvor i varerne også indgår. Vil jeg tro. Dette skal vi diskutere, hvilke komponenter findes der, når vi ser på det fra user stories. Dette vil danne og vise sammenhæng i implementationen af systemet.