\section{Indkøbsliste} \label{Indkoebsliste}
User story 1 - 6, som findes i \myref{sec:krav}, beskriver funktionaliteter for en indkøbsliste.
Det kræves altså, at indkøbslisten kan: Oprettes, tilføjes varer til, vise valgte tilbud, aftjekke varer (ved køb), deles med andre brugere og tilgås fra mobil-enheder.

I dette afsnit vil design og implementering af disse user stories uddybes.

\subsection{Design}
Indkøbslisten er en meget central del af programmet, da stort set alle andre komponenter kan tilføje varer til indkøbslisterne.
Det vigtigste for indkøbslisten er, at den både kan vise varer og tilbud, på både mobilt og desktop layout, dette uddybes i \myref{brugergraenseflade} om brugergrænsefladen.
For at vise varerne skal der kunne tilføjes en vare med bl.a. en mængde og en enhed.
Når brugeren opretter sit login, oprettes en indkøbsliste til brugeren automatisk med navnet ``Min Indkøbliste''.
Brugeren skal kunne vælge, hvilket tilbud på varen de ønsker at købe, og derfor skal der for hver vare på listen kunne vises en liste over varens tilbud.

\subsection{Implementering}
Når en bruger går ind på indkøbslister, kan de se to lister af \class{ShopppingLists} de kan tilgå.
Den ene liste viser \class{ShoppingLists}, hvorpå de er den eneste bruger, og den anden liste viser \class{ShoppingLists}, der indeholder flere brugere.

Når et \class{Item} tilføjes til en \class{ShoppingList}, kaldes metoden \class{GetOffersForItem} i \class{ShoppingListController.cs}, som kan ses i \myref{getoffersforitem}.

\begin{lstlisting}[caption={Metoden \class{GetOffersForItem} finder relevante tilbud og returner dem som en liste}, label=getoffersforitem]
public static List<Offer> GetOffersForItem(IDataBaseContext db, Item item, User user)
{
    return db.OffersFilteredByUserPrefs(user).Where(x => 
     	x.Heading.ToLower().Contains(item.Name.ToLower() + " ") 
     || x.Heading.ToLower().Contains(" " + item.Name.ToLower()) 
     || String.Equals(x.Heading.ToLower(), item.Name.ToLower())).ToList();
}
\end{lstlisting}

Denne metode vil søge efter \class{Offers}, som passer på det tilføjede \class{Item}
Her tages højde for, hvilke butikker den pågældende \class{User} har fravalgt, og dermed tilføjet til sin liste over \class{Pref} (Uddybes i \myref{OffersFilteredByUserPrefs}).
Den naive implementering af denne metode ville være at anvende \class{String.Contains()} metoden, som returnerer sand, hvis den angivne streng er en substreng af den, som metoden kaldes på. 
Men dette vil give unødige falske-positive, eksempelvis hvis en bruger vil købe varen ``Burger'', så vil tilbud på ``hamburgerryg'' blive inkluderet.
Derfor er alle tilbud for en given vare kun med i listen, hvis varens navn findes efterfulgt af et mellemrum, eller forud for et mellemrum, eller passer fuldstændigt.
En bedre implementering kunne laves hvis man kunne vælge kategori på varen der søges på, såsom pålæg.
Herudover kunne man oprette generiske varer såsom chokoladebar, og ved at søge på dette, finder den relaterede varer såsom marsbar.
Der findes en blå knap ud for varen med teksten ``Se tilbud'', når knappen trykkes på udvides tabellen således brugeren kan se det pågældende \class{Item}s liste af \class{Offers}.
Hvis der vælges et af disse \class{Offers} vil det viste navn for det pågældende \class{Item} erstattes af navnet på det valgte \class{Offer}.
Dette gemmes i feltet \class{selectedOffer} på \class{shoppingList\_Item}.

På \myref{OffersFilteredByUserPrefs} ses metoden \class{OffersFilteredByUserPrefs}, som kaldes i \myref{getoffersforitem}. Denne metode sørger for at brugeren kun ser \class{Offers} hvis \class{Heading} ikke indeholder noget fra deres liste af \class{Pref}s.
F.eks. hvis man ikke vil se tilbud på mærket ``vores'', kan man her filtrere \class{Offers} fra, som indeholder ``vores'' i sin \class{Heading}. 
Der er taget et valg om at \class{Offer}s hvis \class{Heading} indeholder kommaer eller ``eller'' i sit navn, skal filtreres væk.
Desuden filtreres et \class{Offer} også væk hvis dets \class{unit} er en tom streng.
Dette er fordi de bliver meget intetsigende og kan skabe forvirring for brugeren. 
Et eksempel kan være at man søger på leverpostej, og \class{SelectOffer} hedder: ``Leverpostej eller kødpølse''. 
Hvis der så på brugerens  \class{ShopplingList} står ``Leverpostej eller kødpølse'', kan de være usikre på hvad det egentlig var de skulle købe. 
Metoden kaldes alle steder hvor der skal vises tilbud på hjemmesiden.

\begin{lstlisting}[caption={\class{OffersFilteredByUserPref} er metoden, der filtrerer \class{Offers} på baggrund af uønskede strenge.}, label=OffersFilteredByUserPrefs] 
public IEnumerable<Offer> OffersFilteredWithString(params string[] args)
{
	//Puts filterstrings into a list of strings called blacklist
    var blacklist = new List<string> { ",", "eller" };
                var fromArgs = new List<string>();
    foreach (var str in args)
    {
        fromArgs.AddRange(str.Split(','));
    }
    blacklist.AddRange(fromArgs);
    
    //Trims and removes all empty strings from blacklist
    blacklist.RemoveAll(x => x.Trim().Equals(string.Empty));

    var res = new List<Offer>();

	//Checks if OfferIsRelevant for all offers
    foreach (var o in Offers)
    {
        if (OfferIsRelevant(o, blacklist))
        {
            res.Add(o);
        }
    }
    return res;
}
\end{lstlisting}

\class{OfferIsReleant()} som kaldes af \myref{OffersFilteredByUserPrefs} kan ses på \myref{OfferIsRelevant}. Dette er metoden som giver resultatet om hvorvidt tilbudet skal tilføjes til listen over tilbud eller ej. 
Hvis den returnere false vil tilbudet ikke blive tilføjet på linie 19 i \myref{OffersFilteredByUserPrefs}.
Der tjekkes om hvorvidt det medsendte  \class{Offer} passer ind i den satte systemtid.
Her tjekkes desuden om navnet på det givne \class{Offer} indeholder nogle af de strenge i blacklist.
Desuden fjernes alle \class{Offer}s uden enhed har.

\fxnote{Denne forklareing burde blive fjernet}
\begin{lstlisting}[caption={\class{OffersFilteredByUserPrefs}, sørger for at tilbudet som modtages som input overholder de forskellige krav sat i blacklisten.}, label=OfferIsRelevant]
private static bool OfferIsRelevant(Offer o, IEnumerable<string> blacklist)
{
    if (o.End < GlobalVariables.CurrentSystemTime)
        return false;

    if(o.Begin > GlobalVariables.CurrentSystemTime)
        return false;

    if (blacklist.Any(item => o.Heading.ToLower().Contains(item.ToLower()) || o.Store.ToLower().Contains(item.ToLower())))
        return false;

    if (o.Unit.Trim() == "")
        return false;

    // Base case.
    return true;
}
\end{lstlisting}

\subsection{Konklusion}

User stories for indkøbslisten er opfyldt, dog er implementeringen der søger efter tilbud for en vare ikke optimal.
Dette skyldes bl.a. det mangelfulde data som beskrevet i \myref{skoddata}.
En bedre implementering kunne laves hvis der f.eks. fandtes kategorier på tilbudene, samt generiske varer.
Som brugertestene viser, i \myref{ss:bt2}, fungerer implementeringen for slutbrugeren på trods af disse mangler.
