\section{Indkøbsliste} \fxnote{Burde nok være før overvågningslisten? - Troels}
% Hvad bruges indkøbslisten til og hvilken user story hører den til.
Fra vores prototype interviews, \myref{section:interview2}, blev der ønsket muligheden for indkøbslister som både kunne bruges individuelt samt deles med andre, hvorpå der både kunne være tilbud eller blot generelle varer. 
Dette er også userstory nr. 1 - 5,\fxnote{US:6 kræver smartphone adgang, er ikke med i dette} \myref{sec:krav}. 
Det er krav at indkøbslisten kan: Oprettes, tilføjes varer til, se valgte tilbud, aftjekke varer (ved køb), dele med andre bruge og tilgås fra smartphones.
%\subsection{Teori}
% Ingen teori??
\subsection{Design}
% Relater den her til overvågningslisten som er en specifik indkøbsliste
Indkøbslisten er en meget central del af programmet, da stort set alle andre dele tilføjer varer til indkøbslisterne.
Det vigtigste for indkøbslisten er at den både kan vise generelle varer og specifikke tilbud. 
Hver indkøbsliste skal også understøtte at den kan deles med andre brugere. 
\subsection{Implementation}
% Web UI
Indekssiden for indkøbslister tilgås via knappen i menubaren, eller via urlen ``/ShoppingLists''.
På den kan en bruger se hvilke indkøbslister personen har, disse er opdelt således at dem som er delte med andre er separeret fra dem som er personlige.
Der er også en kort forklarende teskt, som oplyser brugeren om hvad det er muligt at bruge denne del af programmet til.
Der er en knap til at oprette en ny indkøbsliste, den åbner en såkaldt ``Modal'', som er en lille pop-up boks, hvori brugeren kan skrive indkøbslistens navn. 
Der er på denne side også muligt at ændre titlen på en indkøbsliste eller slette den.

Ved at klikke på knappen ``Tilføj og fjern varer'' eller blot på en indkøbsliste vil man tilgå en ny side som viser indholdet af den.
Fra denne side er det nu muligt at dele indkøbslisten med andre bruger baseret på deres e-mail adresse i programmet.
Det er samtidig muligt at se hvilke personer den er delt med på nuværende tidspunkt.
På siden kan brugere også tilføje nye varer til deres indkøbsliste, det er også muligt at specificere mængden og en tilhørende enhed til hver vare, men dette er valgfrit.
Efter en vare er tilføjet til en indkøbsliste forsøger programmet at finde tilbud som passer dertil, dette forgår i metoden ``GetOffersForItem'' i ``ShoppingListController.cs''.
\begin{lstlisting}[caption="Metoden ``GetOffersForItem'' finder relevante tilbud og returner dem som en liste", label=getoffersforitem]
public static List<Offer> GetOffersForItem(IDataBaseContext db, Item item, User user)
{
    return db.OffersFilteredByUserPrefs(user).Where(x => 
     	x.Heading.ToLower().Contains(item.Name.ToLower() + " ") 
     || x.Heading.ToLower().Contains(" " + item.Name.ToLower()) 
     || String.Equals(x.Heading.ToLower(), item.Name.ToLower())).ToList();
}
\end{lstlisting}
Denne metode vil søge de tilbud, som er i butikker brugeren kan lide og ikke er på listen over til brugeren ikke ønsker at købe, som angivet i brugerinstilliger. \fxnote{Har vi forklaret implementationen af modellaget, entify framework i brug og det filter (OffersFilteredByUserPrefs ? - Troels)}
Den naive implementering af denne metode ville være at anvende ``String.Contains'' metoden, som returnerer sand hvis den angivne string er en substring af den som metoden kaldes på. 
Men dette vil give unødige falske-positive, eksempelvis hvis en bruger vil købe varen ``Burger'', så vil tilbud på ``hamburgerryg'' blive inkluderet.
Derfor er alle tilbud for en given vare kun med i listen hvis vare navnet findes efterfulgt af et mellemrum, eller forud for et mellemrum, eller passer fuldstændigt.
For at tilgå denne information kan brugeren trykke på den blå knap ud for  ``Se tilbud'', dette vil udvide tabellen således brugeren kan se tilbudene og vælge et af dem.
Hvis der vælges et at tilbudene vil varen blive erstattet af informationen om tilbudet, da dette oftere er mere specifikt, derudover vises informationen om tilbudet hver gang siden tilgås.

Det er muligt at aftjekke varer som købte ved at klikke på deres navn, når en bruger holder deres cursor over teksten kommer der en lille besked som viser dette. 
Det er også muligt at fjerne varer helt fra indkøbslisten ved at trykke på den røde knap med et kryds under ``Fjern''. 

Til sidst er det muligt at ryde hele indkøbslisten for varer ved at trykke på ``Ryd Liste'' linket i bunden af siden.
% undgå mobile ui (hvis dette skrives i andet afsnit)
% Backend ??
\subsection{Konklusion}
Det er muligt at have personlige indkøbslister eller dele dem med andre brugere, hvis en indkøbsliste er delt kan alle brugerne anvende den som var det deres egne. 
Det er også muligt at tilføje varer, aftjekke dem, og tilgå tilbud baseret på de varer man har tilføjet. 
Implementeringen udfylder alle userstories, det er dog op til testsne at afgøre om det er tilstrækkeligt udført. 
