\section{Brugertests}
For at kvalitetssikre systemet, er det blevet afprøvet af evt. slutbrugere i udviklingsforløbet såvel som det endelige produkt, i slutningen af projektet.
Dette gøres for at få brugernes indsigt i hvordan systemet bør se ud og agere ifølge brugerne.
Dette hjælper med at udvikle et system den almene bruger har forståelse for, frem for udelukkende ekspertbrugere, som vi i projektgruppen afspejler.
I dette afsnit gennemgås formål, udførsel, såvel som resultater for de to brugertests lavet på systemet, henholdsvis midt i udviklingen og ved det endelige program.


\subsection{Brugertest 1}
Dette var den første brugertest udført på systemet.
På dette tidspunkt har systemet haft de fleste funktionaliteter tilgængelige.
Af de opstillede krav har brugerne i denne omgang ikke kunnet:
\begin{itemize}
   \item{Modtage anbefalinger af opskrifter}
   \item{Overvåge eller modtage notifikationer om tilbudsvare}
   \item{Anvende systemet fra en platform anden end computer}
\end{itemize}
Formålet med brugertesten var at få feedback om, hvorvidt systemet kunne benyttes af potentielle slutbrugere, såvel som at se hvordan de reagerede på systemets design.
Med et delvist færdigt produkt ville vi gerne se en indikation om, i hvor høj grad henholdsvis programmets funktionalitet og brugergrænseflade levede op til brugernes forventninger.

Hertil blev der udvalgt en række personer fra hovedmålgruppen, som blev fundet igennem gruppemedlemmernes netværk, 4 kvinder såvel som 4 mænd, alle brugerne var i alderen 20-24.
Hver enkelt bruger blev introduceret til de samme opgaver, disse kan findes i bilag på \myref{b:brugertest1}, denne process blev filmet vha. SilverBack, og derefter analyseret.
Som resultat af testen identificerede vi en række af problemer som brugerne stødte ind i.
En opsummering af disse problemer for de forskellige komponentdele, indkøbsliste, tilbud, opskrifter og brugerkomponent findes nedenfor.
Resultater som ikke er nævnt fra testen af systemet, består enten af problemer kun enkelte har nævnt, eller små kosmetiske kommentarer til systemet.
Disse bemærkninger og fejl er ikke blevet ignoreret, men en forklaring af dem virker overflødigt, f.eks. at de ikke kunne lide farven. 
Dette betyder dog ikke at de ikke tages med i overvejelserne, og evt. ændres.
En komplet liste over resultatet af testen kan findes på \myref{b:brugertestrespons}.
\begin{enumerate}
   \item \textbf{Indkøbsliste} \begin{itemize}
   								  \item Til indkøbslisten var der få kommentarer, den primære respons herom var til ''Se Tilbud'' funktionaliteten. Hvis en given vare ikke matchede nogle tilbud, blev der ikke vist noget ikon. Som følge heraf gik det ikke op for alle at denne funktionalitet overhovedet eksisterede før de havde søgt på en vare som havde et tilbud. Derfor ønskes det at tydeliggøres hvis der ikke er fundet nogle tilbud på en vare, frem for at intet vises.
   							   \end{itemize}
   \item \textbf{Tilbud}\begin{itemize}
   								  \item Flere var i tvivl om hvorvidt tilbudene var aktuelle på tidspunktet programmet blev anvendt, således kunne en informativ tekst omkring tilbudene tilføjes til denne sektion af systemet, for at give brugeren mere præcis information, om hvad der ses på skærmen.
   								  
   								  \item En anden ønsket funktionalitet var at varer blev kategoriseret, eksempelvis som mejeri, pålæg, kød osv. Dette problem er desværre uløseligt med den valgte løsning.
   								  Dette er som følge af den information, der tildeles systemet igennem eTilbudsavis' API, dette er beskrevet i \myref{api:skoddata}.
   							   \end{itemize}
   \item \textbf{Opskrifter}\begin{itemize}
   								  \item En absolut kritisk fejl der ramte samtlige testpersoner, var at navigere ind på en individuel opskrift. Som følge af mangel på highlighting, var det ikke klart for brugeren at man kunne trykke på en opskrift, løsningen her, bliver således at implementere highlighting.
   								  
   								  \item Som en del af opgaverne skulle man tilføje ingredienser til sin indkøbsliste, her ønskede de fleste en ''Tilføj Alle Varer'' funktion, da det var træls at trykke på varerne enkeltvis.
   								  En mindre gruppe ønskede yderligere at der ligeledes kunne fjernes vare fra sin indkøbsliste, gennem opskrift sektionen af systemet.
   								  
   								  \item Nogle ønskede yderligere at indkøbsliste funktionaliteten med at se tilbud blev sat ind på opskrifter, såvel som en samlet pris på opskrifterne.
   								  Det er dog ikke muligt at tilføje en totalpris, da der både eksistere mange tilbud på samme vare på et givent tidspunkt, samt en mangel på priser på varer, som ikke er på tilbud.
   								  At kunne se tilbud, er ligeledes ikke formålet med opskrifter, derfor vil den funktionalitet være forbeholdt indkøbslisten, for at holde funktionaliteten i de enkelte sektioner, minimalistisk og konsistent.
   							   \end{itemize}
   \item \textbf{Præferencer}\begin{itemize}
   								  \item Samtlige personer mente at ''Præferencer'', var en dårlig navngivning for en funktionalitet der blacklister varer såvel som butikker.
   								  
   								  \item Denne sektion af systemet benytter sig af checkboxes, testpersonerne vidste ikke hvorvidt deres information blev gemt, som følge heraf implementeres snackbars med det formål at opdatere brugeren om gemt information.
   							   \end{itemize}
   \item \textbf{Generel design og funktionalitet}\begin{itemize}
   								  \item Aftjekningsfunktionaliteten i indkøbslisten blev fuldstændig overset, en ændring her vil være at gøre opmærksom på dens eksistens, en fremtidig test kan derefter bedømme, hvor nyttig den er, eftersom den både er overset, og ikke efterspurgt i denne omgang.
   								  \item Det var ikke kun under ''Præferencer'' der opstod forvirring om hvorvidt information blev gemt, derfor implementeres snackbars flere steder i systemet.
   								  \item Under både tilbud og opskrifter, vælger brugeren en indkøbsliste at arbejde på.
   								  Denne funktionalitet er på tidspunktet for udførelsen af testene, placeret i øverste højre hjørne.
   								  Dette er fuldstændig overset af alle brugerne, som oprettede flere lister.
   								  Gennemgang af videoerne viser at brugerne fokusere det meste af deres opmærksomhed når de kommer ind på en side, i den venstre del af siden.
   								  For at gøre mere opmærksom på denne funktionalitet, som gerne skal være noget af det første der benyttes, vil denne blive venstrecentreret.
   							   \end{itemize}
\end{enumerate}

Som følge af disse resultater, kan systemet udvikles og ændres, til at være mere brugervenligt og indbydende for den primære målgruppe.
De testede blev yderligere spurgt en række opfølgende spørgsmål omkring brugen af systemet, her var den generelle konsensus at systemet virkede som et let at benytte og brugbart system, hvorfor der var en interesse.

\subsection{Brugertest 2}
Den anden brugertest blev udført efter vi afsluttede vores sidste sprint, hvor vi arbejde på programmet.
Det vil sige at denne test, var en test af det endelige program.
Testen blev udført af syv testbrugere, i alderen 20-45 år, to af testpersonerne havde også deltage i den første brugertest.
Testpersonerne til denne test bliv, ligesom i første brugertest, fundet gennem gruppemedlemmernes netværk 
Testen blev styret af en testleder, og der sad em notattager og tog notater af hver test.
Efter alle test af alle 7 testpersoner gennemgik testlederen og notattageren notaterne sammen for at kryds om de havde glemt noget.
Efterfølgende så andre gruppemedlemmer videomatrialet igennem for krydstjekkede med notaterne, og transskriberede ligeledes interessante passager ned.

Den opstillede test, som dette forløb følger kan ses på \myref{b:brugertest2}.
De samlede resultater af denne test fremgår af \myref{b:brugertestrespons2} 

Denne test havde tre overordnede formål:
\begin{enumerate}
   \item Finde ud af om problemerne brugertest 1 viste, er blevet løst siden sidst
   \item Undersøge hvor godt programmets brugervenlighed lever op til kravene beskrevet i kravspefikationen
   \item Hjælpe som data til at vurderer om programmet opfylder de forskellige user stories beskrevet i kravspefikationen 
\end{enumerate}

\subsubsection{Test resultater}
I testen fandt vi ud af at vi ud af hvor lettte de forskellige programdele og funktioner var at benytte for vores testpersoner.
\begin{enumerate}
   \item \textbf{Indkøbsliste}
   \begin{itemize}
      \item De fleste oprettede sin egen indkøbsliste, nogle personer benyttede sig dog af den autogenereret indkøbsliste. 
      Fire personer benytter indkøbslisten funktionalitet unden nogle problemer.
      \item Tre testbrugere udtrykker utilfredshed eller forvirring over hvordan der skelnes mellem personlige og delte indkøbslister.
      Dette problem anses som kritisk. 
      \item Desuden blev der fundet følgende tre kosmetiske problemer
      \begin{itemize}
         \item En bruger opdager ikke til at starte med at man kan ændre enheder, på de varer man tilføjer til indkøbslisten.
         \item En bruger er i kort i tvivl om man benytter ''+'' fr at tilføje et tilbud til sinvare på indkøbslisten.
         \item En bruger finder først ud f hvordan man ser tilbud efter et stykke tid.
      \end{itemize}
   \end{itemize}
   \item \textbf{Tilbud}
   \begin{itemize}
      \item En bruger finder det ikke klart at alle tilbuddene er aktuelle.
      \item En bruger er i tvivl om hvilken indkøbsliste tilbuddene bliver tilføjet til.
      \item To brugere er i tvivl om de varer de tilføjer, bliver tilføjet til deres indkøbsliste.
      \item En bruger foreslår at sortere den kronologisk efter udløbsdato.
      \item Desuden blev der ønsket at man kunne se førpris på tilbuddene, og man kunne vælge at se tilbud fra sine fravalgte butikker.
   \end{itemize}
   \item \textbf{Opskrifter}
   \begin{itemize}
      \item Ved opskrifter har ingen problmer med at finde opskrifter eller vurdere dem.
      \item På mobillayoutet kommenterer en enkelt bruger at knaperne for at lave egen udgave, og ligende funktionalitet, er unødvendige, for store, og ligger træls i midten af layoutet.
      \item Ved funktionalitet med at lave sin egen udgave af en opskrift, har nogle brugere lidt forskellige problemer.
      \begin{itemize}
         \item En bruger havde problmer med at mængde skulle være et tal.
         \item En bruger komme
      \end{itemize} 
   \end{itemize}
   \item \textbf{Præferencer}
   \begin{itemize}
      \item 
   \end{itemize}
   \item \textbf{Generel design og funktionalitet}
   \begin{itemize}
      \item
   \end{itemize}
\end{enumerate}
