\section{Brugertests}
For at kvalitetssikre systemet, er det blevet afprøvet af slutbrugere i udviklingsforløbet såvel som endeligt.
Dette gøres for at få brugerens insigt i hvordan systemet bør se ud og agere ifølge brugerne.
Dette hjælper med at udvikle et system der giver mening for den almene bruger, frem for udelukkende ekspertbrugere, som har været involveret i udvikling af systemet.
Følgende i dette afsnit gennemgås formål, udførsel såvel som resultater for de forskellige brugertests lavet på systemet, i de respektive stadier systemet har været i.


\subsection{Brugertest alpha}
Dette var den første række af brugertests udført på systemet.
På tidspunkt for testning har systemet haft de fleste funktionaliter tilgængeligt, af de opstillede krav har brugerne i denne omgang ikke kunnet:
\begin{itemize}
   \item{Modtage anbefalinger af opskrifter}
   \item{Overvåge eller modtage notifikationer om tilbudsvare}
   \item{Anvende systemet fra en platform anden end computer}
\end{itemize}
Formålet med brugertesten var at få feedback om, hvorvidt systemet kunne benyttes af potentielle slutbrugere, såvel som at se hvordan de reagerede på systemets design.
Hertil blev der udvalgt en række personer fra hovedmålgruppen, som blev fundet igennem tidligere interviews, 4 unge kvinder såvel som 4 unge mænd.
Hver enkelt bruger blev introduceret til de samme opgaver, som kan findes i bilag på \myref{}, denne process blev filmet og derefter analyseret på gruppen.
Ud fra testnsne, blev der gjort opmærksom på en række af problemer som slutbrugerne stødte ind i:
\begin{enumerate}
   \item
\end{enumerate}
En nærmere opsummering af den samlede data kan findes i bilag på \myref{}.