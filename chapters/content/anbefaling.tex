\section{Anbefalingssystem}\label{anbefaling}
Udviklingen og implementeringen af et anbefalingssystem vil opfylde user story 14.
For at lave dette kræves implementering af user stories 13 og 15.
Først vil der gennemgås teori om anbefalingssystemer, efterfølgende beskrives vores design og implementering.
I vores program bliver anbefalingssystemet brugt til at sortere opskrifter for brugere, således at den opskrift systemet forventer bruger bedømmer højst, vil blive vist først. 
Anbefalingssystemet har ikke været et kerneområde dette projekt, og derfor foreslår vi også en del forbedringer i konklusionen, som vi ikke har prioriteret at implementere i projektets forløb.

\subsection{Teori}
Inden for anbefalingssystemer findes der flere forskellige metoder.
Metoden man anvender er afhængigt at det mål man har og hvilke data man er i besiddelse af.

Et simpelt upersonligt anbefalingssystem vil være en sortering efter en faktor såsom popularitet eller gennemsnitlig brugervurdering.
Den anden overordnede metode inden for anbefaling er personlig anbefaling, her benytter systemet data om en bruger, til at anbefale.
Ved brug af personlige anbefaling, skal systemet beregne og bruge en smagsprofil.
En smagsprofil er systems målestok for hvorledes en bruger syndes om objekter, disse objekt varierer fra henholdsvis fra indholdsbaserede anbefalinger til kollaborative anbefalinger.  
De vil derfor blive beskrevet i deres respektive afsnit. 
Der findes flere forskellige metoder til at generere personlige anbefalinger, vi vil herunder kort fortælle om to af de mest brugte metoder\citep{RecommenderSystems}.

\subsubsection{Indholdsbaserede anbefalinger}
Indholdsbaserede anbefalinger anbefaler ud fra brugerens tidligere handlinger i et system.
For hver bruger i et sådan system kan man tilskrive denne en smagsprofil, der beskriver hvor godt en bruger synes om forskellige attributter.
Handlingerne kan være vurderinger, sidevisninger, eller tidligere køb. 
Hvert objekt i sådan et system vil have nogle attributter, og ud fra disse, vil man kunne udregne en brugers smagsprofil.
Når systemet så har en smagsprofil på en bruger, vil det kunne sammenligne et objekts attributter med smagsprofilen, for at kunne vurderere hvert objekt imod hinanden.

Ved brug af denne anbefalingsmetode kan \textit{term frequency inverse document frequency} (tf-idf) benyttes, som vægter hver attribut, efter den reciprokke hyppighed attributten opstår i datamængden.
Hvis en attribut ikke forekommer så ofte tillægges den derfor en højere betydning end en attribut med oftere forekomst.

Denne implementering af anbefaling kræver at man har nogle attributter tilhørende de objekter man gerne vil anbefale brugerne.
Den største fordel ved denne metode er at man kan lave personlige anbefalinger lige så snart en bruger har anvendt system nok, til at en smagsprofil kan beregnes.
Ulemperne er at man skal have tilstrækkelige attributter til sine data, før denne metode virker. 
Ligeledes kan denne implementering heller ikke tage højde for betingede regler, eksempelvis hvis man kun kan lide \textit{attribut-1} i forbindelse med \textit{attribut-2} men ikke kan fordrage \textit{attribut-1} i forbindelse med \textit{attribut-3}.

\subsubsection{Kollaborativ anbefalinger}
Kollaborativ anbefalinger er også bygget op om smagsprofiler.
Denne implementering kræver, modsat den indholdsbaserede, ingen data om de objekter som systemet skal anbefale. 
I denne metode tager man i stedet brugernes smagsprofiler, her er smagsprofiler vurderinger eller køb, og holder profilen op mod andre brugeres profiler for at finde anbefalinger til en bruger.
I \myref{tabel:kollabrativ} ses et eksempel på hvordan denne implementering kan fungere i praksis.
Tabellen viser hvordan et kollabrativt system ville anbefale bruger-5 objekt C.
Denne anbefaling fortages ved at sammenlige bruger-5 smagsprofil med de andre brugeres.
Da bruger-2 og bruger-3's smagsprofiler passer bedst, benyttes deres vudering til at antage at bruger5 ikke vil kunne lide objekt C.
\citep{kollabrativEksempel}

Fordelen ved denne er, at den kan bruges på trods af manglende attributter om de objekter der vurderes.
Ligeledes slipper man også for problemet om betingede regler.
Dog er en stor ulempe ved denne metode, det man betegner som ‘cold start’-problemet, hvor man mangler mange brugeres vurderinger, for at kunne give dem anbefalinger af tilstrækkelig kvalitet.\citep{RecommenderSystems}.
\begin{table}[H]
  \centering
    \colorlet{shadecolor}{gray!40}
    \rowcolors{1}{white}{shadecolor}
      \begin{tabular}{l|lccccccc}
      %\hline
      \textbf{Objekter}                 & A        & B       & C       & D  \\ \hline
      Bruger1                           & +        & -       & +       & +  \\
      Bruger2                           &          & +       & -       & -  \\
      Bruger3                           & +        & +       & -       &    \\
      Bruger4  		                      & -        &         & +       &    \\ \hline
      Bruger5  		                      & +        & +       &         & -  \\ \hline
      Systems forudsigelse for bruger5  &          &         & +       &    \\


    \end{tabular}
  \caption{Kollabrativ anbefaling. Tabellen viser hvordan et kollabrativt system ville anbefale bruger5 objekt C.
            Denne anbefaling fortages ved at sammenlige bruger5 smagsprofil med de andre brugeres.
            Da bruger2 og bruger3's smagsprofiler passer bedst, benyttes deres vudering til at antage at bruger5 ikke vil kunne lide (-) objekt C.
            \citep{kollabrativEksempel}}\label{tabel:kollabrativ}
\end{table}


\subsection{Design}
Vi har valgt at implementeret fire forskellige sorteringer.
Den første er en simpel, upersonlig anbefaling, som blot er de gennemsnitlige bedømmelser, sorteret faldende.
Ligeledes har vi valgt at implementere en personlig anbefaling.
Vi har valgt den indholdsbaserede sortering.
Indholdet her er brugeres vurderinger af opskrifter.
Metoden beregner brugerens forventede vurdering efter deres tidligere vurderinger på opskrifter med samme varer.
Valget af den indholdsbaserede sortering, er truffet fordi vi mener at ingredienserne på hver opskrift kan bruges som attributter til at generere smagsprofiler.
Derudover undgår vi ‘cold-start’-problemet fra den kollaborative anbefaling

Vi har valgt ikke at implementere tf-idf af det indholdsbaserede anbefalingssystem, men i stedet et system inspireret deraf. 

Implementeringen baserer anbefalingerne på opskrifters ingredienser.
Når en bruger har bedømt en opskrift, får hver ingrediens på opskriften tildelt samme bedømmelse.
Hvis en ingrediens er bedømt på mere end en opskrift, bruges gennemsnittet af de alle bedømmelserne for ingrediensen.
Hvis en ingrediens ikke har nogen bedømmelse, bliver denne ingrediens tildelt gennemsnittet af alle brugerenes bedømmelser.
Ud fra denne data oprettes en smagsprofil for hver bruger, som indeholder en værdi mellem 1-5, for hver ingrediens, da brugere har mulighed for at vurdere opskrifter i dette interval.


Derefter sorteres alle opskrifter faldende efter den forventede vurdering, som er gennemsnittet af værdier, fra smagsprofilen, for hver vare på opskriften. 
På figur \myref{tabel:opskriftanbefaling} vises modellens anbefalinger på baggrund af ingredienserne i tidligere bedømte opskrifter.
\begin{table}[H]
  \centering
    \colorlet{shadecolor}{gray!40}
    \rowcolors{1}{white}{shadecolor}
      \begin{tabular}{l|lccccccc}
      %\hline
      \textbf{Opskrifter}   & A        & B       & C       & D       & E   \\ \hline
      Indgrediens1          & +        & +       & +       &         &     \\
      Indgrediens2          &          & +       &         & +       &     \\
      Indgrediens3          &          &         & +       &         & +   \\
      Indgrediens4  		    &          &         &         & +       & +   \\ \hline
      Brugers vudering      & 1        & 5       &         &         &     \\
      Systems forudsigelse  &          &         & 3       & 4       &  3  \\


    \end{tabular}
  \caption{Indholdsbaseret anbefalingsmodel.}\label{tabel:opskriftanbefaling}
\end{table}


I brugergrænsefladen har brugeren mulighed for at vælge hvordan opskrifter skal sorteres.
De kan vælge mellem de to beskrevet her, samt at få sorteret dem efter nyeste og ældste.
På forsiden præsenteres ligeledes de opskrifter som systemet regner højest forventede score ud på.

\subsection{Implementering}
Kernen i denne funktionalitet ligger i metoden \class{ReccomendRecipes} som findes i \class{RecipesController}.
Metoden gennemgår alle opskrifter vurderet af en given bruger, og regner på baggrund af ingredienserne i disse, en forventet vurdering ud, som beskrevet i forrige afsnit.

Metoden \class{ReccomendRecipes} er vist i \myref{recommendrecipes}.
Metoden finder først alle opskrifter vurderet af brugeren: \class{recipiesRatedByUser}, dette er ikke vist i \myref{recommendrecipes}.
En \class{foreach}-løkke gennemløber hver opskrift, en bruger har vurderet, og heri bliver den gennemsnitlige vurdering for hver ingrediens tilføjet til en liste af \class{Tuples}, som er en samling af varen, og alle vurderingerne deraf. 

Dernæst gennemgås en ny \class{foreach}, hvori den forventede vurdering for alle opskrifter af brugeren udregnes, på baggrund af ingredienserne i hver opskrift.
En \class{foreach}-løkke gennemløber alle opskrifter i databasen. 
Hvis en bruger allerede har vurderet den nuværende opskrift, vil denne få vurderingen givet.
Hvis brugeren ikke har vurderet opskriften sker der følgende: Gennemløb alle ingredienser på opskriften, tilføj deres gennemsnitlige vurdering til en liste, hvis denne vare ikke tidligere er vurderet af brugen vil vurderingen være den gennemsnitlige vurdering af alle opskrifter. 
Derefter vil opskriften få en vurdering som er den gennemsnitlige vurdering af alle varerne på opskriften.
Til sidste returneres alle opskrifter sorteret efter systemets forventede vurdering til den givne bruger.

\begin{lstlisting}[caption=Metoden \class{RecommendRecipes}. Sorterer opskrifter efter forventede vurdering for en given bruger., label=recommendrecipes]
public IEnumerable<Recipe> RecommendRecipes(User user)
{
    /* [...] */
    // Pair the ingrdients with the rating given
    // If two ingredients are the same, take the avg of both ratings.
    var itemsWithRatings = new List<Tuple<Item, List<decimal>>>();
    foreach (var recipie in recipiesRatedByUser.ToList())
    {
        var score = recipie.Ratings.FirstOrDefault(x => x.User.Username == user.Username).Score;
        /* [...] */
    }

    var avgRating = recipiesRatedByUser.Average(x => x.Ratings.FirstOrDefault(y => y.User.ID == user.ID).Score);
    /* [...] */
    // Calculate the score for every recipie
    foreach (var recipie in _db.Recipes.Include(x => x.Ingredients).Include(x => x.Ratings))
    {
        if (recipie.Ratings.FirstOrDefault(y => y.User.ID == user.ID) != null)
        {
            recipiesRated.Add(new Tuple<Recipe, decimal>(recipie, recipie.Ratings.FirstOrDefault(y => y.User.ID == user.ID).Score));
        }
        else
        {
            var score = new List<decimal>();
            foreach (var ingredient in recipie.Ingredients)
            {
                var itemrating = (itemsWithRatings.FirstOrDefault(x => x.Item1.Equals(ingredient)));
                score.Add(itemrating != null ? derp.Item2.Average() : avgRating);
            }
            recipiesRated.Add(new Tuple<Recipe, decimal>(item1: recipie, item2: score.Any() ? score.Average() : 0M));
        }
    }
    // Sort the recipies descending and return the recipie from the Tuple
    return recipiesRated.OrderByDescending(x => x.Item2).Select(x => x.Item1);
}
\end{lstlisting}

\subsection{Konklusion}
Den simple anbefaling, der sorterer efter højest gennemsnitlige vurderinger, er simpel men brugbar.
Problemet ved denne er at vi ingen nedre grænse har, for hvor mange bedømmelser der skal være.
Altså vil en opskrift med én vurdering på 5 ligge højere end en opskrift med 49 vurderinger på 5 og én vurdering på 4.
Dette er en tydelig svaghed, som kunne forbedres.
Ligeledes ville en forbedring også være, at hvis to eller flere opskrifter har samme gennemsnitlige vurdering, så ville opskriften med flest vurderinger vægtes højest.

Hvis dette skulle gøres kunne ingredienser eventuelt vægtes efter hvor mange gange en bruger har bedømt denne vare.
Ligeledes kunne vi forestille os at en højere vægtning af sjældent forekommende ingredienser ville kunne præcisere anbefalingerne ved at bruge tf-idf.

Disse forbedringer har vi fravalgt at implementere, da de ikke er en del af kerneområdet i dette projektforløb.
