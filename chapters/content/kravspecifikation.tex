\section{Kravspecifikation}\label{sec:krav}

På baggrund af analysen af problemområdet, anvendelsesområdet og interviewene beskrevet i  \myref{section:interview2} kan der nu opstilles krav til systemet, samt hvad det skal kunne.
Følgende user stories er baseret på brugsmønstrene og funktionerne fra \myref{sec:anvendelses}, samt svarene fra de interviewede.
Som nævnt i \myref{chapter:Metode} benytter vi user stories til at formulere kravspefikationen, da vi benytter Scrum.
\begin{enumerate}
	\item Som en bruger vil jeg kunne oprette indkøbslister.
	\item Som en bruger vil jeg kunne tilføje varer til min(e) indkøbsliste(r).
	\item Som en bruger vil jeg kunne se tilbud.
	\item Som en bruger vil jeg kunne se tilbudsvarer jeg har valgt, deres pris, butik og dato på indkøbslisten.
	\item Som en bruger vil jeg kunne aftjekke en vare fra indkøbslisten.
	\item Som en bruger vil jeg kunne overvåge specifikke vare, og få en notifikation når disse varer kommer på tilbud.
	\item Som en bruger vil jeg kunne finde opskrifter.
	\item Som en bruger vil jeg gerne logge ind.
	\item Som en bruger vil jeg kunne finde opskrifter ud fra anbefalinger til mig.
	\item Som en bruger vil jeg kunne vurdere opskrifter jeg har prøvet.
	\item Som en bruger vil jeg kunne indstille mine præferencer.
	\item Som en bruger vil jeg kunne ekskludere tilbud fra butikskæder som ikke er relevante for mig.
	\item Som en bruger vil jeg kunne dele min indkøbsliste med andre.
	\item Som en bruger vil jeg kunne søge på varer, og finde deres tilbud.
	\item Som en bruger vil jeg kunne tilføje ingredienser for en opskrift til min indkøbsliste.
	\item Som en bruger vil jeg kunne tilgå min indkøbsliste fra min smartphone.
	\item Som en bruger vil jeg kunne skalere opskrifterne til et valgt antal personer.
\end{enumerate}

Disse user stories vil blive designet og implementeret i systemet.
Desuden stilles der yderligere krav til projektet og systemet, bl.a. fra funktionaliteter fundet i \myref{subsec:funktioner}, og fra studieordningen.

\subsection{Krav til systemet}
\begin{enumerate}
\item Der skal benyttes C\# til programmering af systemet.
\item Systemet skal kunne tilgås via forskellige enheder, og gemme information fra enhed til enhed.
\item Systemet skal benytte aktuelle tilbud fra diverse dagligvarebutikker.
\item Systemet skal kunne modtage feedback på de opskrifter brugerne prøver.
\item Systemet skal kunne anbefale opskrifter på baggrund af:
\begin{enumerate}
	\item Madvaner (varieret kost).
	\item Bedømmelse på opskrift.
\end{enumerate}
\item Systemet skal kunne fjerne forslag om eksempelvis kød til vegetarer ud fra præferencer.
\end{enumerate}

\subsection{Krav til UI (brugergrænseflade)}
\begin{enumerate}
	\item Systemets UI skal være på dansk.
	\item Systemet skal kunne anvendes på forskellige enheder
	\item Systemets UI skal være responsivt og tilpasse sig den anvendte platform.
	\item Brugerne skal synes det er nemt at skabe sig overblik over systemet og navigering heri.
\end{enumerate}

Alle kravene i dette afsnit vil blive taget i betragtning, under design og implementering af systemet.
Slutteligt i rapporten konkluderes der på hvor vidt disse krav er opfyldt, og desuden vil der udføres endnu en række interviews med brugere for at teste deres tilfredsstillelse med systemet.
I de følgende afsnit vil udviklingsprocessen yderligere beskrives, samt designet og implementationerne af løsninger til kravene stillet i dette afsnit.
