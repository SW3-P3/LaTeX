\section{Kravspecifikation}\label{sec:krav}

På baggrund af analysen af problemområdet, anvendelsesområdet og interviewene beskrevet i  \myref{section:interview2} kan der nu opstilles krav til systemet, samt hvad det skal kunne.
Følgende user stories er baseret på brugsmønstrene og funktionerne fra \myref{sec:anvendelses}, samt svarene fra de interviewede.
Som nævnt i \myref{chapter:Metode} benytter vi user stories til at formulere kravspecifikationen, da vi benytter Scrum.

\begin{enumerate}
	\item Som en bruger vil jeg kunne oprette indkøbslister.
	\item Som en bruger vil jeg kunne tilføje varer til min(e) indkøbsliste(r).
	\item Som en bruger vil jeg kunne se tilbudsvarer jeg har valgt, deres pris, butik og dato på indkøbslisten.
	\item Som en bruger vil jeg kunne aftjekke en vare fra indkøbslisten.
	\item Som en bruger vil jeg kunne dele min indkøbsliste med andre.
	\item Som en bruger vil jeg kunne tilgå min indkøbsliste fra min smartphone.
			
	\item Som en bruger vil jeg kunne se tilbud.
	\item Som en bruger vil jeg kunne søge på varer, og finde deres tilbud. 
	
	\item Som en bruger vil jeg kunne overvåge specifikke varer, og få en notifikation når disse varer kommer på tilbud.
	
	\item Som en bruger vil jeg gerne logge ind.
	\item Som en bruger vil jeg kunne fravælge madvarer.
	\item Som en bruger vil jeg kunne ekskludere tilbud fra butikskæder som ikke er relevante for mig.
	
	\item Som en bruger vil jeg kunne finde opskrifter.
	\item Som en bruger vil jeg kunne finde opskrifter ud fra anbefalinger til mig.
	\item Som en bruger vil jeg kunne vurdere opskrifter jeg har prøvet.
	\item Som en bruger vil jeg kunne tilføje ingredienser for en opskrift til min indkøbsliste.
	\item Som en bruger vil jeg kunne skalere opskrifterne til et valgt antal personer.
\end{enumerate}
Gruppen mener dog også at følgende user story ville være relevant: 
\begin{enumerate}
\setcounter{enumi}{17}
	\item Som en bruger vil jeg kunne lave min egen variation af en opskrift.
\end{enumerate}

Disse user stories vil blive designet og implementeret i systemet.

\subsection{Krav til systemet}

Der stilles yderligere krav til projektet og systemet, bl.a. fra funktionaliteter fundet i \myref{subsec:funktioner}, og fra studieordningen.
\begin{enumerate}
\item Der skal benyttes C\# til programmering af systemet.
\item Systemet skal kunne tilgås via forskellige enheder, og gemme information fra enhed til enhed.
\item Systemet skal benytte aktuelle tilbud fra diverse dagligvarebutikker.
\end{enumerate}

Alle kravene i dette afsnit vil blive taget i betragtning, under design og implementering af systemet.
Slutteligt i rapporten konkluderes der på hvor vidt disse krav er opfyldt, og desuden vil der udføres brugertests med dertilhørende spørgsmål om produktet for at teste systemets brugervenlighed og funktionalitet, samt brugernes tilfredsstillelse med systemet.
