\chapter{something something sum-up} \fxnote{Generelt for dette, where's the soruces?}
Vi er igennem forløbet af projektet stødt på nogle uventede problemer og hændelser, som har ændret vores produkt derefter.\fxnote{Hvilke problemer, og hvad har det betydet? (Sass) EF many-to-many? Data kvalitet?}

Vi mener, at vi har opfyldt alle semesterkravene, der er sat, og særligt har vi lagt vægt på én ting;
hvilket er at skabe et så virkelighedstro forløb som muligt, hvor der skal tages kontakt med eventuelle brugere af produktet.
Vi har haft en prototypetest, og derefter to brugertests, som alle har givet os brugbart feedback, så systemet kunne forbedres og videreudvikles.
Det ville have været fordelagtigt at rykke det en uge bagud, således vi kunne have nået en test mere.
Denne test ville have til formål at lave plads til en hel uge kun med fejlretning på programmet.
Vi mener dette ville have givet et bedre overblik over fejl og mangler, og ændre på små ting, der ville have manglet.
Vi har så vidt muligt prøvet at indsamle tests fra samme personer, for at se om vi kunne opfylde deres krav, men kun få af vores testpersoner har været med igennem hele forløbet.
Dette mener vi ville have givet et bedre overblik over hvorvidt vi har udbedret de rigtige ting. \fxnote{Denne sætning kunne måske omformuleres til noget i retningen af om programmet har forbedret sig igennem iterationerne?}
Vi mærkede ved de personer, der gik igen i den afsluttende brugertest, at det var en stor fordel. \fxnote{``Vi mærkede'' virker meget derp, måske noget i retning af at brugerne gav udtryk for ...}

Et af de krav, der var lagt meget vægt på, var at vores produkt skulle foretage en form for overvågning af data. \fxnote{Uddyb evt. hvorfra kravene er}
Dette opfylder vores produkt også i form af, at brugeren kan overvåge varer, der er på tilbud, samt anbefale nye opskrifter baseret på de vurderinger brugeren giver forskellige opskrifter.
Programmet ændres også hele tiden baseret på hvilket tilbud vi har til rådighed.\fxnote{uddyb...}
Vi mener dog selv, at det ville have været bedre, hvis vi havde haft noget mere overvågning.
Grunden til, at det ikke skete, var at fokus slet ikke lå på det fra vores brugere og det var derfor meget svært at få ind over.\fxnote{Hvad?}
En overvågning ville have kunne skabt et mere avanceret program, som vi har diskuteret manglede lidt.\fxnote{Måske vi ikke skal lægge ord i munden på dem til eksamen}

Vi startede ud med at følge den iterative fremgang, men efter et par uger kunne gruppen ikke finde fokus og overblik.
Vi begyndte på SCRUM\fxnote{Vi har jo ikke brugt hele SCRUM, måske skulle skrives at vi har lånt noget fra SCRUM som forklaret i metode?}, som også er beskrevet tidligere i rapporten, fordi vi mistede overblikket helt og aldeles.
SCRUM har hjulpet os rigtig meget og givet os et rigtigt godt overblik, samt forsynet os med en rigtig god fremgangsmåde for projektarbejdet.
Det ville dog have været meget bedre for gruppen at have kendt til metoden noget før.
Vi brugte et par dage på at få sat det hele op med SCRUM og derefter gik det meget bedre med arbejdet i gruppen.
Det ville have været meget bedre at have metoden i brug fra start af, dette vil have resulteret i et bedre produkt og en bedre stemning i gruppen.
Havde vi brugt det fra starten af tror vi, at vi ville have et større overskud til sidst i forløbet, og måske kunne have haft flere features i vores produkt.\fxnote{proces beskrivelse?}
