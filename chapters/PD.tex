\chapter{Problemområde}

Systemet har forskellige emner der skal modelleres, derfor skal problemområdet analyseres.
Problemområdet defineres ifølge \citep{OOA&D2001} som:

\textit{``Den del af omgivelserne, der administreres, overvåges eller styres ved hjælp af et system.''}

Systemet skal holde styr på tilbud, generelle varer, og opskrifter. 
Der skal registreres hvis en varer er på tilbud, hvor denne vare er på tilbud, og om brugeren ønsker at købe varen.
Opskrifterne har en liste over ingredienser, som består af varer. 
Opskrifterne skal også kunne blive vurderet, så systemet kan anbefale lignende opskrifter til brugeren.
Varer skal også kunne overvåges, og systemet skal give besked til brugeren når et tilbud på varen dukker op.
Systemet skal kunne medbringes på indkøbsturen, og her skal tilbudene og varerne fra systemet kunne indgå på en indkøbsliste.

Med disse informationer kan systemet hjælpe brugeren til at finde billige varer i bestemte butikker, og samtidig anbefale opskrifter der bruger disse tilbudsvarer.
I de følgende afsnit vil disse emner blive beskrevet vha. klassebeskrivelser, en hændelsestabel, og et klassediagram.

\section{Struktur}
For at skabe et overblik over hvilke objekter der findes i problemområdet, samt hvilke klasser disse tilhører og at til sidst kunne kæde dette sammen med hændelser i problemområdet, foretages først en analyse af strukturen.\\

%__klassse-diagram indsættes her - laves ud fra drive/Billeder/2014-10-06 14.52.48.jpg___

Dette er beskrevet vha. et klassediagram som ses i \fxfatal{figursomething}, dette beskriver forholdet mellem de forskellige objekter som findes i problemområdet til indkøb og madlavning. Den følgende tekst uddyber figuren.

Personer i indkøbssituationer kan lave indkøbslister, disse indkøbslister kan været ejet og administreret af en enten en eller flere personer. 
En indkøbsliste kan bestå an ingen til en højt antal varer.
En vare kan komme fra tre forskellige steder.
Varer kan stamme fra et tilbud, eller de kan være ingredienser fra en opskrift. 
Hvis en vare ikke er nogle af disse ting, er varen blot generisk, det vil sige en vare uden forhold til et tilbud eller en opskrift.
Et tilbud består som tidligere nævnt af en vare.
Et tilbud har nogle attributter der beskriver varen, samt pris og lignende.
En avis består af minimum et tilbud men har oftest en samling af tilbud, og en sådan avis har et tilhørsforhold der består af en butik, som udgiver avisen med tilbudene i.\\
Som tidligere nævnt kan en vare også udgøre en ingrediens i en opskrift.
En opskrift består af en eller flere ingredienser, beskrevet som varer.
En person i problemområdet kan, vurderer den en opskrift, for at beskrive hvor tilfredsstillende de fandt opskriften.
En person vil maksimalt kunne give hver opskrift en vurdering, men kan også undlade at vurdere en opskrift.

\section{Klasser}
So who make dis?

\section{Hændelser}
Vi har analyseret de forskellige hændelser der sker i problemområdet. \fxnote{Hvor har vi gjort det :D ?} 
Ud fra disse hændelser vil vi lave en hændelsestabel, der beskriver hvilke klasser forskellige hændelser påvirker.
Formålet med at identificere hændelserne samt at analysere disse i en hændelsestalbel, er at forstå problemområdet bedre og dermed hjælpe med forståelsen for hvordan en løsning ville kunne udarbejdes for at afhjælpe de problemer der findes i problemområdet. Desuden kan tabellen hjælpe med strukturen på klasserne. 
Hvis 2 klasser har alle de samme hændelser, kan der ofte foretages ændringer og dermed opnå en bedre struktur.

\begin{table}[H]
  \centering
    \colorlet{shadecolor}{gray!40}
    \rowcolors{1}{white}{shadecolor}
      \begin{tabular}{l|lccccccc}
      %\hline
       								& \rot{Tilbud}  & \rot{Indkøbsliste} & \rot{Opskrift} & \rot{Vare} & \rot{Person}& \rot{Vurderinger} \\ \hline
      Vare tilføjet til indkøbsliste&               & +      &          & +     & +     &   \\ 
      Vare fjernet fra indkøbsliste	&              	& +      &          & +     & +     &   \\ 
      Vare aftjekket på indkøbsliste&               & +      &          & +     & +     &   \\ 
      Opskrift valgt ???       		& +             & +      &          & +     & +     &   \\ 
      Tilbud oprettet        		& +            	& +      & +        & +     &       &   \\ 
      Tilbud aktiveret        		& +            	& +      & +        & +     &       &   \\ 
      Tilbud udgået          		& +        		& +      & +     	&       &       &   \\ 
      Vare tilføjet til overvågning & +          	&        &          & +     & +     &   \\ 
      Vare fjernet fra overvågning  & +          	&        &          & +     & +     &   \\ 
      Overvågningsvare på tilbud    & +  			&		 &			& + 	& +		&	\\
      Del indkøbsliste       		&               & +      &          &       & +     &   \\ 
      Indkøbsliste oprettet  		&              	& +      &          &       & +     &   \\ 
      Indkøbsliste slettet  		&             	& +      &          &       & +     &   \\ 
      Vurdering givet				&             	&        & +        &       & +		& + \\
      Anbefaling givet				&				&		 & +		&		& +		& + \\
      
    \end{tabular}
  \caption{Hændelsestabel. Viser hvilket klasser, problemområdets hændelser påvirker.}\label{tabel:haendelsestabel}
\end{table}

\fixme{make Troelsstyle-table}

Hændelsestabellen i figur SOMETHING\fxfatal{Derp} viser både hvilke hændelser der findes i problemområdet, samt hvilke klasser de påvirker.
Hvis tabellen læses vandret kan det ses at klasser der bliver påvirket af mange hændelser er klasser som "Indkøbsliste", "Vare" og "Bruger".


Ud fra denne analyse har vi et overblik og hvilke klasser der hører til hvilke hændelser.
Dette indblik kan vi bruge når vi henholdsvis skal formulere en systemdefinition, fortsatte kravspecifikation samt modellere vores løsning.\fxnote{Hvilket indblik? Der skal måske forklares hvilket indblik det helt konkret har givet, et eksempel på hvad det f.eks. betyder for projektet eller noget.} 