The purpose of this paper is to document the development and choices made by the project-group, in order to produce a product in relation to assisting users in their day-to-day grocery shopping. 
Parts of scrum have been used to organize and assist in the development, and the method of object oriented analysis and design has been used to analyse and define the problem. 

Firstly the potential users are polled about their habits with regards to the use of technology, such as smart-phone apps, in relation to grocery shopping.
Additionally the state of the art is researched, which then led to creating a series of user stories which guides the development of the final product.

The application is made using ASP.NET MVC 5, with Bootstrap, jQuery and Entity Framework. 
The application loads real world offers from an API, which it integrates with shopping-lists, a personal watch-list, recipes and user preferences. 
There have been made two user-tests one mid development and one after as an evaluation in order to get user feedback.
To further assure code quality a test project has been made, containing unit-tests of relevant code.

Finally there are suggestions to further improve the website, based on the user feedback as well as the project-group's own suggestions.

