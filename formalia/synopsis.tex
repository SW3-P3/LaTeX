% Abstract indeholder beskrivelse af opgaven, formål/problemstilling, anvendte metoder, resultater
% og konklusioner.

%%%%%%% FORSLAG 1 %%%%%%%
%The purpose of this report is to document the development and choices made by the project-group in order to produce a product in relation to assisting users in their day-to-day grocery shopping. 
%Parts of scrum have been used to organize and assist in the development, and the method of object oriented analysis and design have been used to define the problem. 
%
%Firstly the current market is researched, and potential users are polled about their habits with regards to use technology, such as smart-phone apps, in relation to grocery shopping.
%Additionally the state of the art is researched, which led then to a series of user-stories are made which will guide the development of the final product.
%
%The application is made using ASP.NET MVC 5, with Twitter Bootstrap, jQuery and Entity Framework as a database. 
%The application loads real world data from an API, which it integrates with shopping-lists, a personal watch-list, recipes and user preferences. 
%There have been made 2 user-tests one mid development and one after as an evaluation in order to get user feedback.
%To further assure code quality there have been made unit-tests with a code coverage of over 80\% in the relevant code. 
%
%Finally there are suggestions to further improve the website, based of the user feedback as-well as the project-group.
%
%%%%%% FORSLAG 2 %%%%%%%%%
This project report describes the development of an application capable of assisting users in their daily grocery shopping. 

The application allows the users to make and share shopping-lists with each other, adding general items as-well as real-world offers acquired via an API.
Users can browse recipes in the system and add them to their shopping-list, the application can sort recipes by the expected rating given, by taking the users ratings given to recipes with the same ingredients.
Users can add items to a watch-list from where they can browse offers of said items as well as receiving e-mail notifications when new offers are added to stores. 
The primary features can be accessed via mobile browsers with a mobile experience. 

There are two user-tests one mid development, and one after as an evaluation, in order to get user feedback.
To further assure code quality there have been made unit-tests with a code coverage of about 80\% in the relevant code. 

The product was made using iterative development and parts of Scrum.
The implementation is made using C\#, with the ASP.NET MVC 5 software stack including Twitter Bootstrap and jQuery for front-end, and Entity Framework for data persistence. 
eTilbudsavis' API have been used to provide real-world data.
