\begin{table}[H]
  \centering
    \colorlet{shadecolor}{gray!40}
    \rowcolors{1}{white}{shadecolor}
      \begin{tabular}{l|lccccccc}
      %\hline
       								& \rot{Tilbud}  & \rot{Indkøbsliste} & \rot{Opskrift} & \rot{Vare} & \rot{Person}& \rot{Vurderinger} \\ \hline
      Vare tilføjet til indkøbsliste&               & *      &          & *     &       &   \\
      Vare fjernet fra indkøbsliste	&              	& *      &          & *     &       &   \\
      Vare aftjekket på indkøbsliste&               & *      &          & *     &       &   \\
      Tilbud oprettet        		& +            	&        &          & *     &       &   \\
      Tilbud aktiveret        		& +            	&        &          & *     &       &   \\
      Tilbud udgået          		& +        		&        &      	& *     &       &   \\
      Vare tilføjet til overvågning &           	&        &          & *     &       &   \\
      Vare fjernet fra overvågning  &           	&        &          & *     &       &   \\
      Overvågningsvare på tilbud    & *  			&		 &			& * 	& *		&	\\
      Del indkøbsliste       		&               & *      &          &       & *     &   \\
      Indkøbsliste oprettet  		&              	& +      &          &       & *     &   \\
      Indkøbsliste slettet  		&             	& +      &          &       & *     &   \\
      Forlad Indkøbsliste			&				& *		 & 			&		& *		&   \\
      Opskrift oprettet				&				&		 & +		& *  	& *		& 	\\
      Opskrift slettet				&				&		 & +		& *  	& *		& 	\\		
      Vurdering givet				&             	&        & +        &       & *		& * \\
      Anbefaling givet				&				&		 & +		&		& *		& * \\

    \end{tabular}
  \caption{Hændelsestabel. Viser hvilket klasser, problemområdets hændelser påvirker
  + en hændelse forekommer højest en gang i et hændelsesforløb.
  * en hændelse der kan forekomme flere gange i et hændelsesforløb.\citep{OOA&D2001}
  }\label{tabel:haendelsestabel}
\end{table}
