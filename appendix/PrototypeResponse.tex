\chapter{Prototype interviews respons}\label{ch:protorespons}
Vi har interviewet seks forskellige personer, hvor to af dem tog testen sammen.
{[}1{]} - Kvinde på 48 der bor i en husstand med fire, står for indkøb og madlavning.

{[}2{]} - Par bestående af 22 årig mand og 21 årig kvinde der bor sammen.

{[}3{]} - Kvinde på 21 der bor sammen med sin kæreste.

{[}4{]} - Mand på 22 der bor alene (prøvede ikke prototypen - interview over skype).

{[}5{]} - Mand på 21 der bor sammen med to venner.

Systemet hjælper på problemer, der kan opstå i forbindelse med indkøb og madlavning i hjemmet.

\textbf{Hvilke opgaver i denne process skal sådan et system kunne hjælpe med, eller løse for at være nyttigt for dig?}
\begin{itemize}
	\item Indkøbslister med afkrydsning - således man ikke glemmer hvad der ligger i kurven.{[}1{]}{[}2{]}
	\item Filtrering således der kan ses hvad der kan laves, med fx oksekød eller hvis man er kræsen. {[}2{]} 
	\item Mulighed for at finde nærmeste butikker.{[}3{]}{[}5{]}
	\item Hvad er der på tilbud nu, og hvad er der på tilbud i næste uge.{[}5{]}
	\item Mulighed for at lave en madplan / menu for et selskab.{[}1{]}{[}2{]}{[}5{]}
	\item Foreslå varieret kost . {[}3{]}
	\item Mulighed for at kunne “Tømme køleskabet”.{[}2{]}{[}5{]}
	\item Push-beskeder om overvågning.{[}2{]}{[}5{]}
\end{itemize}

\textbf{Hvorfor valgte du netop de svar fra første spørgsmål som du gjorde? Var det, det første
du tænkte på? Var det et problem du ofte har?}
\begin{itemize}
	\item Problemer man har, primært hvad man skal have og spise.{[}2{]}{[}3{]}
\end{itemize}

\textbf{Finder du de funktioner vi har præsenteret, som du ikke har nævnt relevante, eller forestiller du dig at de ville være unødvendige?}
\begin{itemize}
	\item Tilbuds overvågning er der stor opbakning om.{[}2{]}{[}3{]}{[}4{]}{[}5{]}
	\item Tilgang til tilbud fra alle butikker er godt.{[}1{]}{[}2{]}{[}3{]}{[}5{]}
	\item Fakta app spammede med opskrifts idér og foreslog for meget det samme.{[}3{]}
	\item Opskrifte rating er smart - udforske nye muligheder.{[}2{]}{[}5{]}
	\item Mulighed for at kunne følge en kostplan - deriblandt kalorietæller{[}4{]}.
\end{itemize}

Stor enighed om tilgang på telefon, tablet og computer. {[}1{]}{[}2{]}{[}3{]}{[}4{]}{[}5{]}
Nævnt at en printnings feature af listen ville være smart, så ældre kunne have den med uden en smartphone.{[}3{]}

\textbf{Prototype feedback}

\textbf{Godt:}
		\textbf{Forside:}
	\begin{itemize}
		\item Oversigt over hvor man kan gå hen.{[}3{]}
	\end{itemize}

		\textbf{Opskrifter:}
	\begin{itemize}
		\item Tidsangivelse er godt (Brug af et ur er blevet komplimenteret).{[}1{]}{[}3{]}
		\item Ingredienser skaber overblik og nysgerrighed.{[}2{]}
	\end{itemize}
		\textbf{Tilbud:}
	\begin{itemize}
		\item Pris og mængde på selve tilbuddet.{[}5{]}
	\end{itemize}
		\textbf{Præferencer:}
	\begin{itemize}
		\item Butikskæder er en god ting at have.{[}1{]}{[}2{]}{[}5{]}
		\item Hensyn til allergier.{[}5{]}
	\end{itemize}
		\textbf{Indkøbsliste:}
	\begin{itemize}
		\item Godt man kan blande generiske vare og tilbud.{[}1{]}{[}2{]}
		\item Layout ser godt ud - varenavn, pris, logo.{[}2{]}{[}3{]}{[}5{]}
	\end{itemize}
		\textbf{Overvågning:}
	\begin{itemize}
		\item Smart at have da nogle vare sjæledent er på tilbud (såsom bleer){[}1{]}{[}2{]}{[}5{]}
	\end{itemize}
\textbf{Forslag:}

		\textbf{Forside:}
	\begin{itemize}
		\item Mangler overvåget vare.{[}3{]}
		\item Personlig oplevelse / Tilpasset til den enkelte bruger. {[}2{]}{[}3{]}
		\item Vante søgninger. {[}2{]}
		\item Hvad har man sidste søgt på. {[}2{]}
		\item Ser meget klumpet og uoverskuelig ud.{[}3{]}
	\end{itemize}
		\textbf{Opskrifter:}
	\begin{itemize}
		\item Personantal.{[}1{]}
		\item Tøm køleskabet feature.{[}2{]}{[}3{]}{[}5{]}
		\item Populære / højest rated. {[}2{]}{[}5{]}
		\item Udvid afkrydsning til steges bages, ovn, pande osv.{[}2{]}
		\item Kommentarfelt til de enkelte opskrifter.{[}2{]}
		\item How-to videoer.{[}2{]}
	\end{itemize}
		\textbf{Tilbud:}
	\begin{itemize}
		\item Førpris på hver vare.{[}5{]}
		\item Vis kun de billigste vare (kylling til 30 kr/kg i fakta, fjern kylling til 45kr/kg i netto). {[}3{]}
		\item Sortering, pris, navn osv.{[}2{]}
		\item Fokusering på brugeren, hvad har man søgt på og hvad har man tilføjet til ens liste.{[}2{]}
		\item Der ønskes ingen tilbudsaviser, nu når der kan søges.{[}2{]}{[}5{]} 
		\item Afstand til de enkelte butikker der har tilbudene.{[}2{]}{[}5{]}
		\item Mulighed for at fjerne fristelser - fjern forslået tilbud der ikke ønskes.{[}2{]}
	\end{itemize}
		\textbf{Præferencer:}
	\begin{itemize}
		\item Mangler flere sorterings muligheder end bare kød. Fx Økologi og veganer.{[}2{]}
		\item Både tilvalg og fravalg.{[}2{]}{[}5{]}
	\end{itemize}
		\textbf{Indkøbsliste:}
	\begin{itemize}
		\item Foreslået til at være det første man så. - bestemt på mobilenheder.{[}2{]}
		\item Afstand til tilbudene på listen.{[}2{]}{[}5{]}
		\item Vis alle butikker sagte vare er på tilbud i. {[}2{]}{[}5{]}
		\item Få den tilsendt som sms så man slipper for at skulle gå på nettet.{[}4 - ikke en del af prototype test{]}
	\end{itemize}
		\textbf{Overvågning:}
	\begin{itemize}
		\item Brug overvågningen til at foreslå andre vare af samme kaliber - overvåg kyllingebryst, og få besked og kyllingeinderfilet.{[}5{]}
	\end{itemize}