\chapter{Bruger Test}\label{b:brugertest1}

Vi har tidligere udført en test af vores hjemmeside, men udører nu endnu en for at få yderligere respons på hjemmesidens funktionalitet og design.
Du vil blive stillet nogle opgaver hvor du skal udføre forskellige aktiviteter på hjemmesiden.
Det er hjemmesiden der testes og altså ikke dig som testudfører.
Dvs. er der noget du ikke forstår hvordan man gør, er det hjemmesidens design der er sårligt.
Din opgave er, som udefrakommende at teste hvor brugervenligt systemet er. 
Du bedes gennemlæse hver opgave fuldstændigt og fortælle grupperepræsentanten hvad du forstår du skal i opgaven.
Hvis der er eventuelle uklarheder i opgaven, vil han udrede disse.
Vi vil gerne bede dig tænkt højt, så vi forstår hvorfor du gør hvad du gør når du arbejder med hjemmesiden.
Dvs. hvis du klikker på en knap, vil vi gerne vide hvorfor du klikker på denne knap.
Når du mener at du har afsluttet opgaven bedes du erklære det til grupperepræsentanten.

Efter opgaverne vil vi gerne stille dig en række spørgsmål ang. brugen af hjemmesiden.

\section{Opgaver}

Opgave 1.

Du har problemer i hverdagen med at holde styr på dine tilbud, og overskue ugens tilbud, og vil derfor lave en bruger på ProjectFood.
Du handler ikke i alle butikker i byen, og vil derfor afkrydse hvilke butikker du handler i, så du ikke kan se tilbud fra alle byens butikker.

\textbf{Opret en bruger og fravælg butikker du ikke handler i, og evt. varer du ikke har lyst til at købe. }

Opgave 2.

Du skal forberede indkøbene til i morgen, og vil derfor lave en indkøbsliste for dagen i morgen imens du finder nogle tilbud på de ting du mangler.

\textbf{Udfyld din indkøbsliste med varer og find derefter tilbud for nogle af varerne på indkøbslisten.}

Du ændrer mening, og beslutter at du skal have dobbelt så meget/mange, af den vare du tilføjede som den første til indkøbslisten.

\textbf{Ændre mængden til det dobbelte på den først tilføjede vare.}

Du synes din roommate skal have adgang til indkøbslisten, og beslutter dig for at dele denne med ham/hende.

\textbf{Del din indkøbsliste med brugeren: test@mail.dk}

Opgave 3.
Du er interesseret i at se hvilke tilbud der er denne uge, og kigger derfor på tilbudene fra de forskellige butikker, og vælger nogle derfra.

\textbf{Browse ugens tilbud, og vælg nogle du er interesseret i og tilføj dem til din valgte indkøbsliste.}

Opgave 4.

Du får besøg af din familie i weekenden, og tænker at det ville være smart at lave pulled pork.

\textbf{Find en opskrift på pulled pork til 8 personer, tilføj de varer du skal bruge fra opskriften til din valgte indkøbsliste.}

Opgave 5.

Du har nu lavet opskriften og var begejstret for denne. Derfor vælger du at tildele en vurdering til opskriften.
Du mente dog også at opskriften kunne bruge lidt ekstra chili for at krydre den lidt ekstra.

\textbf{Tildel opskriften en vurdering, og lav desuden din egen variation af opskriften med ekstra chili.}

Du mangler inspiration til din aftensmad, find en anden opskrift vha. hjemmesidens funktionaliteter, som du tror du ville kunne lide.

\textbf{Find en anden opskrift vha, hjemmesidens funktionaliteter som du tror du vile kunne lide.}
\fxnote{Skal vi evt. skrive sortering her?}

Opgave 6.

Du vil gerne have besked om når en bestemt vare kommer på tilbud. Du tilføjer derfor el eller flere vare til din overvågningsliste.

\textbf{Tilføj en eller flere varer til din overvågningsliste.}
Hvis du har mulighed for at tjekke den e-mail du oprettede brugeren med må du gerne gøre dette, helst via din telefon. Du burde have modtaget en e-mail fra hjemmesiden med varerne.\fxnote{Hvordan skal dette lige koordineres?? Skal de trykke på en knap lige nu for at få mailen eller hvordan ? og så forklarer hvordan det ville fungerer når hjemmesiden blev deployed?}

Opgave 7.

Du tager din telefon med ud og handler ind efter din indkøbsliste. Grupperepræsentanten sætter hjemmesiden i mobiltilstand for dig. Forestil dig at du er ude og handle og afkryds varerne som du lægger dem i din kurv.

\textbf{Afkryds varerne på din indkøbsliste i mobiltilstand}

\textbf{Browse derudover rundt på hjemmesiden i mobiltilstand og fortæl hvad du synes om designet på mobilen.}


\section{Opfølgende spørgsmål:}

Første gang vi udførte tests fandt testpersonerne følgende problemstillinger:

\begin{itemize}
\item Det er forvirrende om tilbudene i tilbudsfanen er aktive tilbud eller ej.
\item Hvordan ser man detaljerne om en opskrift?
\item Det tager lang tid at tilføje ingredienserne til sin indkøbsliste.
\item Se tilbud på ingredienslisten \fxnote{ville vi fikse dette??}
\item Svært at vide om indstillingerne vedr. butikker og varer man ikke vil have bliver gemt.
\item Afkrydsning af en vare på indkøbslisten vidste man ikke hvordan man gjorde.
\item Overså det at vælge indkøbslisten i de forskellige vinduer.

\end{itemize}

\begin{itemize}
	\item Spørg ud for hvert emne:
		\begin{itemize}
			\item Forside
			\item Tilbud
			\item Opskrifter
			\item Indkøbslister
			\item Indstillinger
			\item Overvågningslisten
		\end{itemize}
	Var der hvad du forventede under dette emne? - Hvad fungerede godt, hvad fungerede ikke godt?
	\item Hvordan synes du det var at bruge hjemmesiden?
	\item Kunne du finde på at benytte dig af hjemmesiden, hvis den blev udgivet? - Hvorfor, hvorfor ikke?
	\item Var der noget ved hjemmesidens funktionalitet eller udseende som du ikke forstod?
	\item Ville du kunne finde på at bruge hjemmesidens anbefalingsfunktionalitet mht. opskrifter?
	\item Giver forsiden et godt overblik? Mangler der noget information herinde? Var du forvirret over hvor du skulle gå hen da forsiden åbnede første gang?
	
	
	\item Er der nogle ting du føler kunne gøres bedre?
	\item Hvor nemme var opgaverne?
	\item Var opgaverne realistiske?
	\item Var det forstyrrende at du skulle tænke højt?
\end{itemize}	

Tak fordi du ville deltage i testen, det har været meget behjælpelig.

