\chapter{Bruger Test 2}\label{b:brugertest2}

Vi har tidligere udført en test af vores hjemmeside, men udfører nu endnu en for at få yderligere respons på hjemmesidens funktionalitet og design.
Du vil blive stillet nogle opgaver hvor du skal udføre forskellige aktiviteter på hjemmesiden.
Det er hjemmesiden der testes og altså ikke dig som testudfører.
Altså er der noget du ikke forstår hvordan man gør, er det hjemmesidens design der er dårligt.
Din opgave er, som udefrakommende at teste, hvor brugervenligt systemet er. 
Du bedes gennemlæse hver opgave fuldstændigt og fortælle testlederen hvad du forstår du skal i opgaven, før du forsøger at udføre den.
Hvis der er eventuelle uklarheder i opgaven, vil testlederen udrede disse.
Vi vil gerne bede dig tænke højt, så vi forstår, hvorfor du gør som du gør, når du arbejder med hjemmesiden.
Dvs. hvis du klikker på en knap, vil vi gerne vide hvorfor du klikker netop på denne knap.
Når du mener, at du har afsluttet opgaven, bedes du erklære det til testlederen.

Efter opgaverne vil vi gerne stille dig en række spørgsmål angående brugen af hjemmesiden.

\section{Opgaver}

Opgave 1.

Du har problemer i hverdagen med at holde styr på dine tilbud, og overskue ugens tilbud, og vil derfor lave en bruger på ProjectFood.
Du handler ikke i alle butikker i byen, og vil derfor afkrydse hvilke butikker du handler i, så du ikke kan se tilbud fra alle byens butikker.

\textbf{Opret en bruger og fravælg butikker du ikke handler i, og evt. varer du ikke har lyst til at købe. }

Opgave 2.

Du skal forberede indkøbene til i morgen, og vil derfor lave en indkøbsliste for dagen i morgen imens du finder nogle tilbud på de ting du mangler.

\textbf{Udfyld din indkøbsliste med varer og find derefter tilbud for nogle af varerne på indkøbslisten.}

Du ændrer mening og beslutter, at du skal have dobbelt så meget/mange af den vare, som du tilføjede som den første til indkøbslisten.

\textbf{Ændre mængden til det dobbelte på den først tilføjede vare.}

Du synes din roommate skal have adgang til indkøbslisten, og beslutter dig for at dele denne med ham/hende.

\textbf{Del din indkøbsliste med brugeren: test@mail.dk}

Opgave 3.
Du er interesseret i at se hvilke tilbud der er denne uge, og kigger derfor på tilbudene fra de forskellige butikker, og vælger nogle derfra.

\textbf{Gennemse ugens tilbud og vælg nogle, du er interesseret i og tilføj dem til din valgte indkøbsliste.}

Opgave 4.

Du får besøg af din familie i weekenden og tænker, at det ville være smart at lave pulled pork.

\textbf{Find en opskrift på pulled pork til 8 personer og tilføj de varer du skal bruge fra opskriften til din valgte indkøbsliste.}

Opgave 5.

Du har nu lavet opskriften og var begejstret for denne. 
- Du vælger at tildele en vurdering til opskriften.
Du mente dog også, at opskriften kunne bruge chili for at krydre den yderligere.

\textbf{Tildel opskriften en vurdering og lav desuden din egen variation af opskriften med chili.}

Du mangler inspiration til din aftensmad, find en anden opskrift som andre brugere har vurderet højt.

\textbf{Find en anden opskrift vha, hjemmesidens funktionaliteter som du tror du ville kunne lide.}

Opgave 6.

Du vil gerne have besked når en bestemt vare kommer på tilbud. 
Du tilføjer derfor en eller flere vare(r) til din overvågningsliste.

\textbf{Tilføj en eller flere vare(r) til din overvågningsliste.}

Naviger nu ud til forsiden, her kan du se tilbud på de vare(r) du har tilføjet, hvis systemet har fundet nogle.

\textbf{Naviger ud på forsiden og se hvordan den ser ud nu.}


Opgave 7.

Du tager din telefon med ud og handler ind efter din indkøbsliste. 
Testlederen giver dig mulighed for at tilgå mobillayoutet af hjemmesiden. 
Du er ude og handle. 
- Afkryds varerne, som du lægger dem i din kurv.


\textbf{Find din indkøbsliste og derefter afkryds de varer, som du køber.}

Opgave 8.

Du er kommet hjem fra indkøbsturen, og vil bruge din telefon til at følge din egen variation af Pulled pork opskriften.

\textbf{Naviger ind på opskrifter. Find din egen variation af Pulled pork, og vurder mobiloplevelsen i forbindelse med at følge en opskrift. }


\section{Opfølgende spørgsmål:}

Første gang vi udførte tests fandt testpersonerne følgende problemstillinger:

\begin{itemize}
	\item Det er forvirrende om tilbudene i tilbudsfanen er aktive tilbud eller ej.
	\item Hvordan ser man detaljerne om en opskrift?
	\item Det tager lang tid at tilføje ingredienserne til sin indkøbsliste.
	\item Det er svært at vide om indstillingerne vedr. butikker og varer, man ikke vil have, bliver gemt.
	\item Det var uklart hvordan det var muligt at afkrydse varer på sin indkøbsliste.
	\item Overså mulighederne for at vælge indkøbslisten, som man tilføjer varer/tilbud til, i de forskellige vinduer.
\end{itemize}

\begin{itemize}
	\item Spørg ud for hvert emne:
	\begin{itemize}
		\item Forside
		\item Tilbud
		\item Opskrifter
		\item Indkøbslister
		\item Indstillinger
		\item Overvågningslisten
	\end{itemize}
	Var der hvad du forventede under dette emne? - Hvad fungerede godt, hvad fungerede ikke godt?
	\item Hvordan synes du det var at bruge hjemmesiden?
	\item Kunne du finde på at benytte dig af hjemmesiden, hvis den blev udgivet? - Hvorfor, hvorfor ikke?
	\item Var der noget ved hjemmesidens funktionalitet eller udseende som du ikke forstod?
	\item Ville du kunne finde på at bruge hjemmesidens anbefalingsfunktionalitet mht. opskrifter?
	\item Hvad syndes du om tekstmængden på de forskellige sider.
	\item Kunne du forestille dig at bruge mobilversionen i andre scenarier end ved madlavning, inspirationssøgning eller indkøbsturen.
	
	\item Giver forsiden et godt overblik? Mangler der noget information herinde? Var du forvirret over hvor du skulle gå hen da forsiden åbnede første gang?
	\item Påvirkede det dig, at der var færre funktioner tilgængeligt på mobil versionen. - Hvordan påvirkede det dig? (Positivt/Negativt)
	\item Er der nogle ting du føler kunne gøres bedre?
	\item Hvor nemme var opgaverne?
	\item Var opgaverne realistiske?
	\item Var det forstyrrende at du skulle tænke højt?
\end{itemize}	

Tak fordi du ville deltage i testen, det har været meget behjælpelig.
