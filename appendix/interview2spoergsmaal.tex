\chapter{Prototype Interviews}\label{ch:PrototypeInterview}
Dette er skemaet der blev udfyldt i forbindelse med interviewene med prototyper.

\textbf{Konceptuelle spørgmsål}

Systemet skal kunne hjælpe på problemer, der kan opstå i forbindelse med indkøb og madlavning i hjemmet.
Hvilke opgaver i denne process skal sådan et system kunne hjælpe med, eller løse for at være nyttigt for dig?

Hvorfor disse valg? Var det det første du tænkte på? Er det et problem du ofte har?

De funktionaliteter vi selv har tænkt på er: indkøbslister, opskrifter, aktuelle tilbudsvarer og overvågning af tilbud, samt anbefaler varer og/eller opskrifter, baseret på tidligere valg, bedømmelser af opskrifter og præferencer ang. madvarer.

Finder du de funktioner vi har tænkt på, som du ikke selv nævnte irrelevante? - eller forestiller du dig at de kunne bruges?

Hvordan tænker du at du ville tilgå sådan et system?

Eventuelt andre tanker om en sådan løsning?

\textbf{Prototype fremvisning}

Vi vil nu præsentere dig for en tidlig prototype, hvor vi har fokus på funktionaliteten i løsningen, og der er altså ikke tænkt meget over det grafiske.
Vi vil gerne bede dig om at tænke højt under udførelsen. Dvs. når du går ind i et skærmvindue, fortæller du inden du trykker, hvad du regner med du vil finde på skærmbilledet, og når du så har trykket, reflektere du over hvad du ser.

Fremgangsmåde: Lad brugeren teste prototypen og trykke rundt mellem menuerne. Spørg løbende brugeren hvad han/hun forventer af de forskellige “skærme” og funktioner, og spørg efterfølgende hvordan den faktiske skærm levede op til forventningerne.  

Må vi kontakte dig med yderligere undersøgelser?

Kontakt info:

