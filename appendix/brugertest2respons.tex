\chapter{Brugertest 2 respons}\label{b:brugertestrespons2}
Herunder ses den samlede responn for vores syv testpersoner i brugertest 2.
Ud af de 7 testpersoner har 2 af dem ligeledes deltaget i brugertest 1, disse er bruger 4 og 5.

\section{Opgaver}
\subsection{Opgave 1}
\textbf{Oprette bruger}

Opretter bruger uden problemer [1][2][3][4][5]

Har probler med at finde opret bruger-linket og ender på login siden [6][7]

\textbf{Fravælg butikker og evt. varer}

Fravælger butikker og varer uden problemer [3][4][5][6]

Fravælger butikker og varer med få problemer [1][2][7]
\begin{itemize}
	\item Trykker på ''gem'' ved navne-felt, efter at have fravalgt butikker [1]
	\item Tror at alle butikker er fravalgt som default [2]
	\item Naviger forkert til start. Er lidt i tvivl om den selv gemmer [7]
\end{itemize}

\subsection{Opgave 2}
\textbf{Udfylde indkøbsliste og finde tilbud på nogle af disse varer}

Gennemfører dette uden problemer 

\textbf{Ændre mængden på den først tilføjede vare}


\textbf{Dele indkøbslisten med test@mail.dk}

\subsection{Opgave 3} 
\textbf{Gennemse ugens tilbud og vælge nogle tilbud. Tilføj disse til sin indkøbsliste}

\subsection{Opgave 4}
\textbf{Find Pulled Pork opskrift}

\textbf{Ret til 8 personer}

\textbf{Tilføj varer til din indkøbsliste}

\subsection{Opgave 5}
\textbf{Vurder opskrift}

\textbf{Lav egen version af Pulled Pork-opskrift med chili}

\textbf{Find en anden opskrift som du tror du vil kunne lide}

\subsection{Opgave 6}
\textbf{tilføj en eller flere varer tl din indkøbsliste}

\textbf{Naviger ud på forsiden, og læg mærke til ændringerne}

\subsection{Opgave 7}
\textbf{Find indkøbslisten på mobilen og afkryds varer derpå}

\subsection{Opgave 8}
\textbf{Find din egen opskrift på mobilen og beskriv layoutet for opskriften}


\subsection{Opfølgende spørgsmål}
Første gang vi udførte tests fandt testpersonerne følgende problemstillinger:
Det er forvirrende om tilbudene i tilbudsfanen er aktive tilbud eller ej.
 - Brugeren foreslår at tilbuddene kune sorteres kronologisk efter hvornår de udløber.
Hvordan ser man detaljerne om en opskrift?
 - Opskrifte-fanen er intuitiv, det er lidt at se hvad man kan og hvordan.
Det tager lang tid at tilføje ingredienserne til sin indkøbsliste.
 - Tager ikke lang tid, mener brugeren.
Det er svært at vide om indstillingerne vedr. butikker og varer, man ikke vil have, bliver gemt.
 - Snakbar kunne poppe op midtpå. Som default kan det være svært at se om buikker er valgt fra eller til.
\subsubsection{kommentarer til forskellige programdele}
– Forside
	Billeder på opskifter
– Tilbud
	Mulighed for at se tilbud fra butikker man har valgt fra.
– Opskrifter
	Billeder på opskrifter
	Tid kunne stå i timer
	Overskuelig
– Indkøbslister
	Forvirret over at forskellen på delt og personlig, at en personlig liste kan flyttes til delt liste. Ved ikke om varene ryger med.
– Indstillinger
Synes at kodeord burde stå øverst.
checkboxe for almindelige allergerner
– Overvågningslisten
	Datoer for hvornår tilbud udløber.

Hvordan synes du det var at bruge hjemmesiden?\\
 - Ikoner fungerer godt. \\
 - De forskellige farver giver god mening.\\
Kunne du finde på at benytte dig af hjemmesiden, hvis den blev udgivet? - Hvorfor, hvorfor ikke?\\
 - Ja\\
Var der noget ved hjemmesidens funktionalitet eller udseende som du ikke forstod?
 - Dropdown for tilbud er godt, men lidt svær at finde
 - Ændre mængde på varer kan være lidt svær at finde
Ville du kunne finde på at bruge hjemmesidens anbefalingsfunktionalitet mht. opskrifter?
 - Kunne godt tænke sig anbefaling i forhold til tid
Hvad synes du om tekstmængden på de forskellige sider.
 - Brugte det ikke, men mener det ville være godt til ældre der ikke kender ikoner.
Er der nogle ting du føler kunne gøres bedre?
 - Allergi kunne være godt
 - Billeder på opskrifter
Hvor nemme var opgaverne? Var opgaverne realistiske?
 - Til at forstå, ikke svære
Var det forstyrrende at du skulle tænke højt?
 - En lille smule


