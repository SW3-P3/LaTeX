\chapter{Brugertest 2 respons}\label{b:brugertestrespons2}
Herunder ses den samlede responn for vores syv testpersoner i brugertest 2.
Ud af de 7 testpersoner har 2 af dem ligeledes deltaget i brugertest 1, disse er bruger 4 og 5.

\section{Opgaver}
\subsection{Opgave 1}
\textbf{Oprette bruger}

Opretter bruger uden problemer [1][2][3][4][5]

Har probler med at finde opret bruger-linket og ender på login siden [6][7]

\textbf{Fravælg butikker og evt. varer}

Fravælger butikker og varer uden problemer [3][4][5][6]

Fravælger butikker og varer med få problemer [1][2][7]
\begin{itemize}
	\item Trykker på ''gem'' ved navne-felt, efter at have fravalgt butikker [1]
	\item Tror at alle butikker er fravalgt som default [2]
	\item Naviger forkert til start. Er lidt i tvivl om den selv gemmer [7]
\end{itemize}

\subsection{Opgave 2}
\textbf{Udfylde indkøbsliste og finde tilbud på nogle af disse varer}

Gennemfører dette uden problemer [1][2][5][7]

Gennemfører dette med en smule besvær [3][4][6]
\begin{itemize}
	\item Har ikke set man kan ændre enheder, for de ting man tilføjer til indkøbslisten [3]
	\item Er i tvivl om man bruger ''+'' til at tilføje et tilbud, til sin var på indkbslisten [4]
	\item Tilføjer varer uden problemer, finder ikek tilbudene på dem, til start [6]
\end{itemize}

\textbf{Ændre mængden på den først tilføjede vare}

Ændre mængden uden problemer [1][2][3][4][6][7]

En smule problemer med at finder der hvor man ændre mængde for en vare [5]

\textbf{Dele indkøbslisten med test@mail.dk}

Gør dette uden problemer [1][2][3][4][5][7]

Har problemer et kort øjeblik med at finde dele-boksen [6] 

\subsection{Opgave 3} 
\textbf{Gennemse ugens tilbud og vælge nogle tilbud. Tilføj disse til sin indkøbsliste}

Gennemfører dette helt uden besvær [1][6][7]

Er lidt i tvivl om varene er tilføjet til indkøbslisten [2][3]

Tror først at man skal tage valg, for at finde de tilbud, der er aktive [4]

Er i tvivl om hvilken indkøbslisten tilbudene er blevet tilføjet til [5]

\subsection{Opgave 4}
\textbf{Find Pulled Pork opskrift}

Alle bruger klarer dette uden problemer.
4 testere finder den ved at scrolle, 3 benytter sig af søgefunktionen.

\textbf{Ret til 8 personer}
Denne test bliver også udført at alle testpersonerne. 

Enkelt bruger kommenterer at knapperne er små og svære at ramme. [1]

\textbf{Tilføj varer til din indkøbsliste}

Alle testpersoner klarer også denne opgave hurtigt. 
6 af dem tilføer alle varerne til indkøbslisten.

En enkelt bruger vælger funktionen hvor han kun tilføjer nogle af varerne. [5]

En bruger kommenterer at denne funktionalitet ligger langt nede på siden, og derfor er en smule svær at finde [6]

\subsection{Opgave 5}
\textbf{Vurder opskrift}

Klarer denne opgave uden problemer [1][2][3][4][5][6]

Overser denne opgave, udfører den dog efter uden problemer, efter næste opgave [7]   

\textbf{Lav egen version af Pulled Pork-opskrift med chili}

Klarer denne opgave uden nogle problemer [2][3][5][7]

Klarer denne opgave med lidt besvær [1][4][6]
\begin{itemize}
	\item Problemer med at mængden skal være et tal [1]
	\item Kommenterer at det er lidt svært at se felterned ingrediens, mængde og enhed hænger sammen [4]
	\item Gemmer den nye opksift uden at tilføje chili, gør dette efterfølgende [6]
\end{itemize}

\textbf{Find en anden opskrift som du tror du vil kunne lide}

Alle brugere klarer dette uden problemer, dog ikke som vi tiltænkt os.

\subsection{Opgave 6}
\textbf{tilføj en eller flere varer tl din overvågningsliste}

Alle testpersoner klarer også denne test, dog har en enkelt testperson en kommentar: "Det giver ikke mening at jeg skal vælge indkøbsliste for overvågningslisten." [4] 

\textbf{Naviger ud på forsiden, og læg mærke til ændringerne}

Samtlige testpersoner navigerer ud på forsiden uden problemer.

\subsection{Opgave 7}
\textbf{Find indkøbslisten på mobilen og afkryds varer derpå}
Alle brugerer finder deres indkøbsliste på mobiltelefonen.

Samtlige af testpersonerne benytter sig dog af en funktion vi havde tiltænkt til at slette varer, i stedet for den planlagte overstregningsfunktion.

\subsection{Opgave 8}
\textbf{Find din egen opskrift på mobilen og beskriv layoutet for opskriften}

Alle testpersoner navigerer også uden problemer til sin tidligere opretttede opskfift, på mobiltelefonen.

\subsection{Opfølgende spørgsmål}
Desuden fik vi en del god respons på de forskellige programdele, under de opfølgende spørgsmål, efter forstående testopgaver.
Denne respons er opsummeret herunder.
Først spurgtes ind til de problemer vi fandt i brugertest 1, for at se om de individuelle testpersoner, føler disse fejl er løst. 
Her er især person 4 og 5 udsagn vigtige, fordi de begge var med i den tidligere brugertest.

Efter disse, er der opsummeret generel feedback på programmet og kort feedback på selve testen.

\subsubsection{Problemer i forhold til første brugertest}
\textbf{Forvirrende om tilbudene i tilbudsfanen er aktive tilbud eller ej}
\begin{itemize}
	\item Brugeren foreslår at tilbuddene kune sorteres kronologisk efter hvornår de udløber, men har ikke problmer med at se om tilbudene er aktive [1]
	\item Ser ingen problem [2][3][7]
	\item Er forbedret ved at tilføje udløbsdato, men lidt uklart at de er aktuelle [4]
	\item Dette er ikke et problem længere [5]
	\item Er en smule i tvivl [6]
\end{itemize}
\textbf{Hvordan man ser detaljerne om en opskrift}

Ingen testpersoner ser dette som et problem og de to tidligere testpersoner beskriver dette problem som løst.

\textbf{Det tager lang tid at tilføje ingredienserne til sin indkøbsliste}

Ingen testpersoner oplever denne fejl på det nye system.

\textbf{Det er svært at vide om indstillingerne vedr. butikker og varer, man ikke vil have, bliver gemt.}

Havde ingen problemer her [2][3][4][5][6]

Har en smule problemer med se om butikker som default er valgt eller fravalgt [1]

Har en smule svært ved at se feedback når han fjerner butikker [7]

\subsubsection{Kommentarer til forskellige programdele}
\textbf{Forside}

Foreslår visning af tidligere benytte opskfriter [4]

\textbf{Tilbud}

Mulighed for at se tilbud fra butikker man har valgt fra. [1]

Ønsker førpris [2]
	
Ønsker kategorier [3]

\textbf{Opskrifter}

Billeder på opskrifter [1][4]

Tid kunne stå i timer [1]

\textbf{Indkøbslister}

Utilfreds/forvirret  over opdelingen mellem personlig og delte indkøbslister [1][3][6]

\textbf{Indstillinger}

Ønsker mulighed for at frasorterer varer efter alllegener

\textbf{Overvågningslisten}
Mangler datoer for hvornår tilbud udløber [1]

\textbf{Hjemmesiden generelt}

Positiv feedback, fra alle deltagere

\textbf{Kunne du finde på at benytte dig af hjemmesiden, hvis den blev udgivet?}

Ja [1][3][6][7]

I nogle situatiioner [2][4][5]

\textbf{Var der noget ved hjemmesidens funktionalitet eller udseende som du ikke forstod?}

Ingen testpersoner havde kommentarer til andet end de have nævnt tidligere i testen.

\textbf{Ville du kunne finde på at bruge hjemmesidens anbefalingsfunktionalitet mht. opskrifter?}

Ja [1][2][3][6][7]

\textbf{Hvad synes du om tekstmængden på de forskellige sider?}

Alle brugere udtrykte at de ikke læste dette, enkelte testere læste nogle af teksterne, hvis de var i tvivl om noget.

\textbf{Foreslag til forbedringer}

Forslagene til forberinger var alle sammen forslag som var forekommet tidligere i testen 

\subsubsection{Feedback på testen}
Hvor nemme var opgaverne? Var opgaverne realistiske?

Samltlige brugere udtrykte at testen føltes realistisk og opgaverne var lette at udfører

Var det forstyrrende at du skulle tænke højt?

Nej [2][3][4][5]

Ja, lidt [1][6][7]
