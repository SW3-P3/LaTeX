\chapter{Brugertest respons}\label{b:brugertestrespons}
Her følger en opsamling på data, fået fra testen som kan findes på \myref{b:brugertest}.
Dette er delt ind i seks kategorier, fire til hvert testet punkt af systemet, ''Indkøbsliste'', ''Tilbud'', ''Opskrifter'' og ''Præferencer'', ydermere er der kommentar givet efter brugertesten, såvel som observationer lavet under testens forløb.

\textbf{Indkøbsliste} 
\begin{itemize}
   \item Mere end halvdelen af brugerne blev først opmærksomme på at der kunne ses tilbud på en given vare på listen, når de tilføjede en vare med et aktivt tilbud.
   \item En mindre del af de testede havde problemer med at gå ind på indkøbslisten, efter dens oprettelse.
\end{itemize}

\textbf{Tilbud} 
\begin{itemize}
   \item En større del af testpersonerne var i tvivl om hvorvidt tilbuddene var aktuelle på det tidspunkt brugeren benyttede systemet.
   \item Halvdelen af testpersonerne, ønskede en kategorisering af vare.
   \item Enkelte af de testede ville gerne kunne se kiloprisen, faktisk pris og mængde.
   \item En person havde problemer med at bruge tilbudsfanen, da det ikke gik op for vedkommende at en indkøbsliste skal være oprettet.
\end{itemize}

\textbf{Opskrifter} 
\begin{itemize}
   \item Alle de testede havde problemer med at komme ind på en individuel opskrift.
   \item Næsten alle de testede så gerne at man kunne tilføje alle ingredienser med én knap.
   \item Nogle af de testede, ville gerne kunne se en estimeret pris på opskriften.
   \item Få af de testede ville gerne kunne fjerne vare fra indkøbslisten igennem opskriften.
   \item Enkelte ønskede at kunne se om en ingrediens var på tilbud.
   \item Enkelte ønskede en visuelt markering af stjerner når man skulle til at afgive sin vurdering.
   \item En person ønskede at man ikke behøvede at redigere alt på en opskrift for at ændre en enkelt ting, redigerengen er på tiden for testen, opdelt i to faser fremgangsmåde og ingredienser.
\end{itemize}

\textbf{Præferencer} 
\begin{itemize}
   \item Alle testpersonerne mente at navnet var misvisende.
   \item Alle testpersonerne var i tvivl om hvorvidt deres ændringer i afkrydsningsfelterne, blev gemt.
   \item En person fortolkede afkrydsningsfelterne direkte i strid med deres funktionalitet.
\end{itemize}

\textbf{Kommentar til systemet} 
\begin{itemize}
   \item Alle de testede gav udtryk for at systemet kunne være brugbart, og de ville være villige til at benytte det færdige system.
   \item Alle de testede mente at det var let at benytte systemet, dog kunne enkelte stedet godt bruge lidt mere information om brug.
   \item En større del af de testede, var til tider i tvivl om deres ændringer i systemet blev gemt.
   \item Flere af de testede, mente at den sorte farve gjorde det dystert og mindre indbydende.
   \item Enkelte gjorde opmærksomme på at finde tilbud på ost, kunne være lidt træls eftersom tilbud på leverp\textbf{ost}ej blev forslået.
   \item Enkelte mente at det kunne være rart at dele sine lister med andre.
\end{itemize}

\textbf{Observatørs kommentar} 
\begin{itemize}
   \item Ingen af de testede blev opmærksomme på aftjekningsfunktionaliteten under indkøbslisten.
   \item Det er ikke klart for brugeren, at en mængde ikke er påkrævet når man skriver en vare på indkøbslisten.
   \item Næsten ingen ændrer navnet når de redigerer en opskrift.
   \item Sider hvorpå man skal vælge en indkøbsliste, tilbud og opskrifter, er funktionaliteten fuldstændig overset.
\end{itemize}