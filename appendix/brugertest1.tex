\chapter{Bruger Test 1}\label{b:brugertest1}

Dette er en undersøgelse af vores system. Din opgave er, som udefrakommende at teste hvor brugervenligt systemet er. Du bedes gennemlæse hver opgave fuldstændigt og fortælle grupperepræsentanten hvad du forstår du skal i opgaven. Hvis der er eventuelle uklarheder i opgaven, vil han udrede disse. Vi vil gerne bede dig tænkt højt, så vi forstår hvorfor du gør hvad du gør når du arbejder med hjemmesiden. Dvs. hvis du klikker på en knap, vil vi gerne vide hvorfor du klikker på denne knap. Når du mener at du har afsluttet opgaven bedes du erklære det til grupperepræsentanten.

1.

Du har problemer i hverdagen med at holde styr på dine tilbud, og overskue ugens tilbud, og vil derfor lave en bruger på ProjectFood.
Du handler ikke i alle butikker i byen, og vil derfor afkrydse hvilke butikker du handler i, så du ikke kan se tilbud fra alle byens butikker.

\textbf{Opret en bruger sæt dine præferencer for butikker du handler i. }

2.

Du skal forberede indkøbene til i morgen, og vil derfor lave en indkøbsliste for dagen i morgen imens du finder nogle tilbud på de ting du mangler.

\textbf{Udfyld din indkøbsliste med varer og nogle tilbud for disse varer.}

3.

Vi vil bede dig forestille dig at du er ude og handle, og ville vise at du har sat en vare på indkøbslisten ned i din kurv, på papir lister streger man dem normalt ud.

4.
Du er interesseret i at se hvilke tilbud der er denne uge, og kigger derfor på tilbudene fra de forskellige butikker, og vælger nogle derfra.

\textbf{Browse ugens tilbud, og vælg nogle du er interesseret i og tilføj dem til din indkøbsliste.}

5.

Du får besøg af din store familie i weekenden, og tænker at det ville være smart at lave forloren hare til 16 personer.

\textbf{Find en opskrift på forloren hare til 16 personer, og gør din indkøbsliste klar til at handle ind til dette.}

6.

Du har nu lavet opskriften og var begejstret for denne. Derfor vælger du at tildele en vurdering til opskriften.
Du mente dog også at opskriften kunne bruge lidt hvidløg for at krydre den lidt ekstra.

\textbf{Tildel opskriften en vurdering, og lav desuden din egen variation af opskriften med ekstra hvidløg.}


\textbf{Opfølgende spørgsmål}

\begin{itemize}
	\item Spørg ud for hvert emne:
		\begin{itemize}
			\item Tilbud
			\item Opskrifter
			\item Indkøbslister
			\item Indstillinger
		\end{itemize}
	Var der hvad du forventede under dette emne? - Hvad fungerede godt, hvad fungerede ikke godt?
	\item Hvordan synes du det var at bruge hjemmesiden?
	\item Kunne du finde på at benytte dig af hjemmesiden, hvis den blev udgivet? - Hvorfor, hvorfor ikke?
	\item Var der noget ved hjemmesidens funktionalitet eller udseende som du ikke forstod?
	\item Er der nogle ting du føler kunne gøres bedre?
	\item Hvor nemme var opgaverne?
	\item Var opgaverne realistiske?
	\item Var det forstyrrende at du skulle tænke højt?
\end{itemize}	

Tak fordi du ville deltage i testen, den har været meget behjælpelig.

